% --- LaTeX Homework Template - S. Venkatraman ---

% --- Set document class and font size ---

\documentclass[letterpaper, 11pt]{article}

% --- Package imports ---

\usepackage{
  amsmath, amsthm, amssymb, mathtools, dsfont,	  % Math typesetting
  graphicx, wrapfig, subfig, float,                  % Figures and graphics formatting
  listings, color, inconsolata, pythonhighlight,     % Code formatting
  fancyhdr, sectsty, hyperref, enumerate, enumitem } % Headers/footers, section fonts, links, lists

% --- Page layout settings ---

% Set page margins
\usepackage[left=1.35in, right=1.35in, bottom=1in, top=1.1in, headsep=0.2in]{geometry}

% Anchor footnotes to the bottom of the page
\usepackage[bottom]{footmisc}

% Set line spacing
\renewcommand{\baselinestretch}{1.2}

% Set spacing between paragraphs
\setlength{\parskip}{1.5mm}

% Allow multi-line equations to break onto the next page
\allowdisplaybreaks

% Enumerated lists: make numbers flush left, with parentheses around them
\setlist[enumerate]{wide=0pt, leftmargin=21pt, labelwidth=0pt, align=left}
\setenumerate[1]{label={(\arabic*)}}

% --- Page formatting settings ---

% Set link colors for labeled items (blue) and citations (red)
\hypersetup{colorlinks=true, linkcolor=blue, citecolor=red}

% Make reference section title font smaller
\renewcommand{\refname}{\large\bf{References}}

% --- Settings for printing computer code ---

% Define colors for green text (comments), grey text (line numbers),
% and green frame around code
\definecolor{greenText}{rgb}{0.5, 0.7, 0.5}
\definecolor{greyText}{rgb}{0.5, 0.5, 0.5}
\definecolor{codeFrame}{rgb}{0.5, 0.7, 0.5}

% Define code settings
\lstdefinestyle{code} {
  frame=single, rulecolor=\color{codeFrame},            % Include a green frame around the code
  numbers=left,                                         % Include line numbers
  numbersep=8pt,                                        % Add space between line numbers and frame
  numberstyle=\tiny\color{greyText},                    % Line number font size (tiny) and color (grey)
  commentstyle=\color{greenText},                       % Put comments in green text
  basicstyle=\linespread{1.1}\ttfamily\footnotesize,    % Set code line spacing
  keywordstyle=\ttfamily\footnotesize,                  % No special formatting for keywords
  showstringspaces=false,                               % No marks for spaces
  xleftmargin=1.95em,                                   % Align code frame with main text
  framexleftmargin=1.6em,                               % Extend frame left margin to include line numbers
  breaklines=true,                                      % Wrap long lines of code
  postbreak=\mbox{\textcolor{greenText}{$\hookrightarrow$}\space} % Mark wrapped lines with an arrow
}

% Set all code listings to be styled with the above settings
\lstset{style=code}

% --- Math/Statistics commands ---

% Add a reference number to a single line of a multi-line equation
% Usage: "\numberthis\label{labelNameHere}" in an align or gather environment
\newcommand\numberthis{\addtocounter{equation}{1}\tag{\theequation}}

% Shortcut for bold text in math mode, e.g. $\b{X}$
\let\b\mathbf

% Shortcut for bold Greek letters, e.g. $\bg{\beta}$
\let\bg\boldsymbol

% Shortcut for calligraphic script, e.g. %\mc{M}$
\let\mc\mathcal

% \mathscr{(letter here)} is sometimes used to denote vector spaces
\usepackage[mathscr]{euscript}

% Convergence: right arrow with optional text on top
% E.g. $\converge[w]$ for weak convergence
\newcommand{\converge}[1][]{\xrightarrow{#1}}

% Normal distribution: arguments are the mean and variance
% E.g. $\normal{\mu}{\sigma}$
\newcommand{\normal}[2]{\mathcal{N}\left(#1,#2\right)}

% Uniform distribution: arguments are the left and right endpoints
% E.g. $\unif{0}{1}$
\newcommand{\unif}[2]{\text{Uniform}(#1,#2)}

% Independent and identically distributed random variables
% E.g. $ X_1,...,X_n \iid \normal{0}{1}$
\newcommand{\iid}{\stackrel{\smash{\text{iid}}}{\sim}}

% Equality: equals sign with optional text on top
% E.g. $X \equals[d] Y$ for equality in distribution
\newcommand{\equals}[1][]{\stackrel{\smash{#1}}{=}}

% Math mode symbols for common sets and spaces. Example usage: $\R$
\newcommand{\R}{\mathbb{R}}   % Real numbers
\newcommand{\C}{\mathbb{C}}   % Complex numbers
\newcommand{\Q}{\mathbb{Q}}   % Rational numbers
\newcommand{\Z}{\mathbb{Z}}   % Integers
\newcommand{\N}{\mathbb{N}}   % Natural numbers
\newcommand{\F}{\mathcal{F}}  % Calligraphic F for a sigma algebra
\newcommand{\El}{\mathcal{L}} % Calligraphic L, e.g. for L^p spaces

% Math mode symbols for probability
\newcommand{\pr}{\mathbb{P}}    % Probability measure
\newcommand{\E}{\mathbb{E}}     % Expectation, e.g. $\E(X)$
\newcommand{\var}{\text{Var}}   % Variance, e.g. $\var(X)$
\newcommand{\cov}{\text{Cov}}   % Covariance, e.g. $\cov(X,Y)$
\newcommand{\corr}{\text{Corr}} % Correlation, e.g. $\corr(X,Y)$
\newcommand{\B}{\mathcal{B}}    % Borel sigma-algebra

% Other miscellaneous symbols
\newcommand{\tth}{\text{th}}	% Non-italicized 'th', e.g. $n^\tth$
\newcommand{\Oh}{\mathcal{O}}	% Big-O notation, e.g. $\O(n)$
\newcommand{\1}{\mathds{1}}	% Indicator function, e.g. $\1_A$

% Additional commands for math mode
\DeclareMathOperator*{\argmax}{argmax}    % Argmax, e.g. $\argmax_{x\in[0,1]} f(x)$
\DeclareMathOperator*{\argmin}{argmin}    % Argmin, e.g. $\argmin_{x\in[0,1]} f(x)$
\DeclareMathOperator*{\spann}{Span}       % Span, e.g. $\spann\{X_1,...,X_n\}$
\DeclareMathOperator*{\bias}{Bias}        % Bias, e.g. $\bias(\hat\theta)$
\DeclareMathOperator*{\ran}{ran}          % Range of an operator, e.g. $\ran(T) 
\DeclareMathOperator*{\dv}{d\!}           % Non-italicized 'with respect to', e.g. $\int f(x) \dv x$
\DeclareMathOperator*{\diag}{diag}        % Diagonal of a matrix, e.g. $\diag(M)$
\DeclareMathOperator*{\trace}{trace}      % Trace of a matrix, e.g. $\trace(M)$

% Numbered theorem, lemma, etc. settings - e.g., a definition, lemma, and theorem appearing in that 
% order in Section 2 will be numbered Definition 2.1, Lemma 2.2, Theorem 2.3. 
% Example usage: \begin{theorem}[Name of theorem] Theorem statement \end{theorem}
\theoremstyle{definition}
\newtheorem{theorem}{Theorem}[section]
\newtheorem{proposition}[theorem]{Proposition}
\newtheorem{lemma}[theorem]{Lemma}
\newtheorem{corollary}[theorem]{Corollary}
\newtheorem{definition}[theorem]{Definition}
\newtheorem{example}[theorem]{Example}
\newtheorem{remark}[theorem]{Remark}

% Un-numbered theorem, lemma, etc. settings
% Example usage: \begin{lemma*}[Name of lemma] Lemma statement \end{lemma*}
\newtheorem*{theorem*}{Theorem}
\newtheorem*{proposition*}{Proposition}
\newtheorem*{lemma*}{Lemma}
\newtheorem*{corollary*}{Corollary}
\newtheorem*{definition*}{Definition}
\newtheorem*{example*}{Example}
\newtheorem*{remark*}{Remark}
\newtheorem*{claim}{Claim}

% --- Left/right header text (to appear on every page) ---

% Include a line underneath the header, no footer line
\pagestyle{fancy}
\renewcommand{\footrulewidth}{0pt}
\renewcommand{\headrulewidth}{0.4pt}

% Left header text: course name/assignment number
\lhead{MATH-SHU 282 (Complex) -- Homework 1}

% Right header text: your name
\rhead{Yixia Yu (yy5091@nyu.edu)}

% --- Document starts here ---

\begin{document}
\subsection*{Problem 1}
Compute $ \sqrt[4]{i} \text{ and } \sqrt[4]{-i} $ 
\begin{proof}
    by Euler's formula, we have $$
        i=e^{i(\frac{\pi}{2}+2k\pi)},k\in\mathbb{Z}
    $$
    Suppose $ z^4=i $ ,we have \begin{align*}{}{}
    z^4&=e^{i(\frac{\pi}{2}+2k\pi)}\\
    z&=e^{i(\frac{\pi}{8}+\frac{k\pi}{2})}\\
    z&=\cos(\frac{\pi}{8}+\frac{k\pi}{2})+i\sin(\frac{\pi}{8}+\frac{k\pi}{2}),k\in\{0,1,2,3,\cdots\}
    \end{align*}
similarly, we have $$
    -i=e^{i(\frac{3\pi}{2}+2k\pi)},k\in\mathbb{Z}
$$
Suppose $ w^4=-i $ ,we have \begin{align*}{}{}
    w^4&=e^{i(\frac{3\pi}{2}+2k\pi)}\\
    w&=e^{i(\frac{3\pi}{8}+\frac{k\pi}{2})}\\
    w&=\cos(\frac{3\pi}{8}+\frac{k\pi}{2})+i\sin(\frac{3\pi}{8}+\frac{k\pi}{2}),k\in\{0,1,2,3,\cdots\}
    \end{align*}
\end{proof}
\subsection*{Problem 2}
Solve the quadratic equation $$
    z^2+(\alpha+i\beta)z+\gamma+i\delta=0
$$ 
\textbf{solution:}
\\by the quadratic formula, we have 
\[
z = \frac{- (\alpha + i \beta) \pm \sqrt{(\alpha + i\beta)^2 - 4(\gamma + i\delta)}}{2}.
\]
and \[
\Delta = (\alpha + i\beta)^2 - 4(\gamma + i\delta).
\]
\subsection*{Problem 3}
Show that the system of all matrices$$
    \begin{pmatrix}
        \alpha & \beta \\
        -\beta & \alpha
    \end{pmatrix}
$$ with the standard matrix addition and scalar multiplication is isomorphic to $ \mathbb{C} $ 
\begin{proof}
    Define the function 
\[
\varphi\colon \mathbb{C} \to S,\quad \varphi(a+bi) = \begin{pmatrix} a & b \\ -b & a \end{pmatrix},
\]
where $a,b \in \mathbb{R}$ and $i$ is the imaginary unit. Denote the set of all matrices of the form $\begin{pmatrix} \alpha & \beta \\ -\beta & \alpha \end{pmatrix}$ by $S$.

\textbf{1. $\varphi$ is Well-Defined and Linear}\\
For any two complex numbers $z = a+bi$ and $w = c+di$, we have:
\[
\varphi(z+w) = \varphi\bigl((a+c)+(b+d)i\bigr) = \begin{pmatrix} a+c & b+d \\ -(b+d) & a+c \end{pmatrix}.
\]
On the other hand,
\[
\varphi(z) + \varphi(w) = \begin{pmatrix} a & b \\ -b & a \end{pmatrix} + \begin{pmatrix} c & d \\ -d & c \end{pmatrix} = \begin{pmatrix} a+c & b+d \\ -b-d & a+c \end{pmatrix}.
\]
Thus, $\varphi(z+w) = \varphi(z) + \varphi(w)$, showing that $\varphi$ preserves addition.

Similarly, for any real scalar $k$,
\[
\varphi(kz) = \varphi(ka+kbi) = \begin{pmatrix} ka & kb \\ -kb & ka \end{pmatrix} = k \begin{pmatrix} a & b \\ -b & a \end{pmatrix} = k\,\varphi(z),
\]
which shows $\varphi$ is linear with respect to scalar multiplication.

\textbf{2. $\varphi$ Preserves Multiplication}\\
For $z = a+bi$ and $w = c+di$, note that
\[
z\cdot w = (a+bi)(c+di) = (ac-bd) + (ad+bc)i.
\]
Then,
\[
\varphi(z\cdot w) = \varphi\bigl((ac-bd) + (ad+bc)i\bigr) = \begin{pmatrix} ac-bd & ad+bc \\ -(ad+bc) & ac-bd \end{pmatrix}.
\]
Now, compute the product $\varphi(z)\,\varphi(w)$:
\[
\varphi(z)\,\varphi(w) = \begin{pmatrix} a & b \\ -b & a \end{pmatrix}\begin{pmatrix} c & d \\ -d & c \end{pmatrix} 
= \begin{pmatrix} ac-bd & ad+bc \\ -(ad+bc) & ac-bd \end{pmatrix}.
\]
Since the two products are equal, $\varphi(z\cdot w) = \varphi(z)\,\varphi(w)$; hence, $\varphi$ preserves multiplication.

\textbf{3. $\varphi$ is Bijective}\\
\textbf{Injectivity:}  
Suppose $\varphi(a+bi)=\varphi(c+di)$. Then,
\[
\begin{pmatrix} a & b \\ -b & a \end{pmatrix} = \begin{pmatrix} c & d \\ -d & c \end{pmatrix}.
\]
By comparing corresponding entries, we get $a = c$ and $b = d$. Therefore, $a+bi = c+di$, and $\varphi$ is injective.

\textbf{Surjectivity:}  
Take any matrix in $S$ of the form
\[
\begin{pmatrix} \alpha & \beta \\ -\beta & \alpha \end{pmatrix},
\]
where $\alpha, \beta \in \mathbb{R}$. Choosing $z = \alpha + \beta i \in \mathbb{C}$, we observe that
\[
\varphi(z) = \begin{pmatrix} \alpha & \beta \\ -\beta & \alpha \end{pmatrix}.
\]
Thus, every element of $S$ has a preimage in $\mathbb{C}$, proving that $\varphi$ is surjective.
\\Hence, the set
\[
S = \left\{ \begin{pmatrix} \alpha & \beta \\ -\beta & \alpha \end{pmatrix} : \alpha,\, \beta \in \mathbb{R} \right\}
\]
is isomorphic to the field of complex numbers, $\mathbb{C}$.

\[
{\mathbb{C} \cong S,\quad \varphi(a+bi) = \begin{pmatrix} a & b \\ -b & a \end{pmatrix}}
\]

\end{proof}
\subsection*{Problem 4}
Prove that $$
    |\frac{a-b}{1-\bar{a}b}|=1
$$ if either |a|=1 or |b|=1
\begin{proof}
Suppose that $|a|=1$. Then, we can write $a=e^{i\theta}$.So, we have
\begin{align*}{}{}
|\frac{e^{i\theta}-b}{1-e^{-i\theta}b}|=|\frac{e^{i\theta}-b}{1-e^{-i\theta}b}||e^{-i\theta}|=|\frac{e^{-i\theta}e^{i\theta}(e^{i\theta}-b)}{e^{i\theta}-b}|=1
\end{align*}
Suppose that $ |b|=1 $  write 
\(
b = e^{i\theta} 
\)
Then, the expression becomes
\[
\frac{a-b}{1-\overline{a}b} = \frac{a-e^{i\theta}}{1-\overline{a}e^{i\theta}}.
\]

Now, compute the squared moduli of the numerator and denominator.

\textbf{Numerator:}
\[
|a-e^{i\theta}|^2 = (a-e^{i\theta})(\overline{a}-e^{-i\theta})
= |a|^2 - a e^{-i\theta} - e^{i\theta}\overline{a}+1.
\]

\textbf{Denominator:}
\[
|1-\overline{a}e^{i\theta}|^2 = (1-\overline{a}e^{i\theta})(1-a e^{-i\theta})
= 1 - a e^{-i\theta} - e^{i\theta}\overline{a}+|a|^2.
\]

Since the two squared moduli are equal, it follows that
\[
\left|\frac{a-e^{i\theta}}{1-\overline{a}e^{i\theta}}\right| = \frac{|a-e^{i\theta}|}{|1-\overline{a}e^{i\theta}|} = 1.
\]

Thus, we conclude that
\[
\left|\frac{a-b}{1-\overline{a}b}\right| = 1.
\]
\end{proof}
\subsection*{Problem 5}
Prove that $$
    |\frac{a-b}{1-\bar{a}b}|<1
$$ if |a|<1 and |b|<1
\begin{proof}
    Write
\[
a = r_1e^{i\alpha}, \quad b = r_2e^{i\beta}, \quad \text{with } r_1,r_2<1.
\]
Multiplying the numerator and denominator of
\[
\frac{a-b}{1-\overline{a}b}
\]
by \(e^{-i\alpha}\) (which does not change the modulus), we obtain
\[
\frac{a-b}{1-\overline{a}b} = \frac{r_1 - r_2e^{i(\beta-\alpha)}}{1- r_1r_2e^{i(\beta-\alpha)}}.
\]
Set
\[
\theta = \beta-\alpha,
\]
so that
\[
\left|\frac{a-b}{1-\overline{a}b}\right| = \left|\frac{r_1 - r_2e^{i\theta}}{1- r_1r_2e^{i\theta}}\right|.
\]
Taking the squared modulus gives
\[
\left|\frac{r_1 - r_2e^{i\theta}}{1- r_1r_2e^{i\theta}}\right|^2
=\frac{|r_1 - r_2e^{i\theta}|^2}{|1- r_1r_2e^{i\theta}|^2}
=\frac{r_1^2 + r_2^2 - 2r_1r_2\cos\theta}{1 + r_1^2r_2^2 - 2r_1r_2\cos\theta}.
\]
Since
\[
r_1^2 + r_2^2 < 1 + r_1^2r_2^2 \quad\Longrightarrow\quad r_1^2 + r_2^2 - 2r_1r_2\cos\theta < 1 + r_1^2r_2^2 - 2r_1r_2\cos\theta,
\]
we conclude that
\[
\frac{r_1^2 + r_2^2 - 2r_1r_2\cos\theta}{1 + r_1^2r_2^2 - 2r_1r_2\cos\theta} < 1.
\]
That is,
\[
\left|\frac{a-b}{1-\overline{a}b}\right|^2 < 1,
\]
and hence
\[
\left|\frac{a-b}{1-\overline{a}b}\right| < 1.
\]
\end{proof}
\subsection*{Problem 6}
If $ |a_i|<1,\lambda_i\geq0,i=1,\cdots,n\text{ and }\lambda_1+\cdots+\lambda_n=1 $\\
Show that$$
    |\lambda_1a_1+\cdots+\lambda_na_n|<1
$$  
\begin{proof}
    Since for each \( i=1,\dots,n \) we have \(|a_i| < 1\) and \(\lambda_i \ge 0\) with
\[
\lambda_1+\cdots+\lambda_n=1,
\]
by the triangle inequality,
\[
\left|\lambda_1a_1+\cdots+\lambda_na_n\right| \le \lambda_1|a_1|+\cdots+\lambda_n|a_n|.
\]
Because each \(|a_i|<1\), it follows that
\[
\lambda_1|a_1|+\cdots+\lambda_n|a_n| < \lambda_1+\cdots+\lambda_n = 1.
\]
Thus, we obtain
\[
\left|\lambda_1a_1+\cdots+\lambda_na_n\right| < 1.
\]
\end{proof}
\end{document}