% --- LaTeX Homework Template - S. Venkatraman ---

% --- Set document class and font size ---

\documentclass[letterpaper, 11pt]{article}

% --- Package imports ---

\usepackage{
  amsmath, amsthm, amssymb, mathtools, dsfont,	  % Math typesetting
  graphicx, wrapfig, subfig, float,                  % Figures and graphics formatting
  listings, color, inconsolata, pythonhighlight,     % Code formatting
  fancyhdr, sectsty, hyperref, enumerate, enumitem } % Headers/footers, section fonts, links, lists

% --- Page layout settings ---

% Set page margins
\usepackage[left=1.35in, right=1.35in, bottom=1in, top=1.1in, headsep=0.2in]{geometry}

% Anchor footnotes to the bottom of the page
\usepackage[bottom]{footmisc}

% Set line spacing
\renewcommand{\baselinestretch}{1.2}

% Set spacing between paragraphs
\setlength{\parskip}{1.5mm}

% Allow multi-line equations to break onto the next page
\allowdisplaybreaks

% Enumerated lists: make numbers flush left, with parentheses around them
\setlist[enumerate]{wide=0pt, leftmargin=21pt, labelwidth=0pt, align=left}
\setenumerate[1]{label={(\arabic*)}}

% --- Page formatting settings ---

% Set link colors for labeled items (blue) and citations (red)
\hypersetup{colorlinks=true, linkcolor=blue, citecolor=red}

% Make reference section title font smaller
\renewcommand{\refname}{\large\bf{References}}

% --- Settings for printing computer code ---

% Define colors for green text (comments), grey text (line numbers),
% and green frame around code
\definecolor{greenText}{rgb}{0.5, 0.7, 0.5}
\definecolor{greyText}{rgb}{0.5, 0.5, 0.5}
\definecolor{codeFrame}{rgb}{0.5, 0.7, 0.5}

% Define code settings
\lstdefinestyle{code} {
  frame=single, rulecolor=\color{codeFrame},            % Include a green frame around the code
  numbers=left,                                         % Include line numbers
  numbersep=8pt,                                        % Add space between line numbers and frame
  numberstyle=\tiny\color{greyText},                    % Line number font size (tiny) and color (grey)
  commentstyle=\color{greenText},                       % Put comments in green text
  basicstyle=\linespread{1.1}\ttfamily\footnotesize,    % Set code line spacing
  keywordstyle=\ttfamily\footnotesize,                  % No special formatting for keywords
  showstringspaces=false,                               % No marks for spaces
  xleftmargin=1.95em,                                   % Align code frame with main text
  framexleftmargin=1.6em,                               % Extend frame left margin to include line numbers
  breaklines=true,                                      % Wrap long lines of code
  postbreak=\mbox{\textcolor{greenText}{$\hookrightarrow$}\space} % Mark wrapped lines with an arrow
}

% Set all code listings to be styled with the above settings
\lstset{style=code}

% --- Math/Statistics commands ---

% Add a reference number to a single line of a multi-line equation
% Usage: "\numberthis\label{labelNameHere}" in an align or gather environment
\newcommand\numberthis{\addtocounter{equation}{1}\tag{\theequation}}

% Shortcut for bold text in math mode, e.g. $\b{X}$
\let\b\mathbf

% Shortcut for bold Greek letters, e.g. $\bg{\beta}$
\let\bg\boldsymbol

% Shortcut for calligraphic script, e.g. %\mc{M}$
\let\mc\mathcal

% \mathscr{(letter here)} is sometimes used to denote vector spaces
\usepackage[mathscr]{euscript}

% Convergence: right arrow with optional text on top
% E.g. $\converge[w]$ for weak convergence
\newcommand{\converge}[1][]{\xrightarrow{#1}}

% Normal distribution: arguments are the mean and variance
% E.g. $\normal{\mu}{\sigma}$
\newcommand{\normal}[2]{\mathcal{N}\left(#1,#2\right)}

% Uniform distribution: arguments are the left and right endpoints
% E.g. $\unif{0}{1}$
\newcommand{\unif}[2]{\text{Uniform}(#1,#2)}

% Independent and identically distributed random variables
% E.g. $ X_1,...,X_n \iid \normal{0}{1}$
\newcommand{\iid}{\stackrel{\smash{\text{iid}}}{\sim}}

% Equality: equals sign with optional text on top
% E.g. $X \equals[d] Y$ for equality in distribution
\newcommand{\equals}[1][]{\stackrel{\smash{#1}}{=}}

% Math mode symbols for common sets and spaces. Example usage: $\R$
\newcommand{\R}{\mathbb{R}}   % Real numbers
\newcommand{\C}{\mathbb{C}}   % Complex numbers
\newcommand{\Q}{\mathbb{Q}}   % Rational numbers
\newcommand{\Z}{\mathbb{Z}}   % Integers
\newcommand{\N}{\mathbb{N}}   % Natural numbers
\newcommand{\F}{\mathcal{F}}  % Calligraphic F for a sigma algebra
\newcommand{\El}{\mathcal{L}} % Calligraphic L, e.g. for L^p spaces

% Math mode symbols for probability
\newcommand{\pr}{\mathbb{P}}    % Probability measure
\newcommand{\E}{\mathbb{E}}     % Expectation, e.g. $\E(X)$
\newcommand{\var}{\text{Var}}   % Variance, e.g. $\var(X)$
\newcommand{\cov}{\text{Cov}}   % Covariance, e.g. $\cov(X,Y)$
\newcommand{\corr}{\text{Corr}} % Correlation, e.g. $\corr(X,Y)$
\newcommand{\B}{\mathcal{B}}    % Borel sigma-algebra

% Other miscellaneous symbols
\newcommand{\tth}{\text{th}}	% Non-italicized 'th', e.g. $n^\tth$
\newcommand{\Oh}{\mathcal{O}}	% Big-O notation, e.g. $\O(n)$
\newcommand{\1}{\mathds{1}}	% Indicator function, e.g. $\1_A$

% Additional commands for math mode
\DeclareMathOperator*{\argmax}{argmax}    % Argmax, e.g. $\argmax_{x\in[0,1]} f(x)$
\DeclareMathOperator*{\argmin}{argmin}    % Argmin, e.g. $\argmin_{x\in[0,1]} f(x)$
\DeclareMathOperator*{\spann}{Span}       % Span, e.g. $\spann\{X_1,...,X_n\}$
\DeclareMathOperator*{\bias}{Bias}        % Bias, e.g. $\bias(\hat\theta)$
\DeclareMathOperator*{\ran}{ran}          % Range of an operator, e.g. $\ran(T) 
\DeclareMathOperator*{\dv}{d\!}           % Non-italicized 'with respect to', e.g. $\int f(x) \dv x$
\DeclareMathOperator*{\diag}{diag}        % Diagonal of a matrix, e.g. $\diag(M)$
\DeclareMathOperator*{\trace}{trace}      % Trace of a matrix, e.g. $\trace(M)$

% Numbered theorem, lemma, etc. settings - e.g., a definition, lemma, and theorem appearing in that 
% order in Section 2 will be numbered Definition 2.1, Lemma 2.2, Theorem 2.3. 
% Example usage: \begin{theorem}[Name of theorem] Theorem statement \end{theorem}
\theoremstyle{definition}
\newtheorem{theorem}{Theorem}[section]
\newtheorem{proposition}[theorem]{Proposition}
\newtheorem{lemma}[theorem]{Lemma}
\newtheorem{corollary}[theorem]{Corollary}
\newtheorem{definition}[theorem]{Definition}
\newtheorem{example}[theorem]{Example}
\newtheorem{remark}[theorem]{Remark}

% Un-numbered theorem, lemma, etc. settings
% Example usage: \begin{lemma*}[Name of lemma] Lemma statement \end{lemma*}
\newtheorem*{theorem*}{Theorem}
\newtheorem*{proposition*}{Proposition}
\newtheorem*{lemma*}{Lemma}
\newtheorem*{corollary*}{Corollary}
\newtheorem*{definition*}{Definition}
\newtheorem*{example*}{Example}
\newtheorem*{remark*}{Remark}
\newtheorem*{claim}{Claim}

% --- Left/right header text (to appear on every page) ---

% Include a line underneath the header, no footer line
\pagestyle{fancy}
\renewcommand{\footrulewidth}{0pt}
\renewcommand{\headrulewidth}{0.4pt}

% Left header text: course name/assignment number
\lhead{MATH-SHU 238 (HTOP) -- Homework 2}

% Right header text: your name
\rhead{Yixia Yu (yy5091@nyu.edu)}

% --- Document starts here ---

\begin{document}
\subsection*{Problem 1}
Show that the assumption that $\mathbb{P}$ is countably additive is equivalent to the assumption that $\mathbb{P}$ is continuous. That is to say, show that if a function $\mathbb{P} : \mathcal{F} \rightarrow [0, 1]$ satisfies $\mathbb{P}(\emptyset) = 0$, $\mathbb{P}(\Omega) = 1$, and $\mathbb{P}(A \cup B) = \mathbb{P}(A) + \mathbb{P}(B)$ whenever $A, B \in \mathcal{F}$ and $A \cap B = \emptyset$, then $\mathbb{P}$ is countably additive (in the sense of satisfying Definition (1.3.1b)) if and only if $\mathbb{P}$ is continuous (in the sense of Lemma (1.3.5)).
\begin{proof}
    \begin{itemize}
    \item $\Rightarrow$ Suppose $\mathbb{P}$ is countably additive. \\
    Suppose w.l.o.g. that $ \{A_n\}_{n=1}^{+\infty} $ be any increasing sequence of events\\
    and $ A_1\subset A_2\cdots $ \\
    Let $ A=\bigcup_{n=1}^{\infty}A_n $\\
    Let $ B_1=A_1,B_2=A_2\backslash A_1,\cdots B_n=An\backslash A_{n-1} $ and they are all disjoint\\
    Then $$ A=\bigcup_{n=1}^{\infty}B_n= \bigcup_{n=1}^{\infty}A_n$$\\  
    For every n, we have $ A_n=\bigcup_{k=1}^{n}B_k $\\
    By countable additivity, we have $ \mathbb{P}(A_n)=\sum_{k=1}^{n}\mathbb{P}(B_k) $\\
    Then, we have $ \mathbb{P}(A)=\mathbb{P}(\bigcup_{k=1}^{\infty}B_k)=\sum_{k=1}^{\infty}\mathbb{P}(B_k) $\\
    So, $$\lim_{n\to\infty}\mathbb{P}(A_n)= \lim_{n\to\infty}\sum_{k=1}^{n}\mathbb{P}(B_k)=\sum_{k=1}^{\infty}\mathbb{P}(B_k)=\mathbb{P}(A)$$
    Thus, $\mathbb{P}$ is continuous from below. (similarly we can prove that $\mathbb{P}$ is continuous from above by assuming decreasing $ \{A_n\}_{n=1}^{+\infty} $)\\
    \item $\Leftarrow$ Suppose $\mathbb{P}$ is continuous.\\
    $\text{Let } \left\{ A_{n} \right\}_{n=1}^{\infty} \text{ be a sequence of pairwise disjoint sets in } \mathcal{F}.$
     Define $ B_{n} = \bigcup_{k=1}^{n} A_{k}.$\\Then $ \left\{ B_{n} \right\}_{n=1}^{\infty}$ is an increasing sequence with $ \bigcup_{n=1}^{\infty} B_{n} = \bigcup_{n=1}^{\infty} A_{n}.$ \\By continuity,
     $\mathbb{P}\left( \bigcup_{n=1}^{\infty} A_{n} \right) = \lim_{n \to \infty} \mathbb{P}(B_{n}).$
Since the $A_{n}$ are pairwise disjoint,$\mathbb{P}(B_{n}) = \sum_{k=1}^{n} \mathbb{P}(A_{k}).$
\\Thus,$$\mathbb{P}\left( \bigcup_{n=1}^{\infty} A_{n} \right) = \lim_{n \to \infty} \sum_{k=1}^{n} \mathbb{P}(A_{k}) = \sum_{n=1}^{\infty} \mathbb{P}(A_{n}).$$
This shows that $\mathbb{P}$ is countably additive.
    \end{itemize}
\end{proof}
\subsection*{Problem 2}
The 'ménages' problem poses the following question. Some consider it to be desirable that men and women alternate when seated at a circular table. If n heterosexual couples are seated randomly according to this rule, show that the probability that nobody sits next to his or her partner is

$$\frac{1}{n!}\sum_{k=0}^{n}(-1)^k\frac{2n}{2n-k}\binom{2n-k}{k}(n-k)!$$

You may find it useful to show first that the number of ways of selecting k non-overlapping pairs of adjacent seats is $\binom{2n-k}{k}2n(2n-k)^{-1}$.
\begin{proof}
    Assume we have \(n\) couples labelled \(1,2,\dots,n\). Define
\[
A_k = \{\text{couple } k \text{ sits together}\}.
\]
For any chosen \(k\)-tuple \((i_1,i_2,\dots,i_k)\), we wish to count 
\[
\mathbb{N}\Bigl(\bigcap_{j=1}^k A_{i_j}\Bigr).
\]

To eliminate rotational symmetry, choose a couple not among \(\{i_1,\dots,i_k\}\) (say the one with the smallest index not chosen) and fix its man in seat 1. Since men and women alternate, the problem reduces to selecting \(k\) non-overlapping adjacent seat pairs from \(2n\) seats. This is equivalent to choosing \(k\) “compressed” positions from \(2n-k\) spots, which can be done in
\[
\binom{2n-k-1}{k}\text{ ways}.
\]

Next, assign the \(k\) couples to these pairs in \(k!\) ways. The remaining \(n-k-1\) men and \(n-k\) women are arranged in \((n-k-1)! (n-k)!\) ways. Including the initial factor \(2n\) from the possible choices before fixing, we have
\[
\mathbb{N}\Bigl(\bigcap_{j=1}^k A_{i_j}\Bigr) = 2n\,\binom{2n-k-1}{k}\,k!\,(n-k-1)!(n-k)!.
\]
Using the inclusion--exclusion principle, the probability that no couple sits together is
\[
\mathbb{P}\Bigl(\bigcap_{j=1}^n A_j^c\Bigr) = 1-\mathbb{P}\Bigl(\bigcup_{j=1}^n A_j\Bigr)
=\sum_{k=0}^n (-1)^k \binom{n}{k}\,\mathbb{P}\Bigl(\bigcap_{j=1}^k A_{i_j}\Bigr).
\]
Substituting the count and dividing by the total number of arrangements, we obtain
\[
\mathbb{P}\Bigl(\bigcap_{j=1}^n A_j^c\Bigr) = \frac{1}{n!}\sum_{k=0}^n (-1)^k \frac{2n}{2n-k}\binom{2n-k}{k}(n-k)!.
\]
 \end{proof}
    \subsection*{Problem 3}
\textbf{The probabilistic method}. 10 per cent of the surface of a sphere is coloured blue, the rest is red.
Show that, irrespective of the manner in which the colours are distributed, it is possible to inscribe a cube in S with all its vertices red.
\begin{proof}
    Let the sphere \(S\) be partitioned into a blue region \(B\) and a red region, with 
    \[
    \frac{|B|}{|S|} = 0.1.
    \]
    Fix a standard cube inscribed in \(S\) so that its 8 vertices lie on \(S\). For every rotation \(R\) (with \(R\) uniformly distributed over \(\mathrm{SO}(3)\)), denote its vertices by 
    \[
    v_1(R), v_2(R), \dots, v_8(R).
    \]
    Define the event 
    \[
    E_i = \{ R \in \mathrm{SO}(3) \mid v_i(R) \in B \}, \quad i = 1,2,\dots,8.
    \]
    Since the distribution of each vertex is uniform on \(S\), we have
    \[
    P(E_i) = 0.1 \quad \text{for each } i.
    \]
    
    Suppose for contradiction that every rotation \(R\) results in at least one vertex landing in \(B\). Then
    \[
    \mathrm{SO}(3) = \bigcup_{i=1}^{8} E_i.
    \]
    By the Inclusion-Exclusion Principle,
    \[
    P\Bigl(\bigcup_{i=1}^{8} E_i\Bigr) 
    \le \sum_{i=1}^{8} P(E_i) 
    = 8\times 0.1 = 0.8.
    \]
    Thus, the probability that a rotation \(R\) yields no vertex in \(B\) is
    \[
    P\Bigl(\bigcap_{i=1}^{8} E_i^c\Bigr) = 1 - P\Bigl(\bigcup_{i=1}^{8} E_i\Bigr) \ge 1 - 0.8 = 0.2 > 0.
    \]
    This positive probability implies that there exists at least one rotation \(R\) such that none of the vertices fall in the blue region; that is, they all lie in the red region.
    
\end{proof}
\subsection*{Problem 4}
\textbf{Poker.} During a game of poker, you are dealt a five-card hand at random. With the convention that aces may count high or low, show that:

$$\begin{aligned}
& \mathbb{P}(1 \text { pair }) \simeq 0.423, \\
& \mathbb{P}(\text { straight }) \simeq 0.0039, \\
& \mathbb{P}(4 \text { of a kind }) \simeq 0.00024, \\
& \mathbb{P}(2 \text { pairs }) \simeq 0.0475, \\
& \mathbb{P}(\text { flush }) \simeq 0.0020, \\
& \mathbb{P}(3 \text { of a kind }) \simeq 0.021, \\
& \mathbb{P}(\text { full house }) \simeq 0.0014, \\
& \mathbb{P}(\text { straight flush }) \simeq 0.000015 .
\end{aligned}$$
\begin{proof}
    \begin{align*}{}{}
        |\Omega|&=\binom{52}{5}  \\
        |1\text{ pair}|&=13\times\binom{4}{2}\times\binom{12}{3}\times4^3  \\
        P[1\text{ pair}]&=\frac{13\times\binom{4}{2}\times\binom{12}{3}\times4^3}{\binom{52}{5}}\approxeq 0.423\\[1ex]
           |Straight|&=10\times(4^5-4)  \\
           P[Straight]&=\frac{10\times(4^5-4)}{\binom{52}{5}}\approxeq 0.0039\\[1ex]
           |4\text{ of a kind}|&=13\times(52-4)  \\
              P[4\text{ of a kind}]&=\frac{13\times48}{\binom{52}{5}}\approxeq 0.00024\\[1ex]
              |2\text{ pairs}|&=\binom{13}{2}\times\binom{4}{2}^2\times11\times4  \\
              P[2\text{ pairs}]&=\frac{\binom{13}{2}\times\binom{4}{2}^2\times11\times4}{\binom{52}{5}}\approxeq 0.0475\\[1ex]
                |Flush|&=4\times\binom{13}{5} -4\times10 \\
                P[Flush]&=\frac{4\times\binom{13}{5} -4\times10}{\binom{52}{5}}\approxeq 0.0020\\[1ex]
              |3\text{ of a kind}|&=13\times\binom{4}{3}\times\binom{12}{2}\times4^2  \\
                P[3\text{ of a kind}]&=\frac{13\times\binom{4}{3}\times\binom{12}{2}\times4^2}{\binom{52}{5}}\approxeq 0.021\\[1ex]
           |Full House|&=13\times\binom{4}{3}\times12\times\binom{4}{2}  \\
           P[Full House]&=\frac{13\times\binom{4}{3}\times12\times\binom{4}{2}}{\binom{52}{5}}\approxeq 0.0014\\[1ex]
           |Straight flush|&=10\times 4\\
           P[Straight flush]&=\frac{10\times 4}{\binom{52}{5}}\approxeq 0.000015
        \end{align*} 
\end{proof}
\subsection*{Problem 5}
Let $m$ be Lebesgue measure on [0,1], and $0 \leq a \leq b \leq c \leq d \leq 1$ such that $a+d \geq b+c$. Give an example of a sequence of sets $A_1, A_2, \cdots$ in [0,1], such that $m(\liminf_n A_n) = a$, $\liminf_n m(A_n) = b$, $\limsup_n m(A_n) = c$ and $m(\limsup_n A_n) = d$.
\begin{proof}
    
    Define
    \[
    I = [0,a] \quad \text{and} \quad E = [a,d].
    \]
    Let \(\alpha\) be any irrational number(say $ \sqrt{2} $ ) and let \(\{ n\alpha \}\) denote the fractional part of \(n\alpha\). For each \(n\in\mathbb{N}\), set
    \[
    X_n =
    \begin{cases}
    \displaystyle \left[\,a+(d-a)\{n\alpha\},\; a+(d-a)\{n\alpha\}+(b-a)\,\right], & \text{if } n \text{ is even},\\[1ex]
    \displaystyle \left[\,a+(d-a)\{n\alpha\},\; a+(d-a)\{n\alpha\}+(c-a)\,\right], & \text{if } n \text{ is odd},
    \end{cases}
    \]
    and define
    \[
    A_n = I \cup X_n.
    \]
    
    Then we can have
    \[
    m\Bigl(\liminf_{n\to\infty} A_n\Bigr)=a,\quad
    \liminf_{n\to\infty} m(A_n)=b,\quad
    \limsup_{n\to\infty} m(A_n)=c,\quad
    m\Bigl(\limsup_{n\to\infty} A_n\Bigr)=d.
    \]
\end{proof}
\subsection*{Problem 6}
Let $\Omega = \mathbb{R}$ and consider the following subsets of $\mathcal{P}(\mathbb{R})$:
$$\begin{aligned}
& \mathcal{C}_1 := \left\{( -\infty, b] : b \in \mathbb{R} \right\} \\
& \mathcal{C}_2 := \left\{( a, b] : a, b \in \mathbb{R} \right\} \\
& \mathcal{C}_3 := \left\{ A \subset \mathbb{R}, A \text{ is closed } \right\}.
\end{aligned}$$

Show that $\sigma(\mathcal{C}_1) = \sigma(\mathcal{C}_2) = \sigma(\mathcal{C}_3)$.
\begin{proof}
    $ \sigma(C_1)=\sigma(C_2) $ 
    \begin{itemize}
    \item $ C_1\subseteq \sigma(C_2) $ \\Since $(-\infty,b]\in \sigma(C_2)$\\So, $\sigma(C_1)\subseteq\sigma(C_2)$ because $ \sigma(C_1) $ is minimal
    \item $ C_2\subseteq \sigma(C_1) $ 
    \\Writing $ (a,b]=(-\infty,b]\backslash(-\infty ,a]$ \\Since$  (-\infty,b]\text{ and }(-\infty ,a] $ are in $ C_1 $, we have (a,b]$\in \sigma(C_1) $\\So, $ \sigma(C_2)\subseteq\sigma(C_1)$ because $ \sigma(C_2) $ is minimal 
    \end{itemize}
    $ \sigma(C_2)=\sigma(C_3) $ 
      
  \begin{itemize}
    \item \emph{\(\sigma( \mathcal{C}_2)\subseteq \sigma(\mathcal{C}_3)\):}  
    Let \((a,b] \in \mathcal{C}_2\). Notice that for any \(x\in\mathbb{R}\), the set \((-\infty,x]\) is closed. Hence, both \((-\infty,b]\) and \((-\infty,a]\) belong to \(\mathcal{C}_3\). 
    \\Writing
    $(a,b] = (-\infty,b] \setminus (-\infty,a],$
    we conclude that \((a,b]\) is an element of \(\sigma(\mathcal{C}_3)\). Therefore, 
    $\mathcal{C}_2 \subseteq \sigma(\mathcal{C}_3).$ 
    \\So $ \sigma( \mathcal{C}_2)\subseteq \sigma(\mathcal{C}_3)\ $ because $ \sigma(C_2) $ is minimal
    \item\emph{\( \sigma(\mathcal{C}_3) \subseteq \sigma(\mathcal{C}_2)\):}  
    Every open set \( U \subset \mathbb{R} \) can be written as
$
U = \bigcup_{n=1}^\infty I_n,
$
\\where each \( I_n \) is an open interval. For any open interval \( I_n = (a,b) \), we have
\[
(a,b) = \bigcup_{m=1}^\infty \left(a+\frac{1}{m}, b\right].
\]
Thus,
\[
U = \bigcup_{n=1}^\infty \bigcup_{m=1}^\infty \left(a_n+\frac{1}{m}, b_n\right],
\]
showing that \( U \) is a countable union of half–open intervals of the form \((a,b]\). \\Hence,
$
U \in \sigma(\mathcal{C}_2),
$
where 
$
\mathcal{C}_2 = \{ (a,b] : a,b \in \mathbb{R} \}.
$
\\So we have $ U^c\in \sigma(\mathcal{C}_2)$ ,where $ U^c= \mathcal{C}_3$ \\
Therefore, \( \sigma(\mathcal{C}_3) \subseteq \sigma(\mathcal{C}_2) \) because $ \sigma(C_3) $ is minimal

\end{itemize}
\end{proof}
\end{document}