% --- LaTeX Homework Template - S. Venkatraman ---

% --- Set document class and font size ---

\documentclass[letterpaper, 11pt]{article}

% --- Package imports ---

\usepackage{
  amsmath, amsthm, amssymb, mathtools, dsfont,	  % Math typesetting
  graphicx, wrapfig, subfig, float,                  % Figures and graphics formatting
  listings, color, inconsolata, pythonhighlight,     % Code formatting
  fancyhdr, sectsty, hyperref, enumerate, enumitem } % Headers/footers, section fonts, links, lists

% --- Page layout settings ---

% Set page margins
\usepackage[left=1.35in, right=1.35in, bottom=1in, top=1.1in, headsep=0.2in]{geometry}

% Anchor footnotes to the bottom of the page
\usepackage[bottom]{footmisc}

% Set line spacing
\renewcommand{\baselinestretch}{1.2}

% Set spacing between paragraphs
\setlength{\parskip}{1.5mm}

% Allow multi-line equations to break onto the next page
\allowdisplaybreaks

% Enumerated lists: make numbers flush left, with parentheses around them
\setlist[enumerate]{wide=0pt, leftmargin=21pt, labelwidth=0pt, align=left}
\setenumerate[1]{label={(\arabic*)}}

% --- Page formatting settings ---

% Set link colors for labeled items (blue) and citations (red)
\hypersetup{colorlinks=true, linkcolor=blue, citecolor=red}

% Make reference section title font smaller
\renewcommand{\refname}{\large\bf{References}}

% --- Settings for printing computer code ---

% Define colors for green text (comments), grey text (line numbers),
% and green frame around code
\definecolor{greenText}{rgb}{0.5, 0.7, 0.5}
\definecolor{greyText}{rgb}{0.5, 0.5, 0.5}
\definecolor{codeFrame}{rgb}{0.5, 0.7, 0.5}

% Define code settings
\lstdefinestyle{code} {
  frame=single, rulecolor=\color{codeFrame},            % Include a green frame around the code
  numbers=left,                                         % Include line numbers
  numbersep=8pt,                                        % Add space between line numbers and frame
  numberstyle=\tiny\color{greyText},                    % Line number font size (tiny) and color (grey)
  commentstyle=\color{greenText},                       % Put comments in green text
  basicstyle=\linespread{1.1}\ttfamily\footnotesize,    % Set code line spacing
  keywordstyle=\ttfamily\footnotesize,                  % No special formatting for keywords
  showstringspaces=false,                               % No marks for spaces
  xleftmargin=1.95em,                                   % Align code frame with main text
  framexleftmargin=1.6em,                               % Extend frame left margin to include line numbers
  breaklines=true,                                      % Wrap long lines of code
  postbreak=\mbox{\textcolor{greenText}{$\hookrightarrow$}\space} % Mark wrapped lines with an arrow
}

% Set all code listings to be styled with the above settings
\lstset{style=code}

% --- Math/Statistics commands ---

% Add a reference number to a single line of a multi-line equation
% Usage: "\numberthis\label{labelNameHere}" in an align or gather environment
\newcommand\numberthis{\addtocounter{equation}{1}\tag{\theequation}}

% Shortcut for bold text in math mode, e.g. $\b{X}$
\let\b\mathbf

% Shortcut for bold Greek letters, e.g. $\bg{\beta}$
\let\bg\boldsymbol

% Shortcut for calligraphic script, e.g. %\mc{M}$
\let\mc\mathcal

% \mathscr{(letter here)} is sometimes used to denote vector spaces
\usepackage[mathscr]{euscript}

% Convergence: right arrow with optional text on top
% E.g. $\converge[w]$ for weak convergence
\newcommand{\converge}[1][]{\xrightarrow{#1}}

% Normal distribution: arguments are the mean and variance
% E.g. $\normal{\mu}{\sigma}$
\newcommand{\normal}[2]{\mathcal{N}\left(#1,#2\right)}

% Uniform distribution: arguments are the left and right endpoints
% E.g. $\unif{0}{1}$
\newcommand{\unif}[2]{\text{Uniform}(#1,#2)}

% Independent and identically distributed random variables
% E.g. $ X_1,...,X_n \iid \normal{0}{1}$
\newcommand{\iid}{\stackrel{\smash{\text{iid}}}{\sim}}

% Equality: equals sign with optional text on top
% E.g. $X \equals[d] Y$ for equality in distribution
\newcommand{\equals}[1][]{\stackrel{\smash{#1}}{=}}

% Math mode symbols for common sets and spaces. Example usage: $\R$
\newcommand{\R}{\mathbb{R}}   % Real numbers
\newcommand{\C}{\mathbb{C}}   % Complex numbers
\newcommand{\Q}{\mathbb{Q}}   % Rational numbers
\newcommand{\Z}{\mathbb{Z}}   % Integers
\newcommand{\N}{\mathbb{N}}   % Natural numbers
\newcommand{\F}{\mathcal{F}}  % Calligraphic F for a sigma algebra
\newcommand{\El}{\mathcal{L}} % Calligraphic L, e.g. for L^p spaces

% Math mode symbols for probability
\newcommand{\pr}{\mathbb{P}}    % Probability measure
\newcommand{\E}{\mathbb{E}}     % Expectation, e.g. $\E(X)$
\newcommand{\var}{\text{Var}}   % Variance, e.g. $\var(X)$
\newcommand{\cov}{\text{Cov}}   % Covariance, e.g. $\cov(X,Y)$
\newcommand{\corr}{\text{Corr}} % Correlation, e.g. $\corr(X,Y)$
\newcommand{\B}{\mathcal{B}}    % Borel sigma-algebra

% Other miscellaneous symbols
\newcommand{\tth}{\text{th}}	% Non-italicized 'th', e.g. $n^\tth$
\newcommand{\Oh}{\mathcal{O}}	% Big-O notation, e.g. $\O(n)$
\newcommand{\1}{\mathds{1}}	% Indicator function, e.g. $\1_A$

% Additional commands for math mode
\DeclareMathOperator*{\argmax}{argmax}    % Argmax, e.g. $\argmax_{x\in[0,1]} f(x)$
\DeclareMathOperator*{\argmin}{argmin}    % Argmin, e.g. $\argmin_{x\in[0,1]} f(x)$
\DeclareMathOperator*{\spann}{Span}       % Span, e.g. $\spann\{X_1,...,X_n\}$
\DeclareMathOperator*{\bias}{Bias}        % Bias, e.g. $\bias(\hat\theta)$
\DeclareMathOperator*{\ran}{ran}          % Range of an operator, e.g. $\ran(T) 
\DeclareMathOperator*{\dv}{d\!}           % Non-italicized 'with respect to', e.g. $\int f(x) \dv x$
\DeclareMathOperator*{\diag}{diag}        % Diagonal of a matrix, e.g. $\diag(M)$
\DeclareMathOperator*{\trace}{trace}      % Trace of a matrix, e.g. $\trace(M)$

% Numbered theorem, lemma, etc. settings - e.g., a definition, lemma, and theorem appearing in that 
% order in Section 2 will be numbered Definition 2.1, Lemma 2.2, Theorem 2.3. 
% Example usage: \begin{theorem}[Name of theorem] Theorem statement \end{theorem}
\theoremstyle{definition}
\newtheorem{theorem}{Theorem}[section]
\newtheorem{proposition}[theorem]{Proposition}
\newtheorem{lemma}[theorem]{Lemma}
\newtheorem{corollary}[theorem]{Corollary}
\newtheorem{definition}[theorem]{Definition}
\newtheorem{example}[theorem]{Example}
\newtheorem{remark}[theorem]{Remark}

% Un-numbered theorem, lemma, etc. settings
% Example usage: \begin{lemma*}[Name of lemma] Lemma statement \end{lemma*}
\newtheorem*{theorem*}{Theorem}
\newtheorem*{proposition*}{Proposition}
\newtheorem*{lemma*}{Lemma}
\newtheorem*{corollary*}{Corollary}
\newtheorem*{definition*}{Definition}
\newtheorem*{example*}{Example}
\newtheorem*{remark*}{Remark}
\newtheorem*{claim}{Claim}

% --- Left/right header text (to appear on every page) ---

% Include a line underneath the header, no footer line
\pagestyle{fancy}
\renewcommand{\footrulewidth}{0pt}
\renewcommand{\headrulewidth}{0.4pt}

% Left header text: course name/assignment number
\lhead{MATH-SHU 238 (HTOP) -- Homework 1}

% Right header text: your name
\rhead{Yixia Yu (yy5091@nyu.edu)}

% --- Document starts here ---

\begin{document}
\subsection*{Problem 1}
Let $\mathcal{F}$ and $\mathcal{G}$ be $\sigma$-fields of subsets of $\Omega$.
\begin{align*}
(a)\quad& \text{Use elementary set operations to show that $\mathcal{F}$ is closed under}\\
  & \text{countable intersections; that is, if $A_1, A_2, \ldots$ are in $\mathcal{F}$,}\\
  & \text{then so is $\bigcap_i A_i$.}\\
(b)\quad& \text{Let $\mathcal{H} = \mathcal{F} \cap \mathcal{G}$ be the collection of subsets of $\Omega$ lying}\\
  & \text{in both $\mathcal{F}$ and $\mathcal{G}$. Show that $\mathcal{H}$ is a $\sigma$-field.}\\
(c)\quad& \text{Show that $\mathcal{F} \cup \mathcal{G}$, the collection of subsets of $\Omega$ lying in}\\
  & \text{either $\mathcal{F}$ or $\mathcal{G}$, is not necessarily a $\sigma$-field.}
\end{align*}
\begin{proof}
  \begin{align*}
    (a):& \text{Let } A_1, A_2, \ldots \in \mathcal{F}. \text{ Then } A_1^c, A_2^c, \ldots \in \mathcal{F}. \nonumber\\
    &\text{Since } \mathcal{F} \text{ is a $\sigma$-field, we have } \bigcup_i A_i \in \mathcal{F}. \nonumber\\
    &\text{By De Morgan's Law, } \left(\bigcap_i A_i\right)^c = \bigcup_i A_i^c \in \mathcal{F}. \nonumber\\
    &\text{Hence, } \bigcap_i A_i \in \mathcal{F}. \nonumber\\
    (b):& \text{1. Since } \Omega \in \mathcal{F} \text{ and } \Omega \in \mathcal{G}, \text{ we have } \Omega \in \mathcal{H}. \nonumber\\
    &\text{2. } \forall A\in \mathcal{H}, \text{ we have } A \in \mathcal{F} \text{ and } A \in \mathcal{G}, \text{ so } A^c \in \mathcal{F} \text{ and } A^c \in \mathcal{G}, \text{ hence } A^c \in \mathcal{H}. \nonumber\\
    &\text{3. } \forall A_1, A_2, \ldots \in \mathcal{H}, \text{ we have } A_1, A_2, \ldots \in \mathcal{F} \text{ and } A_1, A_2, \ldots \in \mathcal{G}, \nonumber\\
    &\quad\text{so } \bigcup_i A_i \in \mathcal{F} \text{ and } \bigcup_i A_i \in \mathcal{G}, \text{ hence } \bigcup_i A_i \in \mathcal{H}. \nonumber\\
    (c):& \text{Counterexample: Let } \Omega = \{1, 2, 3\}, \mathcal{F} = \{\emptyset, \{1, 2\}, \{3\}, \Omega\}, \mathcal{G} = \{\emptyset, \{1\}, \{2, 3\}, \Omega\}. \nonumber\\
    &\text{Then } \{1, 2\} \in \mathcal{F} \text{ and } \{1\} \in \mathcal{G}, \text{ but } \{1, 2\} \cup \{1\} =  \{1, 2\} \notin \mathcal{F} \cup \mathcal{G}. \nonumber\\
    &\text{Hence, } \mathcal{F} \cup \mathcal{G} \text{ is not necessarily a $\sigma$-field.} \nonumber
  \end{align*}
   \end{proof}
\subsection*{Problem 2}
Let $A_{1}, A_{2}, \ldots$ be a sequence of events. Define

$$B_{n}=\bigcup_{m=n}^{\infty} A_{m}, \quad C_{n}=\bigcap_{m=n}^{\infty} A_{m}$$

Clearly $C_{n} \subseteq A_{n} \subseteq B_{n}$. The sequences $\left\{B_{n}\right\}$ and $\left\{C_{n}\right\}$ are decreasing and increasing respectively with limits

$$\lim B_{n}=B=\bigcap_{n} B_{n}=\bigcap_{n} \bigcup_{m \geq n} A_{m}, \quad \lim C_{n}=C=\bigcup_{n} C_{n}=\bigcup_{n} \bigcap_{m \geq n} A_{m}$$

The events $B$ and $C$ are denoted $\limsup_{n \rightarrow \infty} A_{n}$ and $\liminf_{n \rightarrow \infty} A_{n}$ respectively. Show that
\begin{align*}
  (a)\quad &B=\{\omega \in \Omega: \omega \in A_{n} \text{ for infinitely many values of }n\},\\
  (b)\quad &C=\{\omega \in \Omega: \omega \in A_{n} \text{ for all but finitely many values of }n\},\\
  &\text{We say that the sequence }\{A_n\} \text{ converges to } A = \lim A_n
  \text{ if } B \text{ and } C \text{ are the same set } A.\\
  &\text{Suppose that } A_n \to A \text{ and show that}\\
  (c)\quad &A \text{ is an event, i.e. } A \in \mathcal{F},\\
  (d)\quad &\mathbb{P}(A_n) \to \mathbb{P}(A).
  \end{align*}
\begin{proof}
  \begin{align*}{}{}
  (a):&\text{Since by defination: } B = \bigcap_{n} \bigcup_{m \geq n} A_{m} \\
  \Rightarrow&\text{Suppose } \omega \in B, \text{ then } \omega \in \bigcup_{m \geq n} A_{m} \text{ for all } n \in \mathbb{N} \\
  &\text{This means for each n, we can find an }m\geq n \text{ such that } \omega \in A_{m} \\
  &\text{Hence, } \omega \in A_{n} \text{ for infinitely many values of } n \\
  \Leftarrow&\text{Suppose } \omega \in A_{n} \text{ for infinitely many values of } n, \\
  &\text{then for each } n, \text{ we can find an } m\geq n \text{ such that } \omega \in A_{m} \\
  &\text{This means } \omega \in \bigcup_{m \geq n} A_{m} \text{ for all } n \in \mathbb{N} \\
  &\text{Hence, } \omega \in \bigcap_{n} \bigcup_{m \geq n} A_{m} = B \\
  (b): \Rightarrow&\text{Suppose } \omega \in C, \text{ then }\exists n \in \mathbb{N} \text{ such that }\omega \in \bigcap_{m \geq n} A_{m} \\
  &\text{This means }\exists n \in \mathbb{N} \text{ such that }\omega \in A_{m} \text{ for all } m \geq n \\
  &\text{Hence, }\omega \in A_{n} \text{ for all but finitely many values of } n \\
  \Leftarrow&\text{Suppose }\omega \in A_{n} \text{ for all but finitely many values of } n, \\
  &\text{then }\exists n \in \mathbb{N} \text{ such that }\omega \in A_{m} \text{ for all } m \geq n \\
  &\text{This means }\omega \in \bigcap_{m \geq n} A_{m} \\
  &\text{Hence, }\omega \in \bigcup_{n} \bigcap_{m \geq n} A_{m} = C \\
  (c):&\text{Since } A_n \rightarrow A, \text{ so } \lim_{n\rightarrow+\infty}A_n=\limsup_{n\rightarrow+\infty}A_n=\liminf_{n\rightarrow+\infty}A_n\\
  &\text{So we have } A=B=C\\
  &\text{Since } B_n \text{ are countable unions of } A_m, \text{ so } B_n \in \mathcal{F} \text{ for all } n \in \mathbb{N} \\
  &\text{Then, } B = \bigcap_n B_n \in \mathcal{F} \\
  &\text{Hence, we have } A \in \mathcal{F} \\
  (d):&\text{Since } C_{n} \subseteq A_{n} \subseteq B_{n},\text{ we have } \mathbb{P}(C_{n}) \leq \mathbb{P}(A_{n}) \leq \mathbb{P}(B_{n}) \\
  &\text{Since } C_{n} \text{ increases to } C, B_{n} \text{ decreases to } B, \\
  &\text{ we have } \mathbb{P}(C_{n}) \rightarrow \mathbb{P}(C)=\mathbb{P}(A), \mathbb{P}(B_{n}) \rightarrow \mathbb{P}(B)=\mathbb{P}(A) \\
  &\text{Hence by squeeze theorem, } \mathbb{P}(A_{n}) \rightarrow \mathbb{P}(A)
  \end{align*}
\end{proof}
\subsection*{Problem 3}
Let $\mathcal{E}$ be the left open right closed intervals of $\Omega := \mathbb{R}$ defined in class. Write down the algebra generated by $\mathcal{E}$, and prove your result.
\\We have $\mathcal{E} = \{\text{left open right closed intervals}\} = 
\begin{cases}
    (a,b], & -\infty \leq a < b < +\infty \\
    (a,+\infty)
\end{cases}$
\begin{align*}a(\mathcal{E})&=\text{  "finite disjioint union of elements in }\mathcal{E}"
  \\&=\underbrace{\{ I_1\cup\cdots\cup I_k;I_j \in \mathcal{E} , \text{ and } I_i \cap I_j =\emptyset \}}_f
  \end{align*}
\begin{proof}
  Denote the collection of finite disjioint union of elements in $ \mathcal{E} $  by $ f $ 
  \begin{itemize}
    \item $ f \subseteq a(\mathcal{E}) $ 
    \begin{itemize}
    \item Since $a(\mathcal{E})$ is an algebra containing $\mathcal{E}$, it is closed under finite unions.
    \item Any element of $f$ is a finite disjoint union of intervals from $\mathcal{E}$, hence belongs to $a(\mathcal{E})$.
    \item Thus, $f \subseteq a(\mathcal{E})$
    \end{itemize}
    \item $ a(\mathcal{E}) \subseteq f $ 
    \begin{itemize}
    \item we need to prove that $ f  $ is an algebra
    \begin{itemize}
    \item since $\Omega =\mathbb{R}=(-\infty,+\infty)$, choose $I_1 =(-\infty,a],I_2=(a,+\infty) \in \mathcal{E}$, then $I_1\cup I_2 = \Omega \in f$
    \item for $\forall I_i,I_j\in f , I_i\cup I_j\in f $ by defination
    \item for $\forall (a,b]\in f, \text{ we have }{(a,b]}^c=(-\infty,a]\cup(b,+\infty) \in f$
    \\ for $\forall (a,+\infty)\in f, \text{ we have }(a,+\infty)^c=(-\infty,a] \in f$
    \end{itemize}
    \item since $ \mathcal{E}\subseteq f $ , $f\text{ is an algebra and } a(\mathcal{E}) $ is the smallest algebra containing $ \mathcal{E} $, \\
    Hence, $ a(\mathcal{E}) \subseteq f $
    \end{itemize}
    \end{itemize}
\end{proof}
\subsection*{Problem 4}
Let $\Omega$ be a sample space. Show that any finite algebra on $\Omega$ is a $\sigma-$algebra.
\begin{proof}
Let $\mathcal{A}$ be a finite algebra on $\Omega$, and we can write $\mathcal{A} = \{A_1, A_2, \ldots, A_n\}$. 
\begin{itemize}
\item Since $\mathcal{A}$ is an algebra, $\Omega \in \mathcal{A}$.
\item Since $\mathcal{A}$ is an algebra, if $A_i \in \mathcal{A}$, then $A_i^c \in \mathcal{A}$.
\item Since $\mathcal{A}$ is an algebra, if $A_i, A_j \in \mathcal{A}$, then $A_i \cup A_j \in \mathcal{A}$.
\\Let \(\{B_i\}_{i \in I}\) be a countable family of subsets of \(\Omega\), where \(I\) is a countable index set, and each \(B_i \in \mathcal{A}\). Since \(\mathcal{A}\) is finite, the image of the function \(f: I \to \mathcal{A}\) defined by \(f(i) = B_i\) is finite. That is, there are only finitely many distinct sets in \(\{B_i\}_{i \in I}\). Let these distinct sets be \(C_1, C_2, \dots, C_m\), where \(m \leq n\) and \(C_k \in \mathcal{A}\) for all \(k = 1, 2, \dots, m\).

The countable union \(\bigcup_{i \in I} B_i\) can therefore be written as:
\[
\bigcup_{i \in I} B_i = C_1 \cup C_2 \cup \cdots \cup C_m.
\]

Since \(\mathcal{A}\) is an algebra, it is closed under finite unions. By induction:
\begin{itemize}
    \item For \(m = 1\), \(C_1 \in \mathcal{A}\).
    \item Assume \(C_1 \cup C_2 \cup \cdots \cup C_k \in \mathcal{A}\) for some \(m=k \geq 1\).
    \item Then, (\(C_1 \cup C_2 \cup \cdots \cup C_k) \cup C_{k+1} \in \mathcal{A}\) because \(\mathcal{A}\) is closed under binary unions.
\end{itemize}

Thus, \(C_1 \cup C_2 \cup \cdots \cup C_m \in \mathcal{A}\), and we conclude:
\[
\bigcup_{i \in I} B_i \in \mathcal{A}.
\]

\end{itemize}
Therefore, \(\mathcal{A}\) is a \(\sigma\)-algebra.
\end{proof}
\end{document}