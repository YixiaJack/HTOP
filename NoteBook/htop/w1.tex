\chapter[Honors Theory of Probability ]{Honors Theory of Probability}
\section[Lecture 1 (02-03) -- {Introduction to Probability}]{Lecture 1 (02-03)Introduction to Probability}
Office Hour: Wed 12:30-1:30pm,Fri 3:30-4:30pm W910
\begin{itemize}
    \item Homework: weekly
    \item Grades: 5\% participation, 15\% homework, 40\% midterm, final group project(Max 4 people a group,presetation 20\%, Final report 20\% 10 pages)
\end{itemize}

\subsection{Intro}

Why do we need modern theory of probability?
\begin{example}[]{}
    \begin{itemize}
        \item Coin flip 7 times, what is P[first outcome and fifth outcome are Head]?
    \\sol: \[\Omega=(x1,x2,x3,x4,x5,x6,x7)\]\[ A=(x1,x2,x3,x4,x5,x6,x7),x1=x5=H\]\[P(A)=\frac{|\Omega|}{|A|}=\frac{1}{4} \] 
        \item Stock Price (mathematically,geometric Browmion motion, STochastic process, such that $t \rightarrow S_t$ is continuous but nowehere differentiable)
     \\What is P[$S_T \geq 100$]? 
     \[\Omega= C[0,T]=\{\text{continuous function}, f[0,T] \rightarrow R\}\]
     \[A=\{f \in C[0,T] | f(t) \geq 100\}\]
     \\requires measure theory that defines P[$S_T \geq 100$]
     \\In modern Prob, Probability Space:($\Omega,F,P$), $\Omega$ is sample space, $F$ is $\sigma$-algebra(meaningful subset of $\sigma$), $P$ is probability measure($P:F\rightarrow[0,1]$).
    \end{itemize}
    \end{example}
Topics covered:
\begin{itemize}
    \item Probability space, $\sigma$-algebra, measure ,Conditional Probability and Independence
    \item Random variables (measurable functions),distribution
    \item Expectation (Lebesgue integral),Conditional distribution and expectation, functionos of random variables, Radom-Nikodym Derivative
    \item Random walks
    \item generating functions, characteristic functions
    \item Branching processes
    \item Convergence of random variables, Law of large numbers, Morte-Carlo Method
    \item Central Limit Theorem
    \item Time permitting: Large deviations, Markov Chains
\end{itemize}



\subsection{Probability Space}
($\Omega,F,P$)
\begin{example}[Coinflip and Stock Price]{}
    \begin{itemize}
        \item Coin flip infinite times, \[\Omega=\{(x1,x2,\cdots),x_i=H , T\}\]
        \\ Let $A$ be the event of $10^6$ consecutive Tails, 
        \[ 
            A=\bigcup_{i=1}^{+\infty}\{x_i=x_{i+1}=\cdots=x_{i+10^6-1}=0,(x_i \in \Omega)\} 
             \]
        \\$P[A]=1$
        \item Discrete Stock Price model, t=0,1,2,$\cdots$,T
        \\T is maturing time, time step $\Delta t$ $\ll$ $T$
        
        \[ N=\frac{T}{\Delta t}
        \] \[ \text{price go} \begin{cases} \nearrow \text{by factor}: e^{\sigma \sqrt{\Delta t}}\\\searrow \text{by factor}:e^{-\sigma \sqrt{\Delta t}}
        \end{cases} \]
        \[\Omega=\{(x_1,x_2,\cdots,x_N),x_i=0 \text{ or }1\}\]
        \\Stock price at time t: $\forall \omega \in \Omega$
        \[ S_N(\omega)=S_0e^{\sum_{i=1}^{N}x_i{\sigma \sqrt{\Delta t}}{e^{(N-\sum_{i=1}^{N}x_i)(-\sigma \sqrt{\Delta t})}}}\]
        \[S: \Omega \rightarrow R (\text{Random Variable})\]
        \\Event: return at T is positive but not more than 10\%:
        \[\{\omega \in \Omega:S_N(\omega)\in(S_0,1.1S_0]\}\]
    \end{itemize}
\end{example}
  \begin{example}[Gambling]{}
    \begin{itemize}
        \item Gambling:\begin{itemize}
            \item start with \{0,1,2,$\cdots$\} each time bet an interger amount
            \item if amount of money =0, stays at 0
            \\wealth process: $\Omega-\{0,1,2,\cdots\} \times \{0,1,2,\cdots\} \times \cdots$
            \\wealth after time n: (random variable) $X_n: \Omega \rightarrow N, (x_1,x_2,...x_n) \rightarrow x_n$
            \\$(X_n) $is a Markov Chain (future only depends on present state, but not the past) 
            \begin{align*}
                &\text{Event: \{State j is reached from state i\}}\\
                &=\{\omega \in \Omega: \exists n \in N, X_0(\omega)=i,X_{n}(\omega)=j\}\\
                &=\bigcup_{m=1}^{+\infty}\{\omega\in\Omega:X_o(\omega)=i,X_n(\omega)=j\}
            \end{align*}
          
        \end{itemize}
        
    \end{itemize}
    \end{example}


\section[Lecture 2 (02-05) -- {Algebra}]{Lecture 2 (02-05) -- {Algebra}}
\subsection{algebra and $\sigma$-algebra}
in practice, want $ F $ to be closed under $\bigcap ,\bigcup, ^c $
\begin{definition}[]{}
Let $\mathcal{A}  $ be a collection of subsets of $ \Omega   $ , $\mathcal{A}$ is an algebra iff:
\begin{itemize}
\item $ \Omega \in \mathcal{A} $
\item if A,B$\in \mathcal{A}$ then A$\cup$B $\in \mathcal{A}$
\item if A$\in \mathcal{A}$ then A$^c \in \mathcal{A}$
\end{itemize} 
\end{definition}
\begin{remark}[]{}
if A,B$ \in \mathcal{A} \rightarrow$ A $\cap$ B$ \in \mathcal{A}$, because A$ \cap $B = (A$^c \cup$B$^c)^c$ 

\end{remark}
\textbf{Fact:} 
\begin{enumerate}[label=\circled{\arabic*}]
\item $ P $($ \Omega $)(powerset)=\{A:A $\subset \Omega \} $ is an algebra
\item smallest algebra/trivial algebra: $ \{\emptyset,\Omega \} $
\item Let $ \mathcal{A}_1,\mathcal{A}_2 $ be two algebras of $ \Omega $ $$
\mathcal{A}_1 \cap \mathcal{A}_2 =\{ B\in \Omega:B\in \mathcal{A}_1 \text{   and   }B \in \mathcal{A}_2\} \text{    is an algebra}
$$ 
$$
    \text{if} {(\mathcal{A}_j)}_{j\in J} \text{is a family of algebras, then} \bigcap_{j\in J}\mathcal{A}_j \text{  is an algebra}
$$ 
\item Let $\mathcal{E} $ be any collection of subsets of $ \Omega $ 
$$
    a(\mathcal{E})=\bigcap_{\substack{{\mathcal{A} \supseteq  \mathcal{E}}\\{\mathcal{A} \text{   is an algebra}}}}\mathcal{A} \text{    is an algebra}
$$
\\$ a(\mathcal{E}) $ is an algebra by \circled{3}
\\in fact, $a(\mathcal{E})$ is the smallest algebra containing $\mathcal{E}$
\\It is called the algebra generated by $\mathcal{E}$
\item Let $ A \subseteq \Omega$ ,$ \mathcal{E}=\{A\} $ 
\\Then$$
    a(\mathcal{E})=\{\underbrace{A,A^c,\Omega,\emptyset}_f\}
$$ 
\textbf{Proof:}
\begin{itemize}
    \item $a(\mathcal{E}) \subseteq f$
\\notice that $ f $ is an algebra since $ f\supseteq \mathcal{E} $, therefore $ f \supseteq a(\mathcal{E}) $ because $ a(\mathcal{E}) $ is the smallest algebra containing $ \mathcal{E} $.
 \item $f\subseteq a(\mathcal{E})$:
$$
    A\in a(\mathcal{E}), A^c \in a(\mathcal{E})
$$ 
\\beacuse $a(\mathcal{E})$ is an algebra, $\emptyset,\Omega \in a(\mathcal{E})$
\end{itemize}
\item $\pi =\{A_1,A_2,\cdots ,A_n\}$, $ \Omega=\bigcup_{i=1}^{n}A_i,A_i\cap A_j=\emptyset $ 
Then:$$
    a(\pi)=\{\bigcup_{i\in I}A_i,for I\subset {1,2,\cdots, n}\}=\text{    finite disjoint union of   } {(A_i)}_{i=1}^{n}
$$ 
\item Let $\mathcal{A} $ be an algebra of $\mathbb{R}$
\\Let $ X:\Omega \rightarrow $ be a function 
\\Then $$\underbrace{\{X^{-1}(A),A\in \mathcal{A} \}}_{\{\omega \in \Omega :X(\omega)\in A, \text{  for some } A \in \mathcal{A} \}} \text{  is an algebra of  } \Omega$$  
\\\textbf{Hint:} $$X^{-1}(A\cup B)=(X^{-1}(A))\cup (X^{-1}(B))$$
\item $\Omega = \mathbb{R}$, $\mathcal{E} = \{\text{left open right closed intervals}\} = 
\begin{cases}
    (a,b], & -\infty \leq a < b < +\infty \\
    (a,+\infty)
\end{cases}$
\\Then:\begin{align*}a(\mathcal{E})&=\text{  "finite disjioint union of elements in }\mathcal{E}"
\\&=\underbrace{\{ I_1\cup\cdots\cup I_k;I_j \in \mathcal{E} , I_i \cap I_j =\emptyset \}}_f
\end{align*}
\\\textbf{Hint:} 
\begin{itemize}
\item $ f \subseteq a(\mathcal{E}) $ strightforward
\item $ a(\mathcal{E}) \subseteq f $ \begin{itemize}
\item check $ f  $ is an algebra
\item since $ \mathcal{E}\subset f $ , $ a(\mathcal{E}) $ is the smallest algebra containing $ \mathcal{E} $, $ a(\mathcal{E}) \subseteq f $
\item ${(a,b]}^c=\underbrace{(-\infty,a]\cup(b,+\infty)}_{\in f}$
\end{itemize}
\end{itemize}

\end{enumerate}

\begin{lemma}{}{}
    In Probability, if $ {(A_k)}_{k=1}^n $ are disjoint events,
    we have $P(\bigcup_{k=1}^{+\infty}A_k)=\sum_{k=1}^{+\infty}P(A_k)$
\end{lemma}

\begin{definition}[$ \sigma-algebra $ ]{}
A $ \sigma-algebra $ ($ \sigma-field $) $ \mathcal{A}  $  is an collection of subsets of $ \Omega $ such that
\begin{itemize}
\item $ \Omega \in \mathcal{A}  $
\item if $ A_1,A_2,\cdots \in \mathcal{A} $ then $ \bigcup_{i=1}^{+\infty}A_i \in \mathcal{A} $
\item if $ A \in \mathcal{A}  $, then $ A^c \in \mathcal{A} $ 
\end{itemize}
\end{definition}
\begin{remark}[]{}
if $ A_1,A_2,\cdots \in \mathcal{A} $ then $ \bigcap_{i=1}^{+\infty}A_i \in \mathcal{A} $
\\$ \sigma $ -algebra represents the collection of information determined by partial derivatives
\end{remark}
\begin{example}{}{}
\begin{itemize}
\item coinflips infinity many times$$
    \Omega=\{(x_1,x_2,\cdots )x_i=0,1\}={(0,1)}^{\infty}
$$ 
\\After observing first outcome, $ f_1=\{\emptyset,\Omega,A_0,A_1\} $
\\Where $A_0=\{(0,x_2,x_3,\cdots),x_i=0,1\},A_1=\{(1,x_2,x_3,\cdots),x_i=0,1\} $
\\
\\After observing second outcome, $ f_2=\{\emptyset,\Omega,A_0,A_1,A_{00},A_{01},A_{10},A_{11}\} $
\\Where $A_{00}=\{(0,0,x_3,x_4,\cdots),x_i=0,1\},A_{01}=\{(0,1,x_3,x_4,\cdots),x_i=0,1\} \cdots$
\\
\\After observing first n outcomes, $$ f_n=\{\emptyset,\Omega,{(A_j)}_{j\in {(\sigma 1)}^n} \text{and finite disjioint unions}\}$$
and $$
    f_1\subseteq f_2\subseteq \cdots \subseteq f_n \subseteq \cdots
$$ 
\end{itemize}
\end{example}
\begin{proposition}{}{}
\begin{enumerate}[label=\circled{\arabic*}]
\item intersection of $ \sigma $-algebras is a $ \sigma $-algebra
\item Let $ \mathcal{E} $ be a collecion of subsets of $ \Omega $ 
$$
    \sigma(\mathcal{E})=\bigcap_{\substack{{f \supseteq  \mathcal{E}}\\{f \text{   is an algebra}}}}f \text{    is a smallest   } \sigma \text{-algebra that contains  } \mathcal{E}
$$ 
\item For any collection $\mathcal{E}$, we have $a(\mathcal{E}) \subseteq \sigma(\mathcal{E})$
\item For any collection $\mathcal{E}$, we have $\sigma(a(\mathcal{E})) = \sigma(\mathcal{E})$
\\\textbf{Hint:}
\begin{align*}
    a(\mathcal{E}) \subseteq \sigma(\mathcal{E}) &\rightarrow \sigma(a(\mathcal{E})) \subseteq \sigma(\mathcal{E})
    \\ \mathcal{E} \subseteq \sigma(a(\mathcal{E})) &\rightarrow \sigma(\mathcal{E}) \subseteq \sigma(a(\mathcal{E}))
\end{align*}
\end{enumerate}
\end{proposition}

\section[Recitation 1 (02-07) -- {Problem Solving (Algebra)}]{Recitation 1 (02-07) -- {Problem Solving}}
\subsection{Basic Set Theory}
\begin{definition}[$\cup,\cap,^c$]{}
\textbf{De Morgan's Law:}
\begin{align*}{}{}
    (\bigcup_{j\in J}A_j)^c&=\bigcap_{j\in J}A_j^c 
    \\(\bigcap_{j\in J}A_j)^c&=\bigcup_{j\in J}A_j^c
\end{align*}
\end{definition}
\begin{definition}[$\backslash$]{}
    $A \backslash B = A \cap B^c$
    \\Then $A \backslash B = A \cap B^c = A \backslash (A \cap B)= B^c \backslash A^c$
\end{definition}
\begin{remark}{}{}
$$
    \bigcap_{n \geq 1}A_n=A_1\backslash(\bigcup_{j \geq 2}(A_1-A_j)) \text{    (ex)}
$$ 
\end{remark}
\subsection{Limits of Sets}
\begin{definition}[Limit Sets]{}
Let $(A_n)_{n \geq 1}$ be a sequence of sets, then
$$
    B_k=\bigcup_{n \geq k}A_n, C_k=\bigcap_{n \geq k}A_n
$$ 
\\Then $ B_k $ is increasing, $ C_k $ is decreasing
\\Define:
\begin{align*}
    \limsup_{n \rightarrow +\infty}A_n=\lim_{k \rightarrow +\infty}B_k=\bigcap_{k\geq 1}\bigcup_{n\geq k}A_n
\\\liminf_{n \rightarrow +\infty}A_n=\lim_{k \rightarrow +\infty}C_k=\bigcup_{k\geq 1}\bigcap_{n\geq k}A_n
\end{align*} 
\end{definition}
\begin{definition}[liminf, limsup of sequence]{}
$$
    \limsup{an}=\inf_{k\geq 1}\sup_{n\geq k}a_n, \liminf{an}=\sup_{k\geq 1}\inf_{n\geq k}a_n
$$ 
When $ \limsup{A_n}=\liminf{A_n} $, Then we say $ \lim_{n \rightarrow +\infty}{A_n} $ exist
and $ \lim_{n \rightarrow +\infty}{A_n}=\limsup{A_n}=\liminf{A_n} $  
\end{definition}
In probability,
\begin{align*}{}{}
\limsup{A_n}&=\{A_n\text{ occurs infinitely often}\}
\\&= \{An.i.o\}
\\x \in \limsup{A_n}&\Leftrightarrow \forall k \in \mathbb{N}, \exists n \geq k \text{ such that } x \in A_n
\\&\Leftrightarrow A_n.i.o
\\\liminf{A_n}&=\{A_n\text{ occurs eventually}\}
\\&\Leftrightarrow \exists k \in \mathbb{N}, \forall n \geq k, x \in A_n
\end{align*}
\begin{remark}{}{}
\begin{enumerate}
    \item if $ A_n $ increases, then $ \limsup{A_n}=\lim_{n \rightarrow \infty}{A_n}=\bigcup_{n\geq1}{A_n}$
    \\if $ A_n $ decreases, then $ \liminf{A_n}=\lim_{n \rightarrow \infty}{A_n}=\bigcap_{n\geq1}{A_n}$
    \item $$
        {(\limsup{A_n})}^c=\liminf{A_n}, {(\liminf{A_n})}^c=\limsup{A_n}
    $$  
\end{enumerate}
\end{remark}
\subsection{Exercise}
\begin{enumerate}
    \item $A_k=\begin{cases}
        E, \text{if k is odd}
        \\ F, \text{if k is even}
    \end{cases}$
    \\Then $ \limsup{A_n}=E \cup F, \liminf{A_n}=E \cap F$
    \item Let $ f_n:\mathbb{R}\rightarrow\mathbb{R} $ , Let A=\{$x\in\mathbb{R}:\lim_{n\rightarrow\infty}f_n(x)=f(x)$\}
    Then,\begin{align*}
        A^c&=\bigcup_{k=1}^{+\infty}\bigcap_{m=1}^{+\infty}\bigcup_{n=m}^{+\infty}\{x:|f_n(x)-f(x)|\geq\frac{1}{k}\}
      \\  &=\bigcup_{k=1}^{+\infty}[\limsup_{n}\{x:|f_n(x)-f(x)|\geq\frac{1}{k}\}]
  \end{align*}
    \item Suppose that $ \lim_{n\rightarrow\infty}f_n(x)=f(x),\forall x \in \mathbb{R} $ 
    \\Then\begin{align*}
        \{x:f(x)\leq t\}&=\bigcap_{k=1}^{+\infty}\bigcup_{m=1}^{+\infty}\bigcup_{n=m}^{+\infty}\{x\in \mathbb{R}:f_n(x)< t+\frac{1}{k}\}
    \\&=\bigcap_{k=1}^{+\infty}[\liminf_{n}\{x:f_n(x)< t+\frac{1}{k}\}]
    \end{align*}
\end{enumerate}
\textbf{Proof:}
\begin{enumerate}
    \item ex2: Want 
    \begin{align*}
        A=\bigcup_{k\geq1}^{+\infty}\bigcap_{\geq=1}^{+\infty}\bigcup_{n\geq m}^{+\infty}\{x:|f_n(x)-f(x)|\geq\frac{1}{k}\} 
        \\f_n(x)\rightarrow f(x) \text{ iff }\forall \epsilon>0, \exists N \geq N, |f_n(x)-f(x)|<\epsilon
        \\ \{x:f_n(x)\rightarrow f(x)\}=\bigcap_{\epsilon>0}\bigcup_{N}\bigcap_{n\geq N}\{x:|f_n(x)-f(x)|<\epsilon\}, 
        \\ \text{ USE that } \{|f_n-f|<\epsilon\} \text{ is monotone increasing in } \epsilon
    \end{align*} 
\end{enumerate}
\textbf{Review on mapping:}
$$
    f:X\rightarrow Y, f^{-1}:Y\rightarrow X
$$ 
\textbf{Basic properties:}
\begin{itemize}
\item $f(\bigcup_{i\in I})=\bigcup_{i\in I}f(A_i)$
\item $f(\bigcap_{i\in I})=\bigcap_{i\in I}f(A_i)$
\end{itemize}
For ($ B_i $) subset of Y:
\begin{itemize}
\item if $ B_1 \subset B_2, f^{-1}(B_1)\subset f_-1(B_2)$
\item $ f^{-1}(\bigcup_{i\in I}B_i)=\bigcup_{i\in I}f^{-1}(B_i)$ 
\item $f^{-1}(B^c)=(f^{-1}(B))$
\end{itemize}
\begin{definition}[Indicator Mapping]{}
\begin{align*}
    1_A:X &\rightarrow \{0,1\}
    \\1_A(x)&=\begin{cases}
        1, x\in A
        \\0, x\notin A
    \end{cases}
\end{align*}
\end{definition}
\textbf{Exercise:}
\begin{enumerate}
    \item $1_{limsup{A_n}}=limsup{1_{A_n}}$
    \item $1_{liminf{A_n}}=liminf{1_{A_n}}$
    \end{enumerate}
\textbf{Proof:}
\\if \begin{align*}
    1_{limsup{A_n}}(x)=1 &\Leftrightarrow x\in limsup{A_n}
    \\&\Leftrightarrow \forall k\in \mathbb{N}, \exists n\geq k, x\in A_n
    \\&\Leftrightarrow \forall k\in \mathbb{N}, \exists n\geq k, 1_{A_n}(x)=1
    \\&\Leftrightarrow limsup{1_{A_n}}(x)\geq1 (\text{Definition of limsup})
    \\&\Leftrightarrow 1_{limsup{A_n}}(x)=1
\end{align*}
\textbf{Exercise:}
Let $ \mathcal{A}_1,\mathcal{A}_2  $ be algebras of $\Omega$
\begin{enumerate}
    \item show that $\underbrace{\mathcal{A}_1\cap \mathcal{A}_2}_{B\subset\Omega:B\in \mathcal{A}_1,B\in \mathcal{A}_2} $ is an algebra
    \item show that $ \underbrace{\mathcal{A}_1\cup \mathcal{A}_2}_{B\subset\Omega:B\in \mathcal{A}_1\text{ or }B\in \mathcal{A}_2} $ is an algebra iff $ \mathcal{A}_1\subseteq \mathcal{A}_2 $ or $ \mathcal{A}_2\subseteq \mathcal{A}_1 $
\end{enumerate}
\textbf{Proof:}
Suppose by contradiction that $$\exists A_1\in\mathcal{A}_1\text{ but } A_1\notin \mathcal{A}_2 \text{ and } A_2\in \mathcal{A}_2\text{ but } A_2\notin \mathcal{A}_1 \text{ and } \mathcal{A}_1\cup\mathcal{A}_2 \text{ is an algebra }$$
\\Therefore:
\begin{align*}
    A_1\cup A_2\in \mathcal{A}_1\cup\mathcal{A}_2, A_1\backslash A_2&=A_1\cap \underbrace{A_2^c}_{\mathcal{A}_1\cup\mathcal{A}_2}\in \mathcal{A}_1\cup\mathcal{A}_2
    \\A_2\backslash A_1&=A_2\cap {A_1^c}\in \mathcal{A}_1\cup\mathcal{A}_2
    \\\Rightarrow \text{ at least two of } 
\\(A_1\backslash A_2)\cup &(A_2\backslash A_1),A_1\backslash A_2,A_2\backslash A_1 \text{ are in } \mathcal{A}_1 \text{ or }\mathcal{A}_2
\\\textbf{Assume in } \mathcal{A}_1
\\\Rightarrow (\text{by } &\mathcal{A}_1 \text{ is an algebra, all three sets are in } \mathcal{A}_1)
\\\Rightarrow A_2&=(\underbrace{A_1\cup A_2}_{\in \mathcal{A}_1})\backslash(\underbrace{A_1\backslash A_2}_{\mathcal{A}_1})\in \mathcal{A}_1
\\\textbf{Contradiction!}
\end{align*}
\section{Lecture 3 (02-10)--(Content and Measure)}
\textbf{Recall $ \sigma $-algebra:}
\begin{example}[]{}
 \begin{enumerate}[label=\circled{\arabic*}] 
 \item We know $\Omega = \mathbb{R}$, \\$\mathcal{E} = \{\text{left open right closed intervals}\} = 
\begin{cases}
    (a,b], & -\infty \leq a < b < +\infty \\
    (a,+\infty),&a\in \mathbb{R}
\end{cases}$
\\Then we know:\begin{align*}a(\mathcal{E})&=\{\text{  "finite disjioint union of elements in }\mathcal{E}"\}
\end{align*}
What is $ \sigma(a(\mathcal{E})) $?
\\$\sigma(\epsilon)=$ Borel Sets $\mathcal{B}(R)$ 
\\Any "reasonable" subset of $ \mathbb{R} $ is in $ \sigma(\epsilon) $
\begin{itemize}
\item $(a,b)\in \sigma(\epsilon): (a,b)=\bigcup_{n\geq1}\underbrace{(a,b-\frac{1}{n}]}_{\in \sigma(\epsilon)}\in \sigma(\epsilon)$
\item any singleton $\{a\}\in \sigma(\epsilon)$, because $\{a\}=\bigcap_{n\geq1}\underbrace{(a-\frac{1}{n},a+\frac{1}{n})}_{\in \sigma(\epsilon)}\in \sigma(\epsilon)$
\item any countable set is in $\sigma(\epsilon)$ (e.g. $\mathbb{Q}$)
\item The set of trancsendental numbers is in $\sigma(\epsilon)$, because the set of algebraic numbers is countable
\end{itemize}
 \end{enumerate}
\end{example}
\begin{definition}[measurable set]{}
A pair $ (\Omega, F) $ , where $F$ is a $ \sigma $-algebra of $ \Omega $, is called a measurable space 
\\Any set $ A\in F $  is called a measurable set
\end{definition}
\subsection{Content and Measure}
\begin{definition}[Content]{}
Let $ \mathcal{A} $ be an algebra of $ \Omega $, A set function $ \mu: \mathcal{A} \rightarrow [0,+\infty) $ is called a content iff:
\begin{itemize}
\item $ \mu(\emptyset)=0 $
\item if A,B$\in \mathcal{A} $ and $ A\cap B=\emptyset $, then $ \mu(A\cup B)=\mu(A)+\mu(B) $ (finite additivity)
\end{itemize}
\end{definition}
\begin{lemma}[]{}
Let $ \mu:\mathcal{A} \rightarrow [0,+\infty)  $ be a content, $ \forall A,B \in \mathcal{A}  $ then:
 \begin{enumerate}[label=\circled{\arabic*}] 
 \item $ \mu(A)+\mu(B)=\mu(A\cup B)+\mu(A\cap B) $ 
 \item if $ A\subset B $, and $ \mu(A)<+\infty $ ,then $\mu(B\backslash A)=\mu(B)-\mu(A) $
 \item if $ A\subset B $, then $ \mu(A)\leq \mu(B) $
 \item if $ A_1,A_2,\cdots \in \mathcal{A} $ , then $ \mu(\bigcup_{i=1}^nA_i)=\sum_{i=1}^n\mu(A_i) $
 \item if $ A_1,A_2,\cdots \in \mathcal{A} $ , $ A_i\cap A_j=\emptyset \text{ and } \bigcup_{i=1}^{+\infty}\in A $ then $ \mu(\bigcup_{i=1}^{+\infty}A_i)\geq \sum_{i=1}^{+\infty}\mu(A_i) $
 \end{enumerate}
\end{lemma}
\textbf{Proof:}
\\finite add:
\\1:\begin{align*}{}{}
\mu(B)=\mu(A\cup B)+\mu(B\backslash A)\\
\mu(A)+\mu(B\backslash A)=\mu(A\cup B)\\
\Rightarrow \mu(A)+\mu(B)=\mu(A\cup B)+\mu(A\cap B)
\end{align*}
4:\begin{align*}{}{}
\text{Let } B_1=A_1,B_2=A_2\backslash A_1,B_3=A_3\backslash(A_1\cup A_2),\cdots\\
\text{Then } B_j \text{ are disjoint and }\bigcup_{i=1}^{n}B_i=\bigcup_{i=1}^{n}A_i \\
\text{By finite add for }B_j:\\
\mu(\bigcup_{i=1}^{n}A_i)=\mu(\bigcup_{i=1}^{n}B_i)=\sum_{i=1}^{n}(\mu(A_i)\backslash \bigcup_{i=1}^{j=1}A_j) \leq \sum_{i=1}^{n}\mu(A_i)
\end{align*}
5:\begin{align*}{}{}
\text{For } \forall n,\\
\mu(\bigcup_{i=1}^{+\infty }A_i)\geq \mu(\bigcup_{i=1}^{n}A_i) =\sum_{i=1}^{n}\mu(A_i) \text{ send n}\nearrow +\infty \text{ to conclude}\\
\end{align*}
\begin{remark}[]{}
In general. $ \mu(\bigcup_{i=1}^{+\infty}A_i)\neq \sum_{i=1}^{+\infty}\mu(A_i) $
although $ \mu(\bigcup_{i=1}^{n}A_i)= \sum_{i=1}^{n}\mu(A_i) $
\\require continuity of $\mu $ 
\end{remark}
\textbf{Counterexample:}
\begin{itemize}
\item Let $ \Omega=\mathbb{R},\mathcal{A} =a(\epsilon) $, $ \forall A\in \mathcal{A} $,\\
$$ \mu(A)=\lim_{L\rightarrow+\infty}\frac{\overbrace{|A\cup[0,L]|}^{\text{Length of interval}}}{L} \text{ "Density of A in }[0,+\infty)"$$
\\Then $ \mu $ is a content.
\\Take $ A_i=(i,i+1] $ , then $ \mu((i,i+1])=0, $ But $ \mu(\bigcup_{i=0}^{+\infty})=\mu((0,+\infty))=1$
\item For $ \mathbb{R},a(\epsilon) $, Given $ A\in a(\epsilon) $ 
\begin{align*}{}{}
\mathring{A}&=\{x\in A:\exists r_x>0, s.t. (x-rx,x+rx)\subset A\}\\
\partial{A}&=\bar{A}\backslash\mathring{A}
\end{align*}
$$
\mu(A)=\begin{cases}
        2, & \text{if } 0\in \mathring{A}
        \\1, & \text{if } 0\in \partial{A}\\
        0, & \text{ else }
    \end{cases}
$$
Then $ \mu $ is a content
\\However, $A_i=(\frac{1}{i+1},\frac{1}{i}],\mu(A_i)=0,\text{ But }\mu(\bigcup_{i=1}^{+\infty})=\mu((0,1])$
\end{itemize}
\textbf{Example:}
\begin{itemize}
\item (Discrete Probability)
$$
    \Omega=\{\omega_1,\cdots,\omega_n\} \text{ finite set, }, F=P(\Omega)
$$ 
Set $ A_i=\{\omega_i\} ,P(A_i)=P(\omega_i)=P_i  $ such that $\sum_{i=1}^{n}P_i=1$
\\Then $ P:P(\Omega)\rightarrow[0,1] $ defines a content on $ P(\Omega) $ by extending the P using finite additivity:
$$
    \forall A\in P(\Omega), P(A)=\sum_{\omega\in A}P(\omega)
$$  
\item $ \Omega=\mathbb{R},\text{ algebra }\mathcal{A} =a(\epsilon)=\{\text{ finite disjoint umiom of elements in }\epsilon\} $ 
\\Define $ m:a(\epsilon)\rightarrow [0,+\infty) $ , set $ m([a,b]) $ =b-a,
\\and extend by additivity:$$
    m(I)=\sum_{i=1}^{n}m(I_j), \text{ if } I=I_1\cup\cdots\cup I_n, I_j\cap I_i=\emptyset
$$ 
$\Rightarrow$ m is a content on $ (\mathbb{R},a(\epsilon))$
\\m can be further extended to $ (\mathbb{R},\sigma(\epsilon)) $, called Lebesgue Measure
\end{itemize}
\begin{definition}[countably additive]{}
A content $ \mu:\mathcal{A} \rightarrow[0,+\infty) $ is countably additive if:$$
    \mu(\bigcup_{i=1}^{+\infty}A_i)=\sum_{i=1}^{+\infty}\mu(A_i) \text{ for every disjioint }A_1,A_2,\cdots \in \mathcal{A}
$$  
\end{definition}
\begin{definition}[measure]{}
Let $ (\Omega,F) $ be an measurable space, then a content $ \mu:F\rightarrow [0,+\infty) $ that is countably additive is called a measure 
\end{definition}
\begin{lemma}[]{}
$ m:a(\epsilon)\rightarrow[0,+\infty) \text{ is countably additive} $ 
\end{lemma}
\textbf{Proof:}
\\Let $ A_1,A_2,\cdots \in a(\epsilon),(A_k) \text{ disjioint },A:=\bigcup_{i=1}^{+\infty}A_i\in a(\epsilon) $
\\Want to show :$$
    m(A)=\sum_{i=1}^{+\infty}m(A_i)
$$ 
we can write, using $ A,(A_k)  \in  a(\epsilon) $, $ A=\bigcup_{j=1}^{n}I_j$, where $ I_j \in \epsilon$ and $ (I_j) $ disjioint 
\\$ A_j=\bigcup_{k=1}^{n_i}J_{ik} $, where $ J_{ik}\in\epsilon $,$ (J_{ik}) $ disjioint
\\Then:\begin{align*}{}{}
m(A)&=\sum_{j=1}^{n}m(I_j)\\
&=\sum_{j=1}^{n}\sum_{j=1}^{+\infty}\sum_{k=1}^{n_i}m(I_j\cap J_{ik})\\
&=\sum_{i=1}^{+\infty}m(\underbrace{\bigcup_{j=1}^{n}I_j}_A\cap (\underbrace{\bigcup_{k=1}^{n_i}J_{i_{jk}}}_{A_i}))\\
&=\sum_{i=1}^{+\infty}m(A_i)
\end{align*}   
\section{Lecture 4 (02-12)--{Measure and Extension}}
\begin{align*}{}{}
\Omega=\mathbb{R},m:\epsilon&\rightarrow[0,+\infty),\text{ such that }\begin{cases}
    m([a,b])=b-a, \\ m((a,+\infty))=+\infty
\end{cases}\\
&\text{extend  m  to } a(\epsilon): \forall A\in a(\epsilon),\\
\text{if } A=\bigcup_{i=1}^{n}I_j, I_j &\text{ disjioint, } m(A)=\sum_{j=1}^{n}m(I_j)
\end{align*}
\textbf{Fact:}
\\if $ I\in \epsilon $ s.t. $ I=\bigcup_{i=1}^{+\infty}I_i,(I_i)\text{ disjoint and }I_j\in\epsilon $
\\Then \begin{align*}{}{}
m(I)&=|I|\\
&=\sum_{i=1}^{+\infty}|I_i|=\bigcup_{i=1}^{+\infty}m(I_i)
\end{align*}  
\begin{lemma}[]{}
we want to prove m is a countably additive content on $(\mathbb{R},a(\epsilon))$\\
\begin{align*}{}{}
\text{Let }  A_j\in a(\epsilon),A&=\bigcup_{j=1}^{\infty}\in a(\epsilon),(A_j)\text{ disjoint} \\
\exists (I_i) \text{ such that } I_i\in\epsilon,\text{ disjoint } A&=\bigcup_{i=1}^{n}I_i\\
\exists (J_{ij}) \text{ such that } J_{ij}\in\epsilon,\text{ disjoint } A_j&=\bigcup_{k=1}^{j}J_{ik}\\
m(A)\underbrace{=}_{\text{def}}\sum_{i=1}^{n}m(I_i)&=\sum_{i=1}^{n}m(\bigcup_{j,k}\underbrace{(I_i)\cap(J_{ik})}_{\in\epsilon})\\
&\underbrace{=}_{\text{fact}}\sum_{i=1}^{n}\sum_{j=1}^{+\infty}\sum_{k=1}^{n_j}m(I_i\cap I_{jk})\\
&\underbrace{=}_{\text{finite add}}\sum_{j=1}^{+\infty}m(A_j)
\end{align*}
\end{lemma}
\begin{theorem}[]{}
m extends to a measure on $(\mathbb{R},\sigma(\epsilon)=\mathcal{B}(\mathbb{R}))$ It is the unique measure on $(\mathbb{R},\mathcal{B} (\mathbb{R}))$ such that
$$
    m([a,b])=b-a
$$ 
\end{theorem}
\begin{definition}[]{}
If $ \mu $ is a measure on $ (\Omega,F) $, then $ (\Omega,F,\mu )$ is said to be a measure space \\
If $ \mu(\Omega)=1 $, then $ (\Omega,F,\mu )$ is said to be a probability space
\end{definition}
\begin{lemma}[]{}
Let ($\Omega,F,\mu$) be a measure space, then:
 \begin{enumerate}[label=\circled{\arabic*}] 
 \item stability: Let $ A_1,A_2,\cdots \in F $, then $ \mu(\bigcup_{i=1}^{+\infty}A_i) \leq \sum_{i=1}^{+\infty}\mu(A_i) $
 \item continuity from below: Let $ A_1,A_2,\cdots \in F $, $ A_1\subseteq A_2\subseteq \cdots $, then $$ \mu(\bigcup_{i=1}^{+\infty}A_i)=\lim_{i\rightarrow+\infty}\mu(A_i)=\mu(\lim_{i\rightarrow+\infty}A_i) $$
 \item continuity from above: Let $ A_1,A_2,\cdots \in F $, $ A_1\supseteq A_2\supseteq \cdots $,\\($ \mu(A_i)<+\infty $ we need this for the Counterexample $ A_i=[i,+\infty) $ ) then $$ \mu(\bigcap_{i=1}^{+\infty}A_i)=\lim_{i\rightarrow+\infty}\mu(A_i)=\mu(\lim_{i\rightarrow+\infty}A_i) $$
  \end{enumerate}
\end{lemma}
\textbf{Proof:}
 \begin{enumerate}[label=\circled{\arabic*}] 
 \item Let $ B_1=A_1,B_2=A_2\backslash A_1,\cdots,B_n=A_n\backslash \bigcup_{i=1}^{n-1}A_i $,
 \\Then $ (B_i) $ disjoint, $ \bigcup_{i=1}^{+\infty}A_i=\bigcup_{i=1}^{+\infty}B_i $   
 $$
    \mu(\bigcup_{i=1}^{+\infty}A_i)=\mu(\bigcup_{i=1}^{+\infty}B_i)=\sum_{i=1}^{+\infty}\mu(B_i)=\sum_{i=1}^{+\infty}\mu(A_i\backslash\bigcup_{j=1}{i-1}A_j)\leq \sum_{i=1}^{+\infty}\mu(A_i)
 $$ 
 \item Let $ B_1=A_1,B_2=A_2\backslash A_1,\cdots,B_n=A_n\backslash A_{n-1}\cdots $, 
 \begin{align*}
    \mu(\bigcup_{i=1}^{+\infty}A_i)=\mu(\bigcup_{i=1}^{+\infty}B_i)&=\sum_{i=1}^{+\infty}\mu(B_i)\\
    &=\mu(A_1)+\sum_{i=2}^{+\infty}(\mu(A_i)- \mu A_{i-1})\\
    &=\lim_{n\rightarrow+\infty}(\mu(A_1)+\sum_{i=2}^{n}(\mu(A_i)- \mu A_{i-1}))\\
    &=\lim_{n\rightarrow+\infty}\mu(A_n)
 \end{align*}
 \item \begin{align*}{}{}
 \mu(A_1)-\mu(\bigcap_{i=1}^{+\infty}A_i)&=\mu(A_1\backslash \bigcap_{i=1}^{+\infty}A_i)=\mu(\bigcup_{i=1}^{+\infty}A_1\backslash A_i)\\
 \text{Since }A_1 \nearrow ,\text{ by 2}: &=\lim_{n\rightarrow+\infty}(\mu(A_1)-\mu(A_i))=\mu(A_1)-\lim_{n\rightarrow+\infty}\mu(A_i)
 \end{align*} 
 \end{enumerate}
 \begin{definition}[$ \sigma $-finite measure ]{}
 Given a measure space $ (\Omega,F,\mu) $,  $ \mu $ is said to be finite if $ \mu(\Omega)<+\infty $ \\
 $ \mu $ is $ \sigma $-finite if there exists $ {(E_i)}_{i=1}^{+\infty} $ such that $ \bigcup_{i=1}^{+\infty}E_i\in \Omega$ and $ \mu(E_i)<+\infty $
 \end{definition}
 \begin{example}[]{}
  $(\mathbb{R},\mathcal{B} (\mathbb{R}),m)$ is $ \sigma $-finite
 \end{example}
 \begin{definition}[]{}
 if $ F\in \mathcal{F} $ is such that $ \mu(F)=0 $, then $ F $ is called a $ \mu $-null set 
 \end{definition}
 \begin{example}[]{}
  $ m(\{a\})=0 $ because \begin{align*}{}{}
  m(\{a\})&=m(\bigcap_{n\geq1}(a-\frac{1}{n},a])=\text{(continuous from above)}\lim_{n\rightarrow+\infty}m((a-\frac{1}{n},a])=0\\
  m(Q)&=\sum_{q\in Q}m(\{q\})=0
  \end{align*} 
 \end{example}
 \textbf{Recall:}
 \\Start with $ (R,a(\epsilon),m) $ m is content + countably additive
 \\extention: $ (R,a(\epsilon),m) \rightarrow (R,\sigma(\epsilon),m) $ m is measure
 \begin{theorem}[Caratheodory Extention Theorem]{}
 Let F be an algebra on $ \Omega $ , $ \mu $ be a countably additive content on ($ \Omega,F $), \\
 If $ \mu $ is $ \sigma $-finite, then $\mu$ extends to a measure on $ (\Omega,\sigma(F)) $
 \end{theorem}
 \begin{example}[]{}
  Let $\epsilon=\begin{cases}
    (a,b], & -\infty \leq a < b < +\infty \\
    (a,+\infty),&a\in \mathbb{R}
  \end{cases}$
Let $ m_F:a(\epsilon)\rightarrow[0,+\infty) $ be a content, such that $$ m_F([a,b])=F(b)-F(a) , m_F((a,+\infty))=F(+\infty)-F(a) $$
Where F is a right continuous increasing function on $ \mathbb{R} $, $$ F(\pm \infty)=\lim_{x\rightarrow \pm +\infty}F(x) $$
Then $ m_F $ is a countably additive content on $ (\mathbb{R},a(\epsilon)) $
\\By the extension theorem, $ m_F $ extends to a measure on $ (\mathbb{R},\sigma(a(\epsilon))=\sigma(\epsilon)=\mathcal{B} (\mathbb{R})) $ \\
(F(x)=(x) gives Lebesgue)
 \end{example}
 \begin{definition}[Lebesgue-Stieltjes mesures]{}
 think of $ m_F(A)=\int_{A}\underbrace{dF(x)}_{\text{R-S integral}} $ , F is the distribution function of the measure
 \end{definition}
\subsection{$ \pi $ and $\lambda  $ system  }
 \begin{definition}[$ \pi $ and $\lambda  $ system   ]{}
 Let $ C $ be a collection of sets of $\Omega$\\
 $ C $ is a $ \pi $-system if:
 \begin{itemize}
 \item $ \emptyset\in C $
 \item $ \forall A,B\in C, A\cap B \in C $ 
 \end{itemize}
 C is a $ \lambda $-system if:
 \begin{itemize}
 \item $ \Omega\in C $
 \item if $  A,B\in C, \text{ and }A\subseteq B, \text{ then }B\backslash A \in C $
 \item if $ A_1,A_2,\cdots \in C, \text{ and }A_1\subseteq A_2\subseteq \cdots, \text{ then }\bigcup_{i=1}^{+\infty}A_i\in C $
 \end{itemize}
 \end{definition}
 \begin{example}[]{}
  $ \epsilon=\begin{cases}
    (a,b]\\
    (a,+\infty)
  \end{cases} $ is a $ \pi $-system 
 \end{example}
 \textbf{Exercise:}
 if $ C $ is both a $ \pi $-system and a $ \lambda $-system, then it is a $ \sigma $-algebra
 \begin{lemma}[Dynkins Lemma]{}
 Let C be a $ \pi $-system, then any $ \lambda $-system containing C also contains the $\sigma(C)$
 \\\textbf{Hint:} show that any such  $\lambda$-system is also a $\pi$-system
 \end{lemma}
 \begin{theorem}[Uniqueness Theorem]{}
 Let C be a $ \pi $-system, Let $ \mu_1,\mu_2 $ be two finite measures on $ (\Omega,\sigma(C)) $\\
 Suppose that $ \mu_1(A)=\mu_2(A) \text{ and }\mu_1(\Omega)=\mu_2(\Omega)$ on $ C $, then $ \mu_1(A)=\mu_2(A) $ on $\sigma(C) $
 \end{theorem}
 \begin{proof}
    \begin{align*}{}{}
    D&=\{A\in\sigma(C):\mu_1(A)=\mu_2(A)\}\\
\text{We know } &C\in D. \text{ wnat to show D is a } \lambda\text{-system}\\
\text{If so , by Dynkins Lemma, }&\sigma(C)\subseteq D,\text{ so that }D=\sigma(C)\\
    \end{align*}
    Check $ D $ is a $ \lambda $-system:
    \begin{itemize}
    \item $ \Omega\in D $ follows from $ \mu_1(\Omega)=\mu_2(\Omega) $
    \item if $ A,B\in D, A\subseteq B  \rightarrow \mu_1(A)=\mu_2(A),\mu_1(B)=\mu_2(B) \\
    \mu_1(B\backslash A)=\mu_1(B)-\mu_1(A)=\mu_2(B)-\mu_2(A)=\mu_2(B\backslash A) $\\
    So, $\Rightarrow B\backslash A\in D$
    \item if $ A_n\in D $ and $ A_n $ increasing, $ \lim_{n\rightarrow+\infty}A_n=A $\\
    $$
        \mu_2(A)=\mu_2\lim_{n\rightarrow+\infty}(A_n)\underbrace{=}_{\text{cont.}}\lim_{n\rightarrow+\infty}\mu_2(A_n)=\lim_{n\rightarrow+\infty}\mu_1(A_n)=\mu_1(A)
    $$   
    \end{itemize}
 \end{proof}
 \begin{remark}[]{}
 Also hold for $ \mu_1,\mu_2 ,\sigma$-finite
 \end{remark}

 \section{Recitation 2 (02-14)-Exercise}
 \subsection*{EX1}
 Let $ \Omega  $ be a countable set,$$
    A=\{A\subseteq \Omega:\text{ A is finite , or} A^c \text{ is finite}\}
 $$  
  \begin{enumerate}[label=\circled{\arabic*}] 
  \item show that A is an algebra
  \item Let $ P:A\rightarrow[0,+\infty) $ that $P(A)=\begin{cases}
    0, & \text{if A is finite} \\
    1, & \text{if }A^c\text{ is finite}
  \end{cases} $
  \\Is P a content/measure?
  \\Solution:
  (1):\begin{itemize}
  \item $\emptyset \in \mathcal{A}$ because $\emptyset$ is finite 
  \item $ A\in \mathcal{A} \Rightarrow A^c\in \mathcal{A} $ by defintation
  \item $ A,B\in \mathcal{A} $
  \begin{itemize}
  \item if $ A,B $ are finite, then $ A\cup B $ is finite
  \item if one of A, B is countably infinite , say B $$
    (A\cup B)^c=A^c\cap B^c \text{ is finite} \Rightarrow A\cup B\in \mathcal{A}
  $$ 
  \end{itemize} 
  \end{itemize}
  (2): $ \Omega=\{\omega_1,\omega_2,\cdots\} $ \\
  Let $ A_i=\{\omega_i\} $ so that $ P(\bigcup_i\{\omega_i\})=P(\Omega)=+\infty $ But $ \sum P(\omega_1)=0 $ 
  \end{enumerate}
  \subsection*{EX2}
  Let $ \Omega $ be an uncountable set, $ A=\{\{\omega\},\omega\in\Omega\} $ , compute $ \sigma(A) $ and justify:
\\Solution:
$$
    \sigma(A)=\{\underbrace{A\subseteq \Omega: A \text{ is countable or }A^c \text{ is countable }}_{f}\}
$$     
\begin{proof}
    \begin{itemize}
    \item f is a $ \sigma $-algebra 
     \begin{itemize}
    \item $\emptyset \in f$ because $\emptyset$ is countable
    \item $ A\in f \Rightarrow A^c\in f $ by definition
    \item if $ A_1,A_2,\cdots \in f $ \begin{itemize}
    \item if $ A_i $ are countable, then $ \bigcup_{i=1}^{+\infty}A_i $ is countable
    \item if one of $ A_i $ is uncountable, say $ A_1 \text{ then} A_1^c \text{ countable }$, then $ (\bigcup_{i=1}^{+\infty}A_i)^c $ is uncountable
    \end{itemize}
    \item $\sigma(A)\subseteq f $ by $ \sigma(A) $ is minimal
    \item $\sigma(A)\supseteq f $ \begin{itemize}
    \item if $ A\in f $ countable, $ A=\{\omega_1,\omega_2,\cdots\}= $
    \item if $ A^c\in f $ countable, then
    \end{itemize}
    \end{itemize}
    \end{itemize}
\end{proof}
\subsection*{EX3}
Let $ \Omega=\mathbb{R} $
\begin{align*}{}{}
C_1&=\{(-\infty,b],b\in \mathbb{R}\}\\C_2&=\{(a,b],-\infty\leq a<b<+\infty\}
\\C_3&=\{(a_1,b_1]\cup(a_2,b_2]\cup(a_3,b_3]\cdots(a_n,b_n],-\infty\leq a_1<b_1\leq a_2<\cdots<b_n<+\infty\}
\end{align*}
show that $ \sigma(C_1)=\sigma(C_2)=\sigma(C_3) $
\\Solution:
\begin{proof}
    $ \sigma(C_1)=\sigma(C_2) $ 
    \begin{itemize}
    \item $ C_1\subseteq \sigma(C_2) $ because $(-\infty,b]\in \sigma(C_2) \Rightarrow \sigma(C_1)\subseteq\sigma(C_2)$ because $ \sigma(C_1) $ is minimal
    \item $ C_2\subseteq \sigma(C_1) $ because $ (a,b]=(-\infty,b]\backslash(-\infty ,a]\in \sigma(C_1) \Rightarrow \sigma(C_2)\subseteq\sigma(C_1)$ because $ \sigma(C_2) $ is minimal 
    \end{itemize}
\end{proof}
 \subsection{Special Case:}
 $ \Omega=\{1,2,\cdots,N\}, F=P(\Omega),P(\{1\})=\cdots=P\{N\}=\frac{1}{N} $
 \\For event:$ E\in \Omega,P(E)=\frac{|E|}{|\Omega|} $
 \begin{definition}[Inclusion-Exclusion Principle]{}
 $P(E_1\cup E_2\cup\cdots\cup E_n)$= $ \sum_{i=1}^{n}P(E_i)-\sum_{i<j}P(E_i\cap E_j)+\sum_{i<j<k}P(E_i\cap E_j\cap E_k)+\cdots+(-1)^{r-1}\sum_{i_1<\cdots<i_r}P(E_{i_1}\cap\cdots\cap E_{i_r})+\cdots+(-1)^{n-1}P(E_1\cap E_2\cap\cdots\cap E_n)$
 \end{definition}  
 \begin{remark}[]{}
  \begin{enumerate}[label=\circled{\arabic*}] 
  \item Prove by induction
  \item $ P(E_1\cup E_2\cup\cdots\cup E_n)\leq \sum P(E_i) $ 
  \\$ P(E_1\cup E_2\cup\cdots\cup E_n)\geq \sum P(E_i)-\sum P(E_i\cup E_j) $
  \\ $ P(E_1\cup E_2\cup\cdots\cup E_n)\leq \sum P(E_i)-\sum P(E_i\cap E_j)+ \sum P(E_i\cap E_j\cap E_k)$
  \end{enumerate}
 \end{remark}
 \begin{example}[Brithday Problem]{}
  N people, what is the P[at least two people have the same birthday]?
  \\Solution:
    \begin{align*}{}{}
        \Omega&=\{(x_1,\cdots,x_n,x_i\in\{1,\cdots,365\})\} |\Omega|=365^N \\
        A^c&=\{(x_1,\cdots,x_N)\in \Omega:x_i\neq x_j\} |A^c|=365\cdot 364\cdots(365-N+1)\\
        P(A)&=\frac{|A^c|}{|\Omega|}=1\cdots(1-\frac{1}{365})\cdots(1-\frac{N-1}{365})(use 1-x \leq e^{-x})\\
        &\leq e^{-\sum_{i=0}^{N-1}\frac{i}{365}}\\
        &=e^{-\frac{N(N-1)}{730}}
    \end{align*}
    in fact if N>23, then $P(A^c)<\frac{1}{2}$
 \end{example}
\section{Lecture 5 (02-17)}
\textbf{Useful $ \pi $-system that generates B(R):}
 \begin{enumerate}[label=\circled{\arabic*}] 
 \item $\mathcal{E}=\begin{cases}
    =\{(a,b],-\infty\leq a<&b<+\infty\}\\
    =\{(a,+\infty), a&\in R\}
 \end{cases}$
 \item $ \mathcal{E}_1=\{(a,b],-\infty\leq a<b<+\infty\} $ 
 \item $ \mathcal{E}_2=\{(a,b),-\infty\leq a<b\leq+\infty\}$ 
 \item $ \mathcal{E}_{\text{open}}=\{A\in R,A \text{ open }\} $
 \item $ \mathcal{E}_{\text{closed}}=\{A\in R,A \text{ closed }\}  $ 
 \item $ \mathcal{E}_{\text{half}}  =\{(-\infty,b],b\in \mathbb{R}\} $
 \end{enumerate}
 easy to check $ \pi $-system
 \\Note: \begin{itemize}
 \item A open iff $ \forall x\in A,\exists \mathcal{E}_x>0, \text{ s.t. } (x-\mathcal{E}_2,x+\mathcal{E}_2)\subseteq A$ 
 \item A closed iff $ A^c $ open
 \end{itemize} 
 \begin{proof}
    (2)\begin{align*}{}{}
    \mathcal{E}_1&\subseteq \sigma(\mathcal{E}) \Rightarrow \sigma(\mathcal{E}_1)\subseteq \sigma(\mathcal{E})\\
    \mathcal{E}&\subseteq \sigma(\mathcal{E}_1),(a,+\infty)=\bigcup_{n\geq1}\underbrace{(a,a+n]}_{\mathcal{E}_1}\in \sigma(\mathcal{E}_1)\\
    \Rightarrow &\sigma(\mathcal{E})\subseteq \sigma(\mathcal{E}_1)
    \end{align*}
    (3)\begin{align*}{}{}
        \mathcal{E}_1&\subseteq \sigma(\mathcal{E}_2): (a,b]=\bigcap_{n\geq1}\underbrace{(a,b+\frac{1}{n})}_{\mathcal{E}_2}\in \sigma(\mathcal{E}_2)\\
        \mathcal{E}_2&\subseteq \sigma(\mathcal{E}_1): (a,b)=\bigcup_{n\geq1}\underbrace{(a,b-\frac{1}{n})}_{\mathcal{E}_2}\in \sigma(\mathcal{E}_2)\\
    \end{align*}
    (4)\begin{align*}{}{}
    \mathcal{E}_2\subseteq \mathcal{E}_{\text{open}}\Rightarrow \sigma(\mathcal{E}_2)\subseteq \sigma(\mathcal{E}_{open})\\
    \end{align*}
    \textbf{Fact:}
    \\ every open set $ A\subseteq \mathbb{R},A=\bigcup{i=1}^{+\infty}(\underbrace{(x_i-\mathcal{E}_i,x_i+\mathcal{E}_i)}_{\mathcal{E}_2}\in\sigma(\mathcal{E}_2)) $ 
    \\$ \Rightarrow \mathcal{E}_{open}\subseteq \sigma(\mathcal{E}_2)\Rightarrow \sigma(\mathcal{E}_{open})\subseteq \sigma(\mathcal{E}_2) $ \\
    (e)\\
    \begin{align*}{}{}
    A \text{ open } &\Leftrightarrow A^c \text{ closed implies }\\
    \mathcal{E}_{\text{closed}}&\subseteq \sigma(\mathcal{E}_{\text{open}})\Rightarrow \sigma(\mathcal{E}_{\text{closed}})\subseteq \sigma(\mathcal{E}_{\text{open}})\\
    \mathcal{E}_{\text{open}}&= \sigma(\mathcal{E}_{\text{closed}})
    \end{align*}
 \end{proof}
 \begin{example}[mismatch]{}
  N men and N hats\\
  P[no one finds his own hats]=?
    \\Solution:
   \\$ E_i=\{i^{th} \text{ letter in } i^{th} \text{envelope}\}$
   \\Want to compute $ P(\bigcap_{i=1}^{n}E_i^c)=1-P(\bigcup_{i=1}^{n}E_i) $
   By inclusion-exclusion:\begin{align*}{}{}
    P(\bigcup_{i=1}^{n}E_i)&=\sum_{i=1}^{n}P(E_i)-\sum_{i<j}P(E_i\cap E_j)+\cdots+(-1)^{n-1}P(\sum_{i=1}^{n}E_{i1}\cap\cdots\cap E_{ir})\\
    P(\sum_{i=1}^{n}E_{i1}\cap\cdots\cap E_{ir})&=\frac{|E_{i1}\cap\cdots\cap E_{ir}|}{|\Omega|}=\frac{(n-r)!}{n!}\\
    \sum_{i_1<\cdots<i_r}P(E_{i1}\cap\cdots\cap E_{ir})&=\binom{n}{r}\frac{(n-r)!}{n!}=\frac{1}{r!}\\
    P(\bigcup_{i=1}^{n}E_i)&=1-\frac{1}{2!}+\frac{1}{3!}+\cdots+(-1)^{N-1}\frac{1}{N!}\\
    P(\bigcap_{i=1}^{n}E_i^c)&\rightarrow1-\frac{1}{e}
   \end{align*}
 \end{example}
 \textbf{Exercise:}\begin{itemize}
 \item Circle: 10 couples, \\
 P[no couple sit next to each other]=?
 \item Texas Holder:\\
 P[Sraight]=?
 P[Full House]=?
 sol: \begin{align*}{}{}
 |\Omega|&=\binom{52}{5}  \\
    |Srtaight|&=10\times(4^5-4)  \\
    P[Straight]&=\frac{10\times(4^5-4)}{\binom{52}{5}}\approxeq 0.0039\\
    |Full House|&=13\times\binom{4}{3}\times 12\times\binom{4}{2}  \\
    P[Full House]&=\frac{13\times\binom{4}{3}\times 12\times\binom{4}{2}}{\binom{52}{5}}\approxeq 0.0014
 \end{align*} 
\end{itemize}
Conditional Probability:\\
"If the event B has occurred, what is the probability of event A?" $ P[A\backslash B] $ 
\\N experiment: natural $$
    P[A\backslash B]=\frac{\text{number of occurance of both A and B}}{\text{ number of occurance of B}}=\frac{P[A\cap B]}{P[B]}
$$ 
\begin{definition}[]{}
If $ P[B]>0 $, then the conditional probability of A given B is $ P[A\backslash B]=\frac{P[A\cap B]}{P[B]} $.
\end{definition}
\begin{example}[]{}
 \begin{enumerate}[label=\circled{\arabic*}] 
 \item 2 kids \begin{itemize}
 \item $ P[\text{two boys}\backslash\text{at least one boy}]=? $ 
 \item $ P[\text{two boys}\backslash\text{younger kid is boy}]=? $ 1/2
 \item $ P[\text{two boys}\backslash\text{at least one boy born on Tuesday}]=? $
 \end{itemize}
 (1)
 \begin{align*}{}{}
 A&=\{BB\},B=\{BG,GB,BB\}\\
P[A\backslash B]&=\frac{P[A\cap B]}{P[B]}=\frac{P[A]}{P[B]}=\frac{1/4}{3/4}=\frac{1}{3}\\
 \end{align*}
 (3)
 \begin{align*}{}{}
 \Omega&=\{B_iB_j,B_iG_j,G_iB_j,G_iG_j,i,j=1,\cdots,7\}\\
 A&=\{B_iB_j,i,j=1,\cdots,7\}\\
 B&=\{B_2B_j,B_iB_2,B_2G_j,i,j=1,\cdots,7\} \text{ 13+14 elements}\\
 A\cap B&=\{B_2B_j,B_iB_2,i,j=1,\cdots,7\} \text{ 13 elements}\\
    P[A\backslash B]&=\frac{13}{27}
 \end{align*}
 \end{enumerate}
\end{example}
\section{Lecture 6 (02-19)-{Conditional Probability}}
\begin{definition}[Law of total probability]{}
    $$
    P[A|B]=\frac{P[A\cap B]}{P[B]}
    $$ 
 $$P[A]=P[A|B]\cdot P[B]+P[A|B^c]P[B^c]$$
 More generally, Let $(B_i)_{i=1}^{n}$ be a partition of $\Omega$ (i.e. $B_i\cap B_j=\emptyset$ and $\bigcup_{i=1}^{n}B_i=\Omega$), then:
    $$P[A]=\sum_{i=1}^{n}P[A|B_i]P[B_i]$$
\end{definition}
\textbf{Proof:} 
\begin{align*}{}{}
P[A]=P[A\cap B]+P[A\cap B^c]&=P[A|B]P[B]+P[A|B^c]P[B^c]\\
\end{align*}
\begin{example}[]{}
 \begin{enumerate}[label=\circled{\arabic*}] 
 \item draw balls randomly from box A(3B2W) to B(4B3W) ,then draw from B randomly,P[The second draw is Black]
 \\By Law of total probability: P=P[2nd Black|1st Black]P[1st Black]+P[2nd Black|1st White]P[1st White]\\
 =$ \frac{5}{8}\times \frac{3}{5}+\frac{1}{2}\times\frac{2}{5}=\frac{23}{40} $ 
 \end{enumerate}
\end{example}
\textbf{Reverse Question:}  
\\If A happens, which $ B_i $ is the most likely? \textbf{Bayes' Formula:}
$$
    P[B_i|A]=\frac{P[B_i\cap A]}{P[A]}=\frac{P[A|B_i]P[B_i]}{\sum_{i=1}^{n}P[A|B_i]P[B_i]}
$$ 
\begin{example}[]{}
 \begin{enumerate}[label=\circled{\arabic*}] 
 \item Covid Test:\begin{itemize}
 \item false negative:P[negative|infected]=0.05
 \item false positive:P[positive|not infected]=0.01
 \end{itemize}
 Suppose 5\% of the population are infected, then P[infected|positive]=?
 \\\begin{solution}
   $$ P[V|P]=\frac{P[P|V]P[V]}{P[P|V]P[V]+P[P|V^c]P[V^c]}=\frac{5}{6}$$
 \end{solution}
 \item Prisoner parados:A,B,C. 2 executed, 1 pardoned.
 \\A asked :"Please tell me the name of someone else who will be executed"
 \\Guard:"B will be executed"
 \\P[A survive|B will be executed]=?
 \begin{solution}
    \begin{align*}{}{}
    P[A|\text{Guard says B}]&=\frac{P[\text{Guard says B}|A]P[A]}{P[\text{Guard says B}|A]P[A]+P[\text{Guard says B}|B]P[B]+P[\text{Guard says B}|C]P[C]}\\
    &=\frac{\frac{1}{6}}{\frac{1}{6}+0+\frac{1}{3}}=\frac{1}{3}
    \end{align*}
\begin{solution}
    \begin{align*}{}{}
    \Omega&=\{(\text{survive person,name mentioned by Guard})\}\\
    &=\begin{cases}
        (A,B),&\frac{1}{6}\\
        (A,C),&\frac{1}{6}\\
        (B,C),&\frac{1}{3}\\
        (C,B),&\frac{1}{3}\\
    \end{cases}
    P[A|B]=\frac{P(A,B)}{P(B)}=\frac{\frac{1}{6}}{\frac{1}{6}+\frac{1}{3}}=\frac{1}{3}
    \end{align*}
\end{solution}
 \end{solution}
 \item Two envelope problem:X,2X ,switch or not?
 \begin{itemize}
 \item 
 \end{itemize}
 \end{enumerate}
\end{example}
\section{Lecture 7 (02-21)-{Independence}}
\begin{definition}[Independence]{}
\begin{align*}{}{}
P[A|B]=P[A]
\end{align*} then we say A and B are independent
\\Two events A and B are independent iff $$
    P[A\cap B]=P[A]P[B]
$$ 
(A,B independent $\Rightarrow A^c,B$ independent)
\end{definition}
\begin{definition}[Multiple events Independence]{}
The events $A_1,A_2,\cdots,A_n$ are independent iff $$
    P[A_{1}\cap A_{2}\cap\cdots\cap A_{n}]=P[A_1]P[A_{2}]\cdots P[A_{n}]$$
Notice: Stronger than the condition$
    P[A\cap B]=P[A]P[B]
$ (pairwise independence)
\end{definition}
\begin{example}[pairwise independence<Independence]{}
 $ x_1,x_2,x_3 $ are coin flips:
 \begin{align*}{}{}
 P(x_i=1)&=P(x_i=0)=\frac{1}{2}\\
 A_1&=\{x_2=x_3\},A_2=\{x_1=x_3\},A_3=\{x_1=x_2\}\\
 P[A_i]&=\frac{1}{2},P[A_i\cap A_j]=P[x_1=x_2=x_3]=\frac{1}{4}=P[A_i]P[A_j]\text{ pairwise independent}\\
    P[A_1\cap A_2\cap A_3]&=P[x_1=x_2=x_3]=\frac{1}{4} \text{ not independent}
 \end{align*}
\end{example}
\begin{example}[Independence]{}
 Independent trials, each with success probability p, fail probability 1-p\\
 $P_{n,m}[\underbrace{\text{ first n success }}_{A_{n,m}}$ occured before first m failures]=?
 \begin{solution}
    (Pascal)\\
By Law of total probability: \begin{align*}
    P[A_{n,m}]&=P[A_{n,m}|\text{ 1st success }]P[1\text{st success}]+P[A_{n,m}|\text{ 1st failure }]P[1\text{st failure}]  \\
&=P[A_{n-1,m}]\cdot P+P[A_{n,m-1}]\cdot (1-P)\\
\Rightarrow P_{n,m}&=P_{n-1,m}\cdot P+P_{n,m-1}\cdot (1-P)\\
\text{Boundary condition:}P_{0,m}&=(1-P)^m,P_{n,0}=0
\end{align*}
Or building generating function
\\(Fermat):
\\\{First n success before first m failure\} 
\\$ \Leftrightarrow $ \{at least n success in the first m+n-1 trials\} (ex)
\\$ P[\text{exact k success in n+m-1 trials}]=\binom{n+m-1}{k}p^k(1-p)^{m+n-1-k} $ Binomial distribution
\\$ \Rightarrow P[\text{at least n success in the first m+n-1 trials}]=\sum_{k=n}^{m+n-1}\binom{n+m-1}{k}p^k(1-p)^{m+n-1-k}  $  
 \end{solution}
\end{example}
 \begin{example}[]{}
  Multiple choice test, m options, p-knows the answer, 1-p random guess\\
  P[knows the answer|correct]=$ \frac{p}{p+(1-p)(\frac{1}{m})}=\frac{mp}{mp+1-p} $ Bayes
 \end{example}
 \begin{example}[]{}
  Gambler's ruin:
  \\ bet 1 dollar each time, p-win, 1-p-lose, initial amount of money=$ i\in[0,N] $
  \\$ P_i[\text{Reach N before reaching 0}]= $  win times=N-i+lose times,
  Method2: $ p_i=pp_{i+1}+(1-p)p_{i-1} $
  \\characteristic polynomials: take $p_i=cr^i $
  \begin{align*}{}{}
  pr^2-r+(1-p)=0\\
  r=1,\frac{1-p}{p}\\
  \text{if }p\neq 1-p,&\text{ then }p_i=c_1+c_2(\frac{1-p}{p})^i\\
    \text{if }p=1-p,&\text{ then }p_i=c_1+c_2i, c_1,c_2\text{ determined by }p_0 ,p_N
  \end{align*} 
 \end{example}
 One dimentional random walk:
 \begin{align*}{}{}
 S_n=S_0+X_1+X_2+\cdots+X_n\\
    X_i\text{ are i.i.d. } P(X_i=1)=P,P(X_i=-1)=1-P
    \end{align*}
    \begin{example}[]{}
        \begin{enumerate}[label=\circled{\arabic*}] 
            \item secretary problem:
    \\N candidates
    \\After each interview, immediately make offer or rejection
    \\what is the best strategy
    \\maximize $ P[\text{best candidate is offered}] $ 
    \begin{solution}
        Not making offer to first r candidate, make an offer to the next candidate that is better than \{1,2,...r\}
        \\ P[{Best candidate is offered}]\\$=\sum_{i=0}^{N}\underbrace{P[\text{best candidate is i}]}_{\frac{1}{N}}P[\text{Best candidate is offerd|Best candidate=i}] $ 
    \end{solution}
         \end{enumerate}
    \end{example}
    \section{Lecture 7 (02-24)-{Random walk}}
    \begin{definition}[1-dimentional random wlk]{}
    $ S_n=S_0+x_1+x_2+\cdots+x_n \text{ }(x_i)$ i.d.d(independent+identically distributed)\\
    $ P(x_i=1)=p,P(x_i=-1)=1-p $
    \end{definition}
    Generalization: 2-D(simple) random walk
    $ S_n=S_0+x_1+x_2+\cdots+x_n \text{ }(x_i)$ i.d.d
    $$
        P(x_i=\pm e_1)=P(x_i=\pm e_2)=\frac{1}{4}
    $$ 
    P[random walk(starting at (i,j)) exit the boundary through A]=?
    \\By conditioning, $ P_{i,j}=\frac{1}{4}P_{i+1,j}\frac{1}{4}P_{i-1,j}+\frac{1}{4}P_{i,j-1}+\frac{1}{4}P_{i,j+1} $ \\
    Boundary condition: $ P_{i,j}=1 $ if $ (i,j)\in A $ , and $ P_{i,j}=0 $ otherwise
    \\We have \begin{align*}{}{}
        P_{i,j}&=\frac{1}{4}P_{i+1,j}\frac{1}{4}P_{i-1,j}+\frac{1}{4}P_{i,j-1}+\frac{1}{4}P_{i,j+1}\\
        \Leftrightarrow &\frac{1}{4}(P_{i+1,j}+P_{i-1,j}-2P_{i,j})+\frac{1}{4}(P_{i,j-1}+P_{i,j+1}-2P_{i,j})&=0\\
        \Leftrightarrow \Delta P&=0\\
        \text{Discrete Laplacian on } &Z^2,\\
        \Delta P(x)&=\sum_{y\approx x}\frac{1}{4}(P(y)-P(x))
    \end{align*} 
\subsection{Random Variable and Measurable functions}
\begin{example}[]{}
 \begin{enumerate}[label=\circled{\arabic*}] 
 \item 2 coinflips, $\Omega=\{HH,TT,HT,TH\} ,f=P(\Omega)  $ 
 X= \#heads,then we may write $ X:\Omega\rightarrow \mathbb{N},X=2\cdot 1_{HH}+1_{HT}+1_{TH} $ 
 $$
    P(X\leq x)=P(\{\omega\in \Omega:X(\omega\leq x)\})
 $$ It is called the distribution function of X (right continuous increasing function)
 \end{enumerate}
\end{example}
\begin{definition}[]{}
Let $ (\Omega_1, F_1),(\Omega_2, F_2) $ be two measurable space
\begin{itemize}
\item A map $ X:\Omega_1\rightarrow\Omega_2 $ is called measurable iff $ \forall A \in F_2,X^{-1}\in F_1 $ , $ X^{-1}=\{\omega\in \Omega_1:X(\omega)\in A\} $ 
\item if $(\Omega_1, F_1,P)$ is a probability space,then a measurable funtion $ X:\Omega_1\rightarrow\Omega_2 $ is called a random variable
\item if $(\Omega_2, F_2) =(R,B(R))$ then a measurable function is called a Borel funtion
\item if $X:(\Omega_1, F_1,P)\rightarrow (R,B(R))$ then X is a R-valued random variable and $$
    F_X(x)=P(X^{-1}(-\infty,x])=P(X\leq x)
$$ is called the distribution function of X
\end{itemize} 
\end{definition}
\textbf{Remark:}
By (a), we know that for any $ B\in B(R) $ we can define $ P(X^{-1}(B)) $  
\begin{example}[]{}
 \begin{enumerate}[label=\circled{\arabic*}] 
 \item Let $ (\Omega, F,P) $ be a probability space,$ A\in F $ \\
 Then $ 1_A:\Omega\rightarrow \{0,1\} $ is a random variable, $ 1_A(\omega)=\begin{cases}
    1, & \text{if }\omega\in A\\
    0, & \text{if }\omega\in A^c
 \end{cases} $ 
 We need to show $ \forall B\in B(R),1_A^{-1}(B)\in F $ \\
 In fact, $1_A^{-1}(B)=\begin{cases}
    \emptyset, & \text{if }0,1\notin B\\
    A, & \text{if }1\in B,0\notin B\\
    A^c, & \text{if }0\in B,1\notin B\\
    \Omega, & \text{if }0,1\in B \end{cases}$
 \end{enumerate}
\end{example}
In practice, only need to check the pre-image on a smaller set:
\begin{proposition}[]{}
    Let $ (\Omega_1, F_1),(\Omega_2, F_2) $ be two measurable space
    \\$ \mathcal{E}\subseteq F_2 $ such that $ \sigma(\mathcal{E})=F_2 $
    \\Then $ X:\Omega_1\rightarrow\Omega_2 $ is measurable if $ \forall A\in \mathcal{E}, X^{-1}(A)\in F_2 $   
\end{proposition}
\begin{proof}
    Let $ G=\{B\subseteq \Omega_2,X^{-1}(B)\in F_1\} $
    \\Then G is a $ \sigma $-algebra (ex)
    \\Therefore if $ G\supseteq \mathcal{E}\text{ then }G\supseteq \sigma(\mathcal{E})=F_2 $  
\end{proof}
\begin{corollary}
    Let($\Omega, F $) be a measurable space, then $ X:\Omega\rightarrow R $ is a Borel function iff the following are true.
    \begin{itemize}
    \item $\{x<a\}\in F \text{ for } \forall a \in R$
    \item $\{x\leq a\}\in F \text{ for }  \forall a \in R  $
    \item $\{x> a\}\in F \text{ for }  \forall a \in R  $  
    \item $\{x\geq a\}\in F \text{ for } \forall a \in R  $ 
    \end{itemize}
\end{corollary}
\begin{proof}
    $ \{x<a,a\in\mathbb{R}\}=\{X^{-1}(-\infty, a),a\in\mathbb{R}\} $ 
    \\suffices to show:$ \mathcal{E}=\{ (-\infty, a),a\in\mathbb{R}\} $ 
    \\satisfies $ \sigma(\mathcal{E})=B(\mathbb{R}) $
    \\indeed: \begin{align}
    [a,+\infty) &=(-\infty,a)^c\in \sigma(\mathcal{E})\\
[a,b)&=[a,+\infty)\backslash(b,+\infty)=(-\infty,a)^c\backslash(-\infty,b)^c\in \sigma(\mathcal{E})\\
\Rightarrow \sigma(\mathcal{E})&=B(R)
    \end{align}
\end{proof}
\begin{remark}[]{}
a R-valued random variable is a function $ X:\Omega\rightarrow\mathbb{R} $ such that $ \forall a\in\mathbb{R},\{X< a\}\in F $
\end{remark}
\begin{lemma}[]{}
A distribution funtion F satisfies:
 \begin{enumerate}[label=\circled{\arabic*}] 
 \item $ \lim_{x\rightarrow-\infty}F(x)=0,\lim_{x\rightarrow+\infty}F(x)=1 $ 
 \item $ F(x)\leq F(y) \text{ if }x\leq y $ 
 \item F is right continuous,$ F(x+h)\rightarrow F(x) $ as h decreases to 0
 \end{enumerate}
\end{lemma}
\begin{proof}
   $ F(x)=P(X\leq x)$ (2) is immediate\\
   (1): Let $ A_n=\{x\leq -n\} $ then by continuous of measure:$ lim_{n\rightarrow+\infty}=\lim_{n\rightarrow+\infty}=P(\bigcap_{n\geq1}A_n)=0 $  
   \\Then by monotonicity of F, $ lim_{x\rightarrow-\infty}F(x)=0 $
   \\(3): Let $ B_n=\{X\leq x+\frac{1}{n}\}(\text{decreaing}), $\\ then $ F(x+\frac{1}{n})=P(B_n) \underbrace{\rightarrow}_{n\rightarrow+\infty,cont.}P(\bigcap_{n\geq1}B_n)=P(X\leq  x)=F(x) $ 
   \\again use F increasing to conclude 
\end{proof}
\begin{lemma}[]{}
 \begin{enumerate}[label=\circled{\arabic*}] 
 \item $F(y)-F(x)=P(x<X\leq y),\forall x\leq y$
 \item $ F(x)-\lim_{h\rightarrow+\infty}F(x-h)=P(X=x) $
 \item $ B_n=\{x-\frac{1}{n}<X\leq x\}(\text{decreaing}) $,Then $ \bigcap_{n\geq1}B_n=\{X=x\} $
 \\$\Rightarrow F(x)-F(x-\frac{1}{n})=P(B_n)\underbrace{\rightarrow}_{n\rightarrow+\infty}P(X=x)$ \\again use F increasing to conclude
    \end{enumerate}
\end{lemma}
Naturally: 
\begin{itemize}
\item (absolute) continuous random variable
\item discrete random variable
\end{itemize}
\section{Lecture 8 (02-26)}
properties of measurable functions(R.V.s):
\begin{itemize}
\item Let $\{X<Y\}=\{\omega\in \Omega_1:X(\omega)<Y(\omega)\}$\\
$\{X>Y\}=\{\omega\in \Omega_1:X(\omega)>Y(\omega)\}$
\end{itemize}
\begin{lemma}[]{}
    Let $ (\Omega ,F) $ be a measurable space, $ X,Y $ are Borel Functions
\begin{itemize}
\item $\{X<Y\},\{X\leq Y\},\{X=Y\},\{X\neq Y\}\in F$
\item $X+Y,X\cdot Y,X/Y \text{ are Borel functions}$
\end{itemize}
\end{lemma}
\begin{proof}
(1):Use that Q is dense in R\begin{align*}{}{}
    \{X<Y\}=\bigcup_{q\in Q}\{X<q<Y\}= \bigcup_{q\in Q}\underbrace{\{X<q\}}_{\in F}\cap\underbrace{\{q<Y\}}_{\in F}\in F\\
    \{X=Y\}=\bigcap_{n\geq 1}\{X<Y+\frac{1}{n}\}\cap\bigcap_{n\geq 1}\{X>Y-\frac{1}{n}\}\in F
\end{align*}
(2):Fact: if Y is Borel, then aY+b,$ a,b\in R $ is Borel (ex)
\\Then $$
    \forall a\in \mathbb{R}, \{X+Y<a\}=\{X<a-Y\}\in F \text{ by(1)}
$$  so $ X+Y $ is Borel
$$
    \{X^2<a\}=\begin{cases}
        \emptyset, \\
        \{X<\sqrt{a}\}\cap\{X>-\sqrt{a}\} \in F
    \end{cases}
$$ so $ X^2 $ is Borel
\\$ X\cdot Y=\frac{1}{4}[(X+Y)^2-(X-Y)^2] $ is Borel
\end{proof}
\begin{lemma}[]{}
Let $ (X_n)_{n\geq1} $ be a sequence of Borel functions on $ (\Omega,F) $, \\Then the following are Borel functions:
\begin{itemize}
\item $ \sup_{n\geq1}X_n,\inf_{n\geq1}X_n,\limsup_{n\geq1}X_n,\liminf_{n\geq1}X_n $
\end{itemize} 
In particular, if $ \lim_{n\rightarrow+\infty}X_n $ exists, then $ \lim_{n\rightarrow+\infty}X_n $ is Borel 
\end{lemma}
\begin{proof}
    \begin{align*}{}{}
       \{ \sup_{n\geq1}X_n<a\}=\bigcap_{n\geq1}\{X_n<a\}\in F \forall a \in \mathbb{R} \Rightarrow \sup_{n\geq1}X_n \text{ is a Borel function}\\
       \{ \inf_{n\geq1}X_n>a\}=\bigcap_{n\geq1}\{X_><a\}\in F \forall a \in \mathbb{R} \Rightarrow \inf_{n\geq1}X_n \text{ is a Borel function}\\
       \limsup X_n=\inf_{m\geq1}\sup_{n\geq m}X_n \text{ is a Borel function}\\
         \liminf X_n=\sup_{m\geq1}\inf_{n\geq m}X_n \text{ is a Borel function}
    \end{align*}
\end{proof}
\begin{lemma}[]{}
    Let $ (\Omega_1, F_1),(\Omega_2, F_2),(\Omega_3, F_3) $ be two measurable space
\\$ X:\Omega_1\rightarrow\Omega_2,Y:\Omega_2\rightarrow\Omega_3 $ are measurable, then $ Y\circ X:\Omega_1\rightarrow\Omega_3 $ is measurable
\end{lemma}
\begin{proof}
    \begin{align*}{}{}
        \forall A\in F_3,(Y\circ X)^{-1}(A)&=X^{-1}(\underbrace{Y^{-1}(A)}_{\in F_2})\in F_1
    \end{align*}
\end{proof}
\begin{definition}[$ \sigma $-algebra generated by r.v.]{}
If $ X:\Omega\rightarrow\mathbb{R} $ is a random variable, then $$
    \sigma(x)=\{X^{-1}(A),A\in B(\mathbb{R})\}
$$ is called the $ \sigma $-algebra generated by X
\\Let $ (X_i)_{i\in I} $ be a family of r.v.s$$
    \sigma(X_i,i\in I)=\sigma(\bigcup_{i\in I}\sigma(X_i))$$ is the $ \sigma $-algebra generated by $ (X_i)_{i\in I }$
\end{definition}
\begin{remark}[]{}
$ \sigma(X) $ is the smallest $ \sigma $-algebra such that X is measurable
\end{remark}
\begin{example}[]{}
 \begin{enumerate}[label=\circled{\arabic*}] 
 \item $(\Omega,F,P)$ ,Let $ X:\Omega\rightarrow\mathbb{R} $ be a random variable, $ X=b_1 1_{A_1}+ b_2 1_{A_2}+\cdots b_n 1_{A_n},b_i\in\mathbb{R},A_i\in F ,A_i\cap A_j=\emptyset $
 \\if $ b_j\neq b_j $\\ Then $ \sigma(X)=\sigma\{A_1,A_2,\cdots,A_n\} =$"finite disjioint union of $ A_1\cdots A_n $"
 \begin{proof}
    \begin{itemize}
    \item $ \supseteq $: Note that $ X^{-1}(\{b_1\})=A_1, \cdots, X^{-1}(\{b_n\})=A_n$ 
    \\$ \Rightarrow A_1,\cdots, A_n\in\sigma(x),\sigma(A_1,\cdots, A_n)\in\sigma(x) $ 
    \item $ \subseteq $  \begin{lemma}[]{}
        $ \sigma(X)=\sigma(\{X\leq a\},a\in \mathbb{R}) $ 
        \end{lemma}
    it suffices to show $ \forall a\in \mathbb{R},\{X\leq a\}\in \sigma(\{A_1,\cdots,A_n\}) $
    \\$ \{X\leq a\}= $"finite disjioint union of $ A_i \text{ and }(\bigcup A_i)^c$" 
    \end{itemize}
 \end{proof}
 \end{enumerate}
\end{example}
Two specific cases: Discrete and (Absolutely)continuous random variables
\begin{definition}[]{}
A r.v. is discrete if it takes values in a countable set $ \{X_1,X_2,\cdots\} $
\\prob mass function:$ f(x)=P(X=x) $
\end{definition}
\begin{remark}[]{}
We say $ x_1,x_2,\cdots, $ are atoms of $ F_x $  
\end{remark}
\begin{definition}[]{}
A r.v. is (absolutely) continuous if $$
    F_x(x)=P(X\leq x)=\int_{-\infty}^{x}f(u)du
$$ for some integrable function $ f:R\rightarrow [0,+\infty),f(x)=F'_x(x) $ is called the probability density function of X
\end{definition}
\begin{remark}[]{}
$ F_x $ is absolutely continuous 
\\F is absolutely continuous iff $$ \forall \epsilon>0,\exists \delta>0 $$ s.t. for any finite collecion of intervals$ a_i,b_i $ s.t. $$ \sum_{i=1}^{n}|b_i-a_i|<\delta \Rightarrow \sum_{i=1}^{n}|F(b_i)-F(a_i)|<\epsilon $$
$$
    C^1\Rightarrow \text{absolutely continuous}\Rightarrow \text{uniformly continuous}
$$ 
\end{remark}
\begin{remark}[]{}
X is a singular Continuous R.V. if $ F_x $ is continuous but $ F_x $ is not absolutely continuous (ex: Cantor funtion)
\end{remark}
Discrete R.V.:
\begin{align*}{}{}
F_x(x)=P(X\leq x)=\sum_{x_i\leq x}\underbrace{f(x_i)}_{\text{prob mass function}}\\
f(x)=F_x(x)-\lim{y\nearrow  x-}F_x(y)
\end{align*}
\begin{definition}[]{}
The expectation/mean of a discrete r.v. with prob mass function f is  $$
    E[X]=\sum_{x:f(x)>0}x_if(x_i)$$ whenever the sum is absolutely convergent
\end{definition}
\begin{remark}[]{}
    absolutely convergent $ \Rightarrow $ order of the sum does not matter
\end{remark}
\begin{example}[]{}
 \begin{enumerate}[label=\circled{\arabic*}] 
 \item 2 coin flips,X=\#heads, $ f(0)=f(2)=\frac{1}{4},f(1)=\frac{1}{2} $
 \\$ E[X]=0\cdot\frac{1}{4}+1\cdot\frac{1}{2}+2\cdot\frac{1}{4}=1 $
 \end{enumerate}
\end{example}
\begin{lemma}[]{}
Let X be a r.v. taking values in N.
\\Then $$
    E[X]=\sum_{n=1}^{+\infty}P(X\geq n)
$$ 
\end{lemma}
\begin{proof}
    \begin{align*}{}{}
    E[x]=1\cdot P(X=1)+2\cdot P(X=2)+&\cdots\\
    =P(X=1)+P(X=2)+P(X=3)+&\cdots\\
    P(X=2)+P(X=3)+&\cdots\\
    P(X=3)+&\cdots
    \end{align*}
    $ P(X\geq n) $ 
\end{proof}
\section{Lecture 9 (02-28)-{Expectation}}
\subsection*{EX1}
we have offers $ X_1,X_2,\cdots $ continuous R.V. i.i.d
\\$ T=inf\{n>1,X_n>X_1\} $  Compute $ E[T] $
\begin{solution}
    $$
    E[T]=\sum_{n=2}^{+\infty}nP(T= n)
    $$
    By lemma,\begin{align*}{}{}
        E[T]&=\sum_{n\geq1}P(T\geq n)=\sum_{n\geq1}P(X_1\geq X_2\geq\cdots\geq X_n)=\frac{1}{n-1},n\geq2\\
        &=+\infty
    \end{align*}
\end{solution}
\subsection*{EX2}
Independent trials, each with success probability p, fail probability 1-p
\\Compute P[There is n consecutive successes before m consecutive failures]
Condition on the first trial: $$
    P[A]=p\cdot P[A|H]+(1-p)\cdot P[A|T]
$$ 
\\Multiple conditioning: $ P(E_1\cap E_2\cdots E_n)=P(E_1)P(E_2|E_1)\cdots P(E_n|E_1\cap\cdots E_n) $ \\
Let $ B=\{HH\cdots H\} $ (2nd--n-th) $$ P(A|H)=P(B)P(A|H\cap B)+P(B^c)P(A|H\cap B^c)=p^{n-1}\cdot 1+(1-p^{n-1})\cdot P(A|T) $$ 
Let $ C=\{TT\cdots T\} $ (2nd--m-th) $$ P(A|T)=P(C)P(A|T\cap C)+P(C^c)P(A|T\cap C^c)$$
Let q=1-p,\begin{align*}{}{}
P(A|H)=\frac{p^{n-1}}{p^{n-1}+q^{m-1}-p^{n-1}q^{m-1}}\\
P(A|T)=\frac{(1-q^{m-1})p^{n-1}}{p^{n-1}+q^{m-1}-p^{n-1}q^{m-1}}\\
P(A)=\frac{p^{n-1}(1-q^{m-1})}{p^{n-1}+q^{m-1}-p^{n-1}q^{m-1}}
\end{align*}
\section{Lecture 10 (03-02)-{Var}}
\begin{lemma}[change of variable]{}
Let $ g:R\rightarrow R$ X be a r.v. with probability mass function f ,then $$
    E[g(x)]=\sum_{x:f(x)>0}g(x)f(x)
$$ 
\end{lemma}
\begin{proof}
    \begin{align*}{}{}
    E[g(x)]&=\sum_{y}y\cdot P(g(x)=y)=\sum_{y}\sum_{x:g(x)=y}y\cdot P(X=x)\\
       &=\sum_{x}g(x)P(X=x)=\sum_{x}g(x)f(x)
    \end{align*}
\end{proof}
\begin{definition}[]{}
Let $ k\in N $ ,the k-th moment of X is $ mk:= E[X^k] $ as long as the expectation exists\\
Let $ k\in N $ ,the k-th central moment of X is $ \sigma k:= E[X-E[X]]^k $ as long as the expectation exists\\
\begin{itemize}
\item $ \sigma_2=Var (X)=E(X-E[X])^2 $ varience,"deviation fluctuation" from the mean
\item $ \sigma=\sqrt{Var(X)} $ standard deviation
\end{itemize}
\end{definition}
\textbf{Fact:}
\begin{itemize}
\item $ E[aX+bY]=aE[x]+bE[Y] a,b \in R$ \\
$\rightarrow Var(X)=E(X-E[x])^2=E[X^2]-(E[X])^2$
\end{itemize}
\begin{example}[]{}
 \begin{enumerate}[label=\circled{\arabic*}] 
 \item Bernoulli(p) $P(X=1)=p,P(X=0)=1-p $\\
 $ E[X]=1 \cdot P(X=1)=p, Var(X)=p-p^2$
 \item binomial(n,p) $ P(X=k)=\binom{n}{k}p^k(1-p)^{n-k} $\\
 $$ E[X]=\sum_{k=0}^{n}kP(x=k)=\sum_{k}k\cdot \binom{n}{k}p^k(1-p)^{n-k}$$
 Recall $$ (1+x)^n= \sum_{k=0}^n x^k\cdot \binom{n}{k}$$,differentiate both sides w.r.t x, then $$ n(1+x)^{n-1}=\sum_{k=1}^n k\cdot x^{k-1}\cdot \binom{n}{k} $$
 Let $q=1-p,x=\frac{p}{q}$, then \begin{align*}{}{}
     E[X]&=\sum_{k=1}^{n}k\cdot \binom{n}{k}(\frac{p}{q})^{k-1}\cdot q^n\cdot \frac{p}{q}\\
     &= n\cdot \frac{p}{1-p}\cdot (1+\frac{p}{1-p})^{n-1}=np\\
     Var(X)&=np(1-p)
    \end{align*}
    sol n': we have $ X=Y_1+Y_2+\cdots+Y_n $  such that $ Y_i\sim Bernoulli(p) $ and $ Y_i $ independent \begin{align*}{}{}
    E[X]&=\sum_{i=1}^{n}E[Y_i]=np\\
    Var(X)&=E[X^2]-(E[X])^2=E(\sum_{i=1}^{n}Y_i)^2-(np)^2\\
    &=\sum_{i=1}^{n}E[Y_i]^2-2\sum_{i<j}^{n}E[Y_i]E[Y_j]-(np)^2\\
    &=np+2p^2\frac{n(n-1)}{2}-(np)^2\\
    &=np(1-p)
    \end{align*}
    If x,y are independent, then $ E[XY]=E[X]E[Y] $
    \\Proof: $$ E[XY]=\sum_{x,y}xyP(X=x,Y=y)=\sum_{x,y}xyP(X=x)P(Y=y)$$ $$=\sum_{x}xP(X=x)\sum_{y}yP(Y=y)=E[X]E[Y] $$
 \end{enumerate}
\end{example}
\begin{definition}[]{}
If E[XY]=E[X]E[Y], then we say X,Y are uncorrelated\\
Covariance $$
    Cov(X,Y)=E[XY]-E[X]E[Y]=E[(X-E[X])(Y-E[Y])]
$$ 
Correlation $$
    \rho(X,Y)=\frac{Cov(X,Y)}{\sqrt{Var(X)Var(Y)}}
$$
\end{definition}
\begin{lemma}[]{}
$|\rho(X,Y)|\leq 1,\rho(X,Y)=\pm 1\text{ iff }Y=aX+b \text{ for some }a,b \in \mathbb{R}$
\end{lemma}
\begin{proof}
    \begin{align*}{}{}
        \rho(X,Y)=\frac{Cov(X,Y)}{\sqrt{Var(X)Var(Y)}}=\frac{E[(X-E[X])(Y-E[Y])]}{\sqrt{Var(X)Var(Y)}}\\
        \text{by Cauchy-Schwarz inequality: } |\rho(X,Y)|\leq 1\\
    \end{align*}
\end{proof}
Cauchy-Schwarz inequality: $$
    |E[XY]|\leq \sqrt{E[X^2]E[Y^2]}
$$
\begin{proof}
    \begin{align*}{}{}
    \text{Since }0&\leq E[(aX-bY)^2]\forall a,b \in R \Rightarrow a^2E[X^2]+2abE[XY]+b^2E[Y^2]\geq0\\
    \Rightarrow \frac{1}{4}\Delta&=E[XY]^2-E[X^2]E[Y^2]\leq 0
    \end{align*}
\end{proof}
\begin{example}[]{}
 \begin{enumerate}[label=\circled{\arabic*}] 
 \item $ X\sim Geometric(p),P(X=k)=(1-p)^{k-1}\cdot p $\\
 $ E[X]=\frac{1}{p},Var(x)=\frac{1-p}{p^2} $ 
 \end{enumerate}
\end{example}
\section{Lecture 11 (03-05)-{poisson random variable}}
Poisson random variable:
\\Observe the number of customers in the past days, $ X_i $ is the number of customers on day i
\\How to predict the number of customers tomorrow?
\begin{itemize}
\item One may take $ \bar{X_n}=\frac{1}{n}\sum_{i=1}^{n}X_i $ 
\item n intervals during each interval, at most 1 customers
\\$\Rightarrow$ number of customers in n intervals is $ Bern(p) $
\\Take P, s.t. $np=E[X]=\lambda$
\\number of customer is Binomial($ n,\frac{\lambda}{n} $)\begin{align*}{}{}
P(X=k)&=\binom{n}{k}\binom{\lambda}{n}^k(1-\frac{\lambda}{n})^{n-k}\\
&=\frac{n!}{(n-k)!n^k}\frac{\lambda^k}{k!}(1-\frac{\lambda}{n})^{n-k}\\
&\Rightarrow(n\rightarrow+\infty,k \text{ fixed})\frac{\lambda^k}{k!}e^{-\lambda}
\end{align*}
\end{itemize}
\begin{definition}[]{}
X is a Poisson(x) is given by probability mass function $$
    f(k)=\frac{\lambda^k}{k!}e^{-\lambda},k\in N
$$ 
\end{definition}
if $ X\sim Poisson(\lambda) $, then $$
    E[X]=\sum_{k\in N^+}k\cdot \frac{\lambda^k}{k!}e^{-\lambda}=\lambda \sum_{k=0}^{+\infty}\frac{\lambda^k}{k!}e^{-\lambda}=\lambda e^{\lambda-\lambda}=\lambda
$$ 
Continuous r.v.$$
    F_x(x)=P(X\leq x)=\int_{-\infty}^{x}f(u)du, f(x)=F'_x(x)
$$ is the probability density function of X\begin{align*}{}{}
P(x\leq X\leq X+dx)=\int_{x}^{x+dx}f(u)du\approx f(x)dx\\
\end{align*}
Expectation: the expectation of a r.v. X is defined by$$
    E[X]=\int_{-\infty}^{+\infty}xf(x)dx=\int_{-\infty}^{+\infty}xdF_x(x)
$$ 
k-th moment:\begin{align*}{}{}
E[x^k]=\sum_{-\infty}^{+\infty}x^kf(x)dx\\
Var[x]=E[(X-E[X])^2]=\int_{-\infty}^{+\infty}(x-E[X])^2f(x)dx
\end{align*}
Recall if X is a N-valued r.v. then $$
    E[x]=\sum_{n\in N}P(x\geq n)
$$ 
\begin{lemma}[]{}
if X is a non-negative r,v, with density function f then\begin{align*}{}{}
E[X]=\int_{0}^{+\infty}P(X> x)dx=\int_{0}^{+\infty}(1-F_x(x))dx
\end{align*}
\end{lemma}
\begin{proof}
    \begin{align*}{}{}
    \int_{0}^{+\infty}P(X>x)dx&=\int_{0}^{+\infty}\int_{x}^{+\infty}f(y)dy dx\\
    &=\int_{0}^{+\infty}\int_{0}^{y}dxf(y)dy=\int_{0}^{+\infty}yf(y)fy=E[X]\\
    \end{align*}
\end{proof}
How to define $ \int xdF_x(x) $ in general?
\begin{definition}[Lebesgue integral and expectation]{}
recall Riemann integral $$
    \sum f(x_i^\star)\Delta x_i\rightarrow \int f(x)dx
$$ make sense if f has finitely many discontinuities (for $ f=1_Q $ R-I does not exist)
\\Lebesgue integral:\begin{align*}{}{}
\{x\in\mathbb{R}:f(x)\in [y_i,y_i+\Delta y_i)\}=f^{-1}([y_i,y_i+\Delta y_i))\\
\text{idea: }\sum m(f^{-1}([y_i,y_i+\Delta y_i)))\cdot \Delta y_i\rightarrow \int f(x)dx\\
\int 1_Q(x)dx=0
\end{align*}
\end{definition} 
More generally, given any measure space $ (\Omega, \mathcal{F},\mu) $ and Borel function f 
\\Define $ \int_\Omega f du $ 
\\Step1: If $ f=1_A $ ,where $ A\in F $ define $ \int_\Omega f d\mu=\int_\Omega 1_A d\mu=\mu(A) $
\\Step2: simple functions: \begin{align*}{}{}
f=\sum_{i=1}^{n}a_i1_{A_i},A_i\in F \text{ and }a_i\geq0,A_i\cap A_j=\emptyset\\
\text{Define }\int_\Omega f d\mu=\sum_{i=1}^{n}a_i\mu(A_i)
\end{align*}
Fact: if f,g are simple function then f+g,fg,max\{f,g\},min\{f,g\} are simple functions
\begin{proposition}[]{}
if f,g are simple function, then\begin{itemize}
\item $ \int_\Omega af d\mu=a  \int_\Omega f d\mu \forall a\in\mathbb{R}$
\item $ \int_\Omega (f+g)d\mu=\int_\Omega f d\mu+\int_\Omega g d\mu$
\item if $ f\leq g  $ then $ \int_\Omega f d\mu\leq \int_\Omega g d\mu $
\end{itemize}
Step3: approximates non-negative Borel functions by simple functions
\\Let $ f\geq0 $ Borel, then $ f=\sup_if_i  $ ,where$$
    f_i=\sum_{k=0}^{i\cdot 2^i}\frac{k-1}{2^i}1_{\{\frac{k-1}{2^i}\leq f<\frac{k}{2^i}\}}+i1_{\{f>i\}}
$$ $ f_i\geq0 $ are simple and $ \lim_{i\rightarrow\infty}f_i=\sup_if_i =f $ 
\end{proposition}
\begin{definition}[]{}
For every non-negative Borel function f, define $$
    \int_\Omega f d\mu=\sup_i\int_\Omega f_i d\mu
$$ Q:if $ f=\sup f_i,f=\sup g_i $ Does $ \sup_i \int f_id\mu=\sup_i \int g_id\mu $? 
\end{definition}
Consistency follows from:\\
Monotone Convergence theorem: For every increasing sequence $ \{f_n\} $ of measurable functions:$$
    \limsup_n \int_\Omega f_n d\mu=\int_\Omega \limsup_n f_n d\mu
$$  (If ($ X_n $)) is a sequence of r.v.s, $ X_n\nearrow x $  then $ \lim_{n\rightarrow+\infty} E[X_n]=E[\lim_n X_n]=E[X] $
\\Assume MCT:\begin{align*}{}{}
\text{If }f=\sup f_i=\sup g_i\\
\text{Then } g_i\leq \sup_i f_i
\end{align*}
\section{Recitation (03-07)}
\subsection*{Problem 1}
Let $ X\sim Poisson(\lambda_1),Y\sim Poisson(\lambda_2), independent $ 
\\Compute the probability mass function of $ X+Y $
\begin{align*}{}{}
P(X+Y=n)&=\sum_{k=0}^{n}P(X=k)P(Y=n-k)\\
&=\sum_{k=0}^{n}\frac{\lambda_1^k}{k!}e^{-\lambda_1}\cdot \frac{\lambda_2^{n-k}}{(n-k)!}e^{-\lambda_2}\\
&=\frac{e^{-(\lambda_1+\lambda_2)}}{n!}\sum_{k=0}^{n}\binom{n}{k}\lambda_1^k\lambda_2^{n-k}\\
&=\frac{e^{-(\lambda_1+\lambda_2)}}{n!}(\lambda_1+\lambda_2)^n
\end{align*}
\textbf{$ \sum_kf_x(k)f_Y(n-k)=f_X*f_Y $ }
\subsection*{Problem 2}
m balls, n boxes, uniform random
\\Compute the E[number of empty boxes]
\\number of empty boxes:$ 1_{b_1empty}+\cdots1_{b_nempty} $ 
\\linearity of expectation: \begin{align*}{}{}
E[\#]&=\sum_{i=1}^{n}E 1_{b_i empty}\\
&=nP(\text{box 1 is empty})\\
&=n(\frac{n-1}{n})^m
\end{align*} 
\\Note: \begin{align*}{}{}
E[X]=\int XdP\\
E[X]=\int 1_A dP=P(A)
\end{align*}
\subsection*{Problem 3}
Coupon collector: n types of coupons, pick one at random
\\$ T_n $ =time to complete the collection of n coupons
\\Compute $ E[T_n] $
Let $ T_i $ =time to collect the i-th new coupon
\\$ T_n=T_1+T_2+\cdots+T_n $
\\$ E[T_1]=1, T_j-T_{j-1}\sim Geo(1-\frac{i-1}{n})$ every trial with success prob:$ 1-\frac{i-1}{n} $
\begin{align*}{}{}
E[T_j-T_{j-1}]=\frac{1}{1-\frac{i-1}{n}}\\
E[T_n]=\sum_{j=1}^{n}E[T_j-T_{j-1}]=n\sum_{j=1}^{n}\frac{1}{j}
\end{align*} 
\subsection*{Problem 4}
Let X be a r.v., $ E[X]=1 $
\\show that $$ t\in(0,1), P(X>t)>\frac{(1-t)^2}{E[X^2]}$$
(hint:Cauchy Schwarz)
\begin{align*}{}{}
\text{Let }Y&=1_{X>t}\\
E[\mathbf{1}_{X>t}]=P(X>t)=E[Y]=E[Y^2]\\
\text{By Cauchy-Schwarz: }E[XY]&\leq \sqrt{E[X^2]E[Y^2]}\\
E[X^2]E[Y^2]&\geq E[XY]^2=E[X(1-1_{x\leq t})]^2\\
&=(1-E[X1_{X\leq t}])^2\\
\geq (1-t)^2
\end{align*}
\begin{definition}[Paykey-Zygmund inequality]{}
    for $ t\in (0,1) $ 
$$P(Y>tE[Y])\geq (1-t)^2\frac{(E[Y])^2}{E[Y^2]}$$
second moment method
\end{definition}
\section{Lecture 12 (03-10)}
Midterm March 26
\begin{definition}[Monetone Convergence Theorem]{}
 For any increasing sequence of function:$ \{f_n\} $ such that $ \{f_n\} $ is bounded from below 
\\Then $$
    \lim_{n\rightarrow+\infty}\int f_n d\mu=\int \lim_{n\rightarrow+\infty}f_n d\mu
$$ 
\end{definition}:
\textbf{Proof:}
\\Since $ \int f_nd\mu\leq \int  f\mu \forall n\in N $ Then $ \limsup \int f_n d\mu\leq fd\mu $  
\\\textbf{Fact:}
\\if $ \phi $ is a simple function, then $ u(A):=\int_{A}d\mu $ defines a measure(ex.)
\\Take a sequence $ \{\phi_k\} $ of the simple function: $ \phi_k\nearrow f $    
\\Let $ \alpha\in(0,1) $ Fix a given $ \phi_k $
\\Let $ A_n=\{f_n>\alpha\phi_k\}=\{\omega\in\Omega:f_n(\omega)>\alpha\phi_k(\omega)\} $
\\Then $ (A_n)\nearrow  $, and \begin{align*}{}{}
\lim_{n\rightarrow+\infty}&=\bigcup_{n\geq1}A_n=\{\exists n\in N s.t.f_n>\alpha\phi_k\}\\
&=\{sup_nf_n>\alpha \phi k\}=\Omega\\
\text{Since } \int_{\Omega}f_nd\mu& \geq \int_{A_n}f_nd\mu\geq \alpha\int_{A_n}\phi_kd\mu\\
\text{send } n&\rightarrow+\infty \liminf \int_{\Omega} f_nd\mu\geq \alpha \lim_{n\rightarrow +\infty}\int_{A_n}\phi_kd\mu=\alpha\int_{\Omega}\phi_kd\mu\\
\text{Send } \phi_k &\rightarrow f \text{ and } \alpha\rightarrow 1 \text{ to complete the pf }
\end{align*}
\begin{proposition}[]{}
  Let  f, g be Lebesgue measurable. Then \begin{itemize}
 \item $$
    \int_{\Omega}af+bgd\mu=a\int_{\Omega}fd\mu+b\int_{\Omega}gd\mu
 $$ 
 \item if $ f\leq g $ then $$
    \int_{\Omega}fd\mu\leq \int_{\Omega}gd\mu
 $$  
 \end{itemize}
\end{proposition}
\begin{proposition}[]{}
    f is (Lebesgue) integrable iff |f| is integrable.
    \\If there exists a function Y, s.t. $|f| \leq Y$, and Y is integrable, then f is integrable.
\end{proposition}
\textbf{Proof:}
\begin{align*}
    & \text{ Note that } f = f^+ - f^-, \text{ } |f| = f^+ + f^-\\
    &\text{Therefore } \int_\Omega |f|d\mu < \infty \Leftrightarrow \int_\Omega f^+d\mu < \infty \text{ and } \int_\Omega f^-d\mu < \infty \Leftrightarrow \int_\Omega fd\mu < \infty.\\
    \\
    &\text{If } |f| \leq Y, \text{ then } \int_\Omega |f|d\mu \leq \int_\Omega  Y d\mu < \infty \Rightarrow \text{ |f| integrable } \Rightarrow f \text{ integrable.}\\
    \\
    &\textbf{eg.} \text{ } (R, B(R), m), \text{ } f(x) = \sin x. \text{ Is } f \text{ Lebesgue integrable?}\\
    &|f(x)| \leq 1 \Rightarrow \int_{R} |f|dx \geq \sum_k \int_{k\pi+\frac{\pi}{6}}^{(k+1)\pi-\frac{\pi}{6}} |sinx| dx =+\infty\\
    &\Rightarrow f \text{ is NOT integrable}
    \end{align*}
\begin{definition}[]{}\begin{align*}{}{}
  \text{We say } f& = g \text{ almost everywhere (a.e.) if } \{f \neq g\} \text{ has measure } 0.\\
 \text{r.v.s. } X& = Y \text{ almost surely (a.s.) if } \mathbb{P}(\{X \neq Y\}) = 0
\end{align*}
  \end{definition}
  \begin{theorem}
    If $\mu(A) = 0$, then $\int_A f \, d\mu = 0$ for any measurable function $f$.
    \end{theorem}
    
    \begin{proof}
    We prove this in three steps:
    
    \textbf{Step 1: Simple functions.}
    
    For a simple function $s = \sum_{i=1}^n a_i \chi_{E_i}$, where $\chi_{E_i}$ is the characteristic function of set $E_i$:
    $$\int_A s \, d\mu = \sum_{i=1}^n a_i \mu(A \cap E_i)$$
    
    Since $\mu(A) = 0$, we have $\mu(A \cap E_i) \leq \mu(A) = 0$ for any measurable set $E_i$. Therefore, $\mu(A \cap E_i) = 0$.
    
    Hence, $\int_A s \, d\mu = \sum_{i=1}^n a_i \cdot 0 = 0$
    
    \textbf{Step 2: Non-negative measurable functions.}
    
    For any non-negative measurable function $f \geq 0$, there exists an increasing sequence of simple functions $\{s_n\}$ such that $s_n \uparrow f$ pointwise.
    
    By the Monotone Convergence Theorem:
    $$\int_A f \, d\mu = \lim_{n \to \infty} \int_A s_n \, d\mu$$
    
    From Step 1, for each $n$, $\int_A s_n \, d\mu = 0$. Therefore, $\int_A f \, d\mu = \lim_{n \to \infty} 0 = 0$
    
    \textbf{Step 3: General measurable functions.}
    
    For any measurable function $f$, we can decompose it as $f = f^+ - f^-$, where $f^+ = \max(f, 0)$ and $f^- = \max(-f, 0)$ are both non-negative measurable functions.
    
    By the linearity of the integral:
    $$\int_A f \, d\mu = \int_A f^+ \, d\mu - \int_A f^- \, d\mu$$
    
    From Step 2, $\int_A f^+ \, d\mu = 0$ and $\int_A f^- \, d\mu = 0$. Therefore, $\int_A f \, d\mu = 0 - 0 = 0$
    
    Thus, if $\mu(A) = 0$, then $\int_A f \, d\mu = 0$ for any measurable function $f$.
    \end{proof}
    
    \begin{corollary}
    If $f = g$ almost everywhere (a.e.), then $\int_\Omega f \, d\mu = \int_\Omega g \, d\mu$.
    \end{corollary}
    
    \begin{proof}
    Let $E = \{x \in \Omega : f(x) \neq g(x)\}$. Since $f = g$ a.e., we have $\mu(E) = 0$.
    
    Consider $h = f - g$. Then $h = 0$ on $\Omega \setminus E$, and $h \neq 0$ only on $E$.
    
    Therefore:
    $$\int_\Omega (f - g) \, d\mu = \int_\Omega h \, d\mu = \int_E h \, d\mu + \int_{\Omega \setminus E} h \, d\mu = \int_E h \, d\mu + 0$$
    
    Since $\mu(E) = 0$, by our theorem, $\int_E h \, d\mu = 0$. Thus, $\int_\Omega (f - g) \, d\mu = 0$.
    
    By the linearity of the integral:
    $$\int_\Omega f \, d\mu - \int_\Omega g \, d\mu = \int_\Omega (f - g) \, d\mu = 0$$
    
    Therefore, $\int_\Omega f \, d\mu = \int_\Omega g \, d\mu$.
    \end{proof}
    \begin{proposition}[]{}
         Let $(\Omega, \mathcal{F}, \mu)$ be a measure space. Let $f: \Omega \to [0,+\infty)$ be a Borel function.\\
        
        Then $\nu(A) = \int_A f\,d\mu$, $\forall A \in \mathcal{F}$, defines a measure.
        \end{proposition}
        
        \begin{definition}[]{}
          We say $f$ is the Radon-Nikodym derivative (or density) of $\nu$ with respect to $\mu$. Write $f = \frac{d\nu}{d\mu}$.
        \end{definition}
        
        \begin{proof}[Proof of Proposition]
        \begin{itemize}
            \item $\nu(\emptyset) = 0$ is obvious.
            \item Countable additivity: Let $(A_i)_{i=1}^{+\infty}$ be disjoint. Let $A = \cup_{i=1}^{+\infty} A_i$.

Then:
\begin{align*}
\nu(A) &= \int_A f\,d\mu \\
&= \int_{\Omega} f \cdot \mathbf{1}_A \,d\mu \\
&= \int_{\Omega} f \cdot \left(\lim_{n\to\infty} \sum_{i=1}^{n} \mathbf{1}_{A_i}\right) \,d\mu \\
&\stackrel{\text{MCT}}{=} \lim_{n\to\infty} \int_{\Omega} f \cdot \left(\sum_{i=1}^{n} \mathbf{1}_{A_i}\right) \,d\mu \\
&= \lim_{n\to\infty} \sum_{i=1}^{n} \int_{\Omega} f \cdot \mathbf{1}_{A_i} \,d\mu \\
&= \lim_{n\to\infty} \sum_{i=1}^{n} \int_{A_i} f \,d\mu \\
&= \lim_{n\to\infty} \sum_{i=1}^{n} \nu(A_i) \\
&= \sum_{i=1}^{+\infty} \nu(A_i)
\end{align*}
            Therefore, $\nu$ is a measure.
        \end{itemize}
        \end{proof}
        \begin{definition}[]{}
            We say $\nu$ is absolutely continuous with respect to $\mu$ (denoted $\nu \ll \mu$) if for any $A \in \mathcal{F}$ such that $\mu(A) = 0$, then $\nu(A) = 0$.
            \end{definition}
            
            \begin{example}[]{}
            If $\nu(A) = \int_A f \, d\mu$ for some non-negative Borel function $f$, then $\mu(A) = 0 \Rightarrow \nu(A) = \int_A f \, d\mu = 0$. Therefore, $\nu \ll \mu$.
            \end{example}
            \begin{example}[Lebesgue Measure Equivalence]{}
                $M_{Leb} \ll 2M_{Leb}$ and $2M_{Leb} \ll M_{Leb}$
                \end{example}
                \begin{definition}[Lebesgue Measure]{def:lebesgue-measure}
                    The Lebesgue measure, denoted by $m$ or $\lambda$, is a complete measure on the $\sigma$-algebra of Lebesgue measurable subsets of $\mathbb{R}^n$ that satisfies:
                    \begin{enumerate}
                        \item (Normalization) The measure of the unit cube is 1: $m([0,1]^n) = 1$.
                        \item (Translation invariance) For any measurable set $E$ and any point $x \in \mathbb{R}^n$, $m(E+x) = m(E)$, where $E+x = \{y+x : y \in E\}$.
                        \item (Countable additivity) For any countable collection $\{E_i\}_{i=1}^{\infty}$ of pairwise disjoint measurable sets, $m(\cup_{i=1}^{\infty} E_i) = \sum_{i=1}^{\infty} m(E_i)$.
                    \end{enumerate}
                    \end{definition}
                    
                    \begin{theorem}[Properties of Lebesgue Measure]{thm:lebesgue-properties}
                    The Lebesgue measure has the following properties:
                    \begin{enumerate}
                        \item For an interval $[a,b] \subset \mathbb{R}$, $m([a,b]) = b-a$.
                        \item More generally, for a rectangle $[a_1,b_1] \times \cdots \times [a_n,b_n] \subset \mathbb{R}^n$, $m([a_1,b_1] \times \cdots \times [a_n,b_n]) = \prod_{i=1}^{n} (b_i-a_i)$.
                        \item For any $c \in \mathbb{R}$ and any measurable set $E \subset \mathbb{R}^n$, $m(cE) = |c|^n m(E)$, where $cE = \{cx : x \in E\}$.
                        \item There exist subsets of $\mathbb{R}$ that are not Lebesgue measurable.
                    \end{enumerate}
                    \end{theorem}
                \begin{definition}[Equivalent Measures]{}
                If $\mu \ll \nu$ and $\nu \ll \mu$, then we say $\mu$ and $\nu$ are equivalent (denoted $\mu \sim \nu$).
                \end{definition}
                
                \begin{example}[Dirac Measures and Counting Measure]{}
                Consider $(\mathbb{N}, \mathcal{P}(\mathbb{N}))$: Dirac measure. For each $k \in \mathbb{N}$, define $\mu_k$ such that 
                \[
                \mu_k(A) = 
                \begin{cases} 
                1 & \text{if } k \in A \\
                0 & \text{else}
                \end{cases}
                \]
                
                Counting measure: $\nu(A) = \sum_{k \in \mathbb{N}} \mu_k(A)$ (= "number of elements in $A$").
                
                Therefore, $\mu_k \ll \nu$ for all $k \in \mathbb{N}$, but $\nu \not\ll \mu_k$.
                \end{example}
                \begin{theorem}[Radon-Nikodym Theorem]{thm:radon-nikodym}
                    If $\mu, \nu$ are $\sigma$-finite measures, and $\nu \ll \mu$, then there exists a Borel function $f$, such that
                    $$\forall A \in \mathcal{F}, \quad \nu(A) = \int_A f \, d\mu.$$
                    \end{theorem}
                    
                    \begin{proposition}[Equivalent Characterization of Absolute Continuity]{prop:abs-cont-equiv}
                        $\nu \ll \mu \Leftrightarrow \forall \varepsilon > 0, \exists \delta > 0, \text{ s.t. } \forall A \in \mathcal{F}, \text{ with } \mu(A) < \delta \text{ we have } \nu(A) < \varepsilon$.
                        \end{proposition}
                        
                        \begin{proof}
                        We will prove both directions of the equivalence.
                        
                        \textbf{($\Rightarrow$) Necessity:} Suppose $\nu \ll \mu$. We need to show that for any $\varepsilon > 0$, there exists $\delta > 0$ such that for all $A \in \mathcal{F}$ with $\mu(A) < \delta$, we have $\nu(A) < \varepsilon$.
                        
                        We will prove this by contradiction. Suppose, contrary to our claim, that there exists some $\varepsilon > 0$ such that for all $n \in \mathbb{N}$, there exists $A_n \in \mathcal{F}$ with $\mu(A_n) < \frac{1}{2^n}$ but $\nu(A_n) \geq \varepsilon$.
                        
                        Let us define $B_k = \bigcup_{n=k}^{\infty} A_n$ for each $k \in \mathbb{N}$. Then $\{B_k\}_{k=1}^{\infty}$ forms a decreasing sequence of sets whose limit is $\lim_{k\to\infty} B_k = \bigcap_{k=1}^{\infty} B_k = \liminf_{n\to\infty} A_n$.
                        
                        For each $k$, we can estimate:
                        \begin{align*}
                        \mu(B_k) &\leq \sum_{n=k}^{\infty} \mu(A_n) \\
                        &< \sum_{n=k}^{\infty} \frac{1}{2^n} \\
                        &= \frac{1}{2^{k-1}}
                        \end{align*}
                        
                        Since $A_k \subseteq B_k$ for each $k$, and $\nu(A_k) \geq \varepsilon$, it follows that $\nu(B_k) \geq \varepsilon$ for all $k$.
                        
                        Now, let $B = \bigcap_{k=1}^{\infty} B_k$. By the continuity of measure for decreasing sequences:
                        \begin{align*}
                        \mu(B) &= \lim_{k\to\infty} \mu(B_k) \\
                        &\leq \lim_{k\to\infty} \frac{1}{2^{k-1}} \\
                        &= 0
                        \end{align*}
                        
                        Similarly, by the continuity of measure:
                        \begin{align*}
                        \nu(B) &= \lim_{k\to\infty} \nu(B_k) \\
                        &\geq \varepsilon > 0
                        \end{align*}
                        
                        This gives us a set $B$ with $\mu(B) = 0$ but $\nu(B) \geq \varepsilon > 0$, which contradicts our assumption that $\nu \ll \mu$. Therefore, our original claim must be true.
                        
                        \textbf{($\Leftarrow$) Sufficiency:} Suppose that for any $\varepsilon > 0$, there exists $\delta > 0$ such that for all $A \in \mathcal{F}$ with $\mu(A) < \delta$, we have $\nu(A) < \varepsilon$.
                        
                        We need to show that $\nu \ll \mu$, that is, for any $A \in \mathcal{F}$ with $\mu(A) = 0$, we have $\nu(A) = 0$.
                        
                        Let $A \in \mathcal{F}$ with $\mu(A) = 0$. For any $\varepsilon > 0$, by our assumption, there exists a $\delta > 0$ such that for all $E \in \mathcal{F}$ with $\mu(E) < \delta$, we have $\nu(E) < \varepsilon$.
                        
                        Since $\mu(A) = 0 < \delta$, it follows that $\nu(A) < \varepsilon$. But this is true for any $\varepsilon > 0$, no matter how small. Therefore, $\nu(A) = 0$, which proves that $\nu \ll \mu$.
                        \end{proof}
\section{Lecture 13 (03-12)}
\begin{definition}[Radon-Nikodym Derivative]{def:rn-derivative}
    If $\nu(A) = \int_A f \, d\mu$ for some nonnegative Borel measurable function $f$, then $f$ is the Radon-Nikodym derivative of $\nu$ with respect to $\mu$.
    \end{definition}
    
    \begin{definition}[Absolute Continuity]{def:absolute-continuity}
    $\nu \ll \mu$ if and only if $\forall A \in \mathcal{F}$ such that $\mu(A) = 0 \Rightarrow \nu(A) = 0$.
    \end{definition}
    
    \begin{theorem}[Radon-Nikodym Theorem]{thm:radon-nikodym}
    If $\mu, \nu$ are $\sigma$-finite measures, then $\nu \ll \mu \Leftrightarrow \nu(A) = \int_A f \, d\mu, \forall A \in \mathcal{F}$ for some nonnegative measurable function $f$.
    \end{theorem}
    
    \begin{definition}[Equivalent Measures]{def:equivalent-measures}
    $\mu \sim \nu$ if and only if $\mu \ll \nu$ and $\nu \ll \mu$.
    \end{definition}
    
    \begin{proposition}[Continuous Random Variables]{prop:continuous-rv}
    A continuous random variable $X$ satisfies $\mathbb{P}_X(A) = \mathbb{P}(X \in A), \forall A \in \mathcal{B}(\mathbb{R})$.
    
    $\mathbb{P}_X \ll m \Leftrightarrow \exists f$ such that $\mathbb{P}_X(A) = \int_A f(x) \, dm$, where $f$ is the probability density function.
    \end{proposition}
    
    \begin{remark}[]{}
    We often write $\int_A f \, dm = \int_A f(x) \, dx$ when $m$ is the Lebesgue measure.
    \end{remark}
    \begin{example}[Exponential Distribution]{ex:exponential}
        Consider the measure space $(\mathbb{R}, \mathcal{B}(\mathbb{R}), m)$ where $m$ is the Lebesgue measure. Define the function:
        \begin{align*}
        g(x) = \lambda e^{-\lambda x} \mathbf{1}_{[0,+\infty)}(x), \quad \lambda > 0
        \end{align*}
        
        Then for every $B \in \mathcal{B}(\mathbb{R})$, 
        \begin{align*}
        \mathbb{P}_X(B) = \int_B g(x) \, dx
        \end{align*}
        defines a measure.
        
        To verify this is a probability measure, we check:
        \begin{align*}
        \mathbb{P}_X(\mathbb{R}) &= \int_{-\infty}^{+\infty} \lambda e^{-\lambda x} \mathbf{1}_{[0,+\infty)}(x) \, dx \\
        &= \int_0^{+\infty} \lambda e^{-\lambda x} \, dx \\
        &= \left[ -e^{-\lambda x} \right]_0^{+\infty} \\
        &= 0 - (-1) = 1
        \end{align*}
        Therefore, $\mathbb{P}_X$ defines a probability measure.
        
        This is the \textbf{exponential distribution} with parameter $\lambda$, denoted as $X \sim \text{Exp}(\lambda)$.
        
        \textbf{Fact:} If $X \sim \text{Exp}(\lambda)$, then:
        \begin{align*}
        \mathbb{E}[X] &= \int_0^{+\infty} x \lambda e^{-\lambda x} \, dx \\
        &= -\int_0^{+\infty} x \, d(e^{-\lambda x}) \\
        &= -\left[ x e^{-\lambda x} \right]_0^{+\infty} + \int_0^{+\infty} e^{-\lambda x} \, dx \\
        &= 0 - 0 + \frac{1}{\lambda} \left[ -e^{-\lambda x} \right]_0^{+\infty} \\
        &= \frac{1}{\lambda}
        \end{align*}
        
        Also:
        \begin{align*}
        \mathbb{E}[X^2] &= \int_0^{+\infty} x^2 \lambda e^{-\lambda x} \, dx \\
        &= \ldots \\
        &= \frac{2}{\lambda^2}
        \end{align*}
        \end{example}
        \begin{lemma}[Standardization of Normal Distribution]{lem:normal-standardization}
            If $X \sim N(\mu, \sigma^2)$, then $Y = \frac{X-\mu}{\sigma}$ is $N(0,1)$ (standard normal).
            \end{lemma}
            
            \begin{proof}
            $$\mathbb{P}(Y \leq a) = \mathbb{P}\left(X \leq \mu + a\sigma\right) = \int_{-\infty}^{\mu+a\sigma} \frac{1}{\sqrt{2\pi}\sigma} e^{-\frac{(x-\mu)^2}{2\sigma^2}} \, dx$$
            \end{proof}
            
            Making the substitution $y = \frac{x-\mu}{\sigma}$, we get:
            \begin{align*}
            \mathbb{P}(Y \leq a) &= \int_{-\infty}^{a} \frac{1}{\sqrt{2\pi}} e^{-\frac{y^2}{2}} \, dy
            \end{align*}
            which is the CDF of a standard normal distribution.
            
            \begin{lemma}[Moment Generating Function of Standard Normal]{lem:mgf-standard-normal}
            If $X \sim N(0,1)$, then $\mathbb{E}[e^{tX}] = e^{\frac{1}{2}t^2}$.
            \end{lemma}
            
            \begin{proof}
            \begin{align*}
            \mathbb{E}[e^{tX}] &= \int_{-\infty}^{+\infty} e^{tx} \cdot \frac{1}{\sqrt{2\pi}} e^{-\frac{x^2}{2}} \, dx \\
            &= \frac{1}{\sqrt{2\pi}} \int_{-\infty}^{+\infty} e^{-\frac{1}{2}(x^2-2tx)} \, dx \\
            &= \frac{1}{\sqrt{2\pi}} \int_{-\infty}^{+\infty} e^{-\frac{(x-t)^2}{2}} e^{\frac{t^2}{2}} \, dx \\
            &= e^{\frac{t^2}{2}} \cdot \frac{1}{\sqrt{2\pi}} \int_{-\infty}^{+\infty} e^{-\frac{(x-t)^2}{2}} \, dx \\
            &= e^{\frac{t^2}{2}} \cdot 1 \\
            &= e^{\frac{t^2}{2}}
            \end{align*}
            where in the fourth line we used the substitution $y = x - t$.
            \end{proof}
            
            \begin{remark}[Applications of MGF]{rem:mgf-applications}
                (1) If $\mathbb{E}[e^{tX}] < +\infty$, then:
                \begin{align*}
                \mathbb{E}[e^{tX}] &= \mathbb{E}\left[\sum_{k=0}^{\infty} \frac{t^k}{k!}X^k\right] \\
                &= \sum_{k=0}^{\infty} \frac{t^k}{k!}\mathbb{E}[X^k]
                \end{align*}
                Therefore, the moment generating function determines all moments of the random variable.
                
                (2) Let $X$ be a continuous random variable with density function $f$. Then:
                \begin{align*}
                \mathbb{E}[e^{tX}] = \int_{-\infty}^{+\infty} e^{tx}f(x)\,dx = \mathcal{L}[f]
                \end{align*}
                where $\mathcal{L}[f]$ denotes the Laplace transform of $f$.
                
                (3) Fact: If $M_X(t) = \mathbb{E}[e^{tX}]$ converges absolutely in a neighborhood of 0, then it uniquely determines the probability density function $f$.
                
                (4) If $X \sim N(\mu,\sigma^2)$, compute $\mathbb{E}[e^{tX}]$.
                \end{remark}
\begin{proposition}[Lack of memory]{}
If $ X\sim Exp(\lambda) $, Then $$
    P(X>s+t|X>s)=P(X>t)
$$  
\end{proposition}   
\begin{proof}
    \begin{align*}
    \mathbb{P}[X > t + s | X > s] &= \frac{\mathbb{P}[X > t + s, X > s]}{\mathbb{P}[X > s]} \\
    &= \frac{\mathbb{P}[X > t + s]}{\mathbb{P}[X > s]} \\
    &= \frac{\int_{t+s}^{+\infty} \lambda e^{-\lambda x} \, dx}{\int_{s}^{+\infty} \lambda e^{-\lambda x} \, dx} \\
    &= \frac{e^{-\lambda(t+s)}}{e^{-\lambda s}} \\
    &= e^{-\lambda t} \\
    &= \mathbb{P}[X > t]
    \end{align*}
    \end{proof}
    
    \begin{example}[Probability Density Function of Normal Distribution]{ex:normal-pdf}
    The probability density function of a normal distribution is given by:
    \begin{align*}
    f(x) = \frac{1}{\sqrt{2\pi}\sigma} e^{-\frac{(x-\mu)^2}{2\sigma^2}}, \mu \in \mathbb{R}, \sigma > 0
    \end{align*}
    
    This is also known as the Gaussian distribution, denoted as $N(\mu, \sigma^2)$.
    
    Special case: When $\mu = 0$ and $\sigma = 1$, we have:
    \begin{align*}
    f(x) = \frac{1}{\sqrt{2\pi}} e^{-\frac{x^2}{2}}
    \end{align*}
    which is called the standard normal distribution, denoted as $N(0,1)$.
    \end{example}
    \begin{proposition}[Properties of Normal Distribution]{}
        \begin{enumerate}
        \item \textbf{Normalization:} Verify that $\int_{\mathbb{R}} f(x) \, dx = 1$, since 
        \begin{align*}
        \int_{-\infty}^{+\infty} e^{-\frac{x^2}{2}} \, dx = \sqrt{2\pi}
        \end{align*}
        
        \item \textbf{Mean and Variance:} If $X \sim N(\mu, \sigma^2)$, then $\mathbb{E}[X] = \mu$ and $\operatorname{Var}(X) = \sigma^2$.
        
        \item \textbf{Special Case - Standard Normal Distribution:} If $X \sim N(0,1)$, then:
        \begin{align*}
        \mathbb{E}[X] &= \int_{-\infty}^{+\infty} x \cdot \frac{1}{\sqrt{2\pi}} e^{-\frac{x^2}{2}} \, dx = 0 \\
        \operatorname{Var}(X) &= \mathbb{E}[X^2] = \int_{-\infty}^{+\infty} x^2 \cdot \frac{1}{\sqrt{2\pi}} e^{-\frac{x^2}{2}} \, dx \\
        &= \frac{1}{\sqrt{2\pi}} \frac{d}{d\lambda} \left[ \int_{-\infty}^{+\infty} e^{-\lambda x^2} \, dx \right]_{\lambda=\frac{1}{2}} \\
        &= \frac{1}{\sqrt{2\pi}} \frac{d}{d\lambda} \left[ \sqrt{\frac{\pi}{\lambda}} \right]_{\lambda=\frac{1}{2}} \\
        &= \frac{1}{\sqrt{2\pi}} \cdot \left( -\frac{1}{2} \right) \cdot \frac{d}{d\lambda} \left[ \lambda^{-\frac{1}{2}} \right]_{\lambda=\frac{1}{2}} \cdot \sqrt{\pi} \\
        &= \frac{1}{\sqrt{2\pi}} \cdot \left( -\frac{1}{2} \right) \cdot \left( -\frac{1}{2} \right) \cdot \lambda^{-\frac{3}{2}} \cdot \sqrt{\pi} \Big|_{\lambda=\frac{1}{2}} \\
        &= \frac{1}{4} \cdot \frac{\sqrt{2\pi}}{\sqrt{2\pi}} \cdot 2\sqrt{2} \\
        &= 1
        \end{align*}
        \end{enumerate}
        \end{proposition}
        \begin{example}[Cauchy Distribution]{ex:cauchy}
            Let $f(x) = \frac{1}{\pi(1+x^2)}, \quad x \in \mathbb{R}$.
            
            \begin{align*}
            \int_{-\infty}^{+\infty} f(x) \, dx &= \int_{-\infty}^{+\infty} \frac{1}{\pi(1+x^2)} \, dx \\
            &= \frac{1}{\pi} \left[ \arctan(x) \right]_{-\infty}^{+\infty} \\
            &= \frac{1}{\pi} \cdot \pi = 1
            \end{align*}
            
            This defines a random variable $Y \sim \text{Cauchy}(0,1)$.
            
            \textbf{Fact:} $\mathbb{E}[e^{tX}] = \int \frac{e^{tx}}{\pi(1+x^2)} \, dx$ converges only for $t = 0$.
            
            The expected value is also divergent:
            \begin{align*}
            \mathbb{E}[X] = \int \frac{x}{\pi(1+x^2)} \, dx \quad \text{diverges}
            \end{align*}
            
            \textbf{Characteristic Function:} While the moment generating function doesn't exist for the Cauchy distribution, the characteristic function is well-defined:
            \begin{align*}
            \phi(t) = \mathbb{E}[e^{itX}] = \int e^{itx} \frac{1}{\pi(1+x^2)} \, dx
            \end{align*}
            \end{example}
\section{Recitation (03-14)}
\subsection*{Problem 1}
We know that $ X\sim Exp(\lambda) $,Then $ P(X>t+s|X> s)=P(X>t) $
\\Identify all distribution that satisfy the lack of memory property.
\begin{proof}
    We proceed in several steps to identify all distributions that satisfy the memoryless property.
    
    \textbf{Step 1:} First, we recall the definition of the memoryless property. A random variable $X$ has the memoryless property if
    \begin{align}
    \mathbb{P}(X > t+s | X > s) = \mathbb{P}(X > t) \quad \forall s, t \geq 0 \tag{1}
    \end{align}
    
    We know that if $X \sim \text{Exp}(\lambda)$ (i.e., $X$ follows an exponential distribution with parameter $\lambda$), then $X$ satisfies property (1). We want to prove that exponential distributions are the only ones with this property.
    
    \textbf{Step 2:} Let $G(t) = \mathbb{P}(X > t)$ be the survival function of $X$. Using property (1), we have:
    \begin{align*}
    \mathbb{P}(X > t+s | X > s) &= \mathbb{P}(X > t) \\
    \frac{\mathbb{P}(X > t+s, X > s)}{\mathbb{P}(X > s)} &= \mathbb{P}(X > t) \\
    \frac{\mathbb{P}(X > t+s)}{\mathbb{P}(X > s)} &= \mathbb{P}(X > t) \\
    \frac{G(t+s)}{G(s)} &= G(t)
    \end{align*}
    
    Rearranging, we get the functional equation:
    \begin{align}
    G(t+s) = G(t) \cdot G(s) \quad \forall s, t \geq 0 \tag{2}
    \end{align}
    
    \textbf{Step 3:} Define $H(t) = -\log G(t)$. Taking the logarithm of both sides of equation (2):
    \begin{align}
    \log G(t+s) &= \log G(t) + \log G(s) \\
    -H(t+s) &= -H(t) - H(s) \\
    H(t+s) &= H(t) + H(s) \quad \forall s, t \geq 0 \tag{3}
    \end{align}
    
    This is the Cauchy functional equation. We will now prove that any continuous solution to this equation must have the form $H(t) = \lambda t$ for some constant $\lambda > 0$.
    
    \textbf{Step 4:} We claim that there exists $\lambda$ such that $H(t) = \lambda t$ for all $t \geq 0$. We prove this in several sub-steps:
    
    (a) For integer values: Let $\lambda = H(1)$. For any integer $k \in \mathbb{Z}^+$, we have:
    \begin{align*}
    H(k) &= H(1 + 1 + \ldots + 1) \quad \text{($k$ times)} \\
    &= H(1) + H(1) + \ldots + H(1) \quad \text{(using equation (3) repeatedly)} \\
    &= k \cdot H(1) = \lambda k
    \end{align*}
    
    (b) For rational values: For any rational number $\frac{p}{q}$ where $p, q \in \mathbb{Z}^+$, using equation (3), we have:
    \begin{align*}
    H\left(\frac{p}{q} \cdot q\right) &= q \cdot H\left(\frac{p}{q}\right) \\
    H(p) &= q \cdot H\left(\frac{p}{q}\right) \\
    \lambda p &= q \cdot H\left(\frac{p}{q}\right) \\
    H\left(\frac{p}{q}\right) &= \lambda \cdot \frac{p}{q}
    \end{align*}
    
    (c) For all real values: By the continuity of $H$ and the density of rational numbers in $\mathbb{R}$, for any real number $t \geq 0$, there exists a sequence of rational numbers $\{t_n\}$ such that $t_n \to t$. By the continuity of $H$, we have:
    \begin{align*}
    H(t) = \lim_{n \to \infty} H(t_n) = \lim_{n \to \infty} \lambda t_n = \lambda t
    \end{align*}
    
    Therefore, $H(t) = \lambda t$ for all $t \geq 0$.
    
    \textbf{Step 5:} Since $H(t) = -\log G(t) = \lambda t$, we have:
    \begin{align*}
    G(t) = e^{-\lambda t}, \quad \lambda > 0
    \end{align*}
    
    This is exactly the survival function of an exponential distribution with parameter $\lambda$. Note that $\lambda$ must be positive since $G(t)$ is a decreasing function of $t$ (as $t$ increases, the probability of surviving beyond $t$ decreases).
    
    \textbf{Step 6:} Finally, the probability density function of $X$ is the negative derivative of the survival function:
    \begin{align*}
    f(t) = -\frac{d}{dt}G(t) = -\frac{d}{dt}e^{-\lambda t} = \lambda e^{-\lambda t}, \quad t \geq 0
    \end{align*}
    
    This is the probability density function of an exponential distribution with parameter $\lambda$.
    
    In conclusion, the only continuous probability distributions that satisfy the memoryless property are exponential distributions.
    \end{proof}
\subsection*{Problem 2}
$ (\Omega,F,\mu) $ measure space, X is a integrable borel function, Y is an  simple function 
\\Show that (1)$$
    \exists A\in F ,\mu(A)<+\infty ,s.t. \int_\Omega Y d\mu=\int_A Y d\mu
$$ 
(2) $$
    \exists \epsilon>0, \exists A_\epsilon\in F, \mu(A_\epsilon)<+\infty, s.t. |\int_{A_\epsilon} X d \mu-\int_\Omega Xd\mu|<\epsilon
$$ 
\begin{proof}
    \textbf{Part 1:} Let $Y$ be a simple function, which can be written as
    \begin{align*}
        Y = \sum_{i=1}^n a_i 1_{A_i}
    \end{align*}
    where $a_i \in \mathbb{R}$, $A_i \cap A_j = \emptyset$ for $i \neq j$, and $a_i \neq 0$. Let $A = \bigcup_{i=1}^n A_i$. 
    
    Since $Y$ is zero outside $A$, we have
    \begin{align*}
        \int_\Omega Y \, d\mu = \int_A Y \, d\mu
    \end{align*}
    
    Also, $\mu(A) = \sum_{i=1}^n \mu(A_i) < +\infty$ since each $A_i$ has finite measure.
    
    \textbf{Part 2:} Suppose that $X \geq 0$ is a non-negative Borel function. Since $X$ is integrable, there exists a sequence of simple functions $\{Y_n\}$ such that $Y_n \uparrow X$ pointwise and $\lim_{n \to \infty} \int_\Omega Y_n \, d\mu = \int_\Omega X \, d\mu$.
    
    For any $\epsilon > 0$, there exists $N \in \mathbb{N}$ such that
    \begin{align*}
        \left|\int_\Omega Y_N \, d\mu - \int_\Omega X \, d\mu\right| < \epsilon
    \end{align*}
    
    From Part 1, there exists $A_N \in \mathcal{F}$ with $\mu(A_N) < +\infty$ such that
    \begin{align*}
        \int_\Omega Y_N \, d\mu = \int_{A_N} Y_N \, d\mu
    \end{align*}
    
    This implies
    \begin{align*}
        \left|\int_{A_N} Y_N \, d\mu - \int_\Omega X \, d\mu\right| < \epsilon
    \end{align*}
    
    Since $0 \leq Y_N \leq X$ and $Y_N$ approximates $X$ on $A_N$, we have
    \begin{align*}
        \int_{A_N} X \, d\mu - \epsilon \leq \int_{A_N} Y_N \, d\mu \leq \int_{A_N} X \, d\mu
    \end{align*}
    
    Which leads to
    \begin{align*}
        \left|\int_{A_N} X \, d\mu - \int_\Omega X \, d\mu\right| < \epsilon
    \end{align*}
    
    Now, suppose $X$ is an arbitrary integrable Borel function. Let $X = X^+ - X^-$, where $X^+ = \max(X, 0)$ and $X^- = \max(-X, 0)$ are the positive and negative parts of $X$, respectively.
    
    By applying the above result to $X^+$ and $X^-$, we can find $A^+, A^- \in \mathcal{F}$ with $\mu(A^+), \mu(A^-) < +\infty$ such that
    \begin{align*}
        \left|\int_{A^+} X^+ \, d\mu - \int_\Omega X^+ \, d\mu\right| < \frac{\epsilon}{2}\\
        \left|\int_{A^-} X^- \, d\mu - \int_\Omega X^- \, d\mu\right| < \frac{\epsilon}{2}
    \end{align*}
    
    Let $A = A^+ \cup A^-$. Then $\mu(A) < +\infty$ and
    \begin{align*}
        \left|\int_A X \, d\mu - \int_\Omega X \, d\mu\right| &= \left|\int_A (X^+ - X^-) \, d\mu - \int_\Omega (X^+ - X^-) \, d\mu\right|\\
        &\leq \left|\int_A X^+ \, d\mu - \int_\Omega X^+ \, d\mu\right| + \left|\int_A X^- \, d\mu - \int_\Omega X^- \, d\mu\right|\\
        &< \frac{\epsilon}{2} + \frac{\epsilon}{2} = \epsilon
    \end{align*}
    
    Therefore, for any integrable Borel function $X$ and any $\epsilon > 0$, there exists $A_\epsilon \in \mathcal{F}$ with $\mu(A_\epsilon) < +\infty$ such that
    \begin{align*}
        \left|\int_{A_\epsilon} X \, d\mu - \int_\Omega X \, d\mu\right| < \epsilon
    \end{align*}
    \end{proof}
\subsection*{Problem 3}
    Let $X$ be a continuous non-negative random variable with $\mathbb{E}[X] < \infty$. Then:
    \begin{enumerate}
        \item $\displaystyle \mathbb{E}[X] = \int_0^{+\infty} \mathbb{P}[X > y] \, dy - \int_0^{+\infty} \mathbb{P}[X <- y] \, dy$
        \item $\displaystyle \lim_{y \to +\infty} y \cdot \mathbb{P}[X > y] = 0$
    \end{enumerate}
    \begin{proof}
        Let's prove each part separately.
        
        \textbf{Part (1):} We begin with the definition of expectation. For a continuous random variable $X$ with probability density function $f_X(x)$:
        \begin{align*}
        \mathbb{E}[X] &= \int_{-\infty}^{+\infty} x \cdot f_X(x) \, dx
        \end{align*}
        
        We can decompose $X$ into its positive and negative parts:
        \begin{align*}
        \mathbb{E}[X] &= \mathbb{E}[X^+] - \mathbb{E}[X^-]
        \end{align*}
        where $X^+ = \max(X, 0)$ and $X^- = \max(-X, 0)$.
        
        In class, we showed that for a non-negative random variable $Z$:
        \begin{align*}
        \mathbb{E}[Z] = \int_0^{+\infty} \mathbb{P}[Z > y] \, dy
        \end{align*}
        
        Applying this to $X^+$ and $X^-$:
        \begin{align*}
        \mathbb{E}[X^+] &= \int_0^{+\infty} \mathbb{P}[X^+ > y] \, dy = \int_0^{+\infty} \mathbb{P}[X > y] \, dy\\
        \mathbb{E}[X^-] &= \int_0^{+\infty} \mathbb{P}[X^- > y] \, dy = \int_0^{+\infty} \mathbb{P}[-X > y] \, dy = \int_0^{+\infty} \mathbb{P}[X < -y] \, dy
        \end{align*}
        
        Therefore:
        \begin{align*}
        \mathbb{E}[X] &= \mathbb{E}[X^+] - \mathbb{E}[X^-]\\
        &= \int_0^{+\infty} \mathbb{P}[X > y] \, dy - \int_0^{+\infty} \mathbb{P}[X < -y] \, dy
        \end{align*}
        
        \textbf{Part (2):} We need to prove that $\lim_{y \to +\infty} y \cdot \mathbb{P}[X > y] = 0$.
        
        For any $y > 0$, we have:
        \begin{align*}
        y \cdot \mathbb{P}[X > y] &= y \cdot \int_y^{+\infty} f_X(x) \, dx\\
        &\leq \int_y^{+\infty} x \cdot f_X(x) \, dx\\
        &= \mathbb{E}[X \cdot \mathbf{1}_{\{X > y\}}]
        \end{align*}
        
        We claim that $\mathbb{E}[X \cdot \mathbf{1}_{\{X > y\}}] \to 0$ as $y \to +\infty$.
        
        This follows because $X$ is integrable ($\mathbb{E}[X] < \infty$), and as $y$ increases, the set $\{X > y\}$ becomes smaller. By the dominated convergence theorem:
        \begin{align*}
        \lim_{y \to +\infty} \mathbb{E}[X \cdot \mathbf{1}_{\{X > y\}}] = 0
        \end{align*}
        
        Therefore:
        \begin{align*}
        \lim_{y \to +\infty} y \cdot \mathbb{P}[X > y] = 0
        \end{align*}
        
        This result can also be approached using approximation by simple functions. Additionally, we can leverage the following fact from measure theory: In a measure space $(\Omega, \mathcal{F}, \mu)$, if $f$ is integrable, then for any $\varepsilon > 0$, there exists $\delta > 0$ such that for any $A \in \mathcal{F}$ with $\mu(A) < \delta$, we have $\left|\int_A f \, d\mu\right| < \varepsilon$.
        \end{proof}
\section{Lecture 14 (03-17)}
\begin{theorem}[Chain Rule for Radon-Nikodym Derivatives]{}
    Suppose $\nu \ll \mu$ (i.e., $\nu$ is absolutely continuous with respect to $\mu$). Then a function $f$ is integrable with respect to $\nu$ if and only if $f \cdot \frac{d\nu}{d\mu}$ is integrable with respect to $\mu$. Furthermore, in this case:
    \[
    \int f \, d\nu = \int f \cdot \frac{d\nu}{d\mu} \, d\mu
    \]
    where $\frac{d\nu}{d\mu}$ denotes the Radon-Nikodym derivative of $\nu$ with respect to $\mu$.
    \end{theorem}
    
    \begin{remark}[Probabilistic Interpretation]{}
    If $Q \ll P$ are probability measures, and $X$ is a $Q$-integrable random variable, then:
    \[
    \mathbb{E}^Q[X] = \mathbb{E}^P\left[X \cdot \frac{dQ}{dP}\right]
    \]
    where $\frac{dQ}{dP}$ is the Radon-Nikodym derivative (likelihood ratio) of $Q$ with respect to $P$.
    \end{remark}
    
    \begin{proof}
    We proceed in several steps:
    
    \textbf{Step 1:} First, we verify the theorem for indicator functions. For any measurable set $A$:
    \begin{align*}
    \int 1_A \, d\nu &= \nu(A)\\
    &= \int_A \frac{d\nu}{d\mu} \, d\mu \quad \text{(by definition of Radon-Nikodym derivative)}\\
    &= \int 1_A \cdot \frac{d\nu}{d\mu} \, d\mu
    \end{align*}
    
    \textbf{Step 2:} By linearity, we extend the result to simple functions. Let $\varphi = \sum_{i=1}^n a_i 1_{A_i}$, where $A_i \cap A_j = \emptyset$ for $i \neq j$. Then:
    \begin{align*}
    \int \varphi \, d\nu &= \sum_{i=1}^n a_i \int 1_{A_i} \, d\nu\\
    &= \sum_{i=1}^n a_i \int 1_{A_i} \cdot \frac{d\nu}{d\mu} \, d\mu \quad \text{(by Step 1)}\\
    &= \int \sum_{i=1}^n a_i 1_{A_i} \cdot \frac{d\nu}{d\mu} \, d\mu\\
    &= \int \varphi \cdot \frac{d\nu}{d\mu} \, d\mu
    \end{align*}
    
    \textbf{Step 3:} For a non-negative Borel function $g \geq 0$, there exists a sequence $\{\varphi_n\}_{n \geq 1}$ of simple functions such that $\varphi_n \uparrow g$ pointwise. Applying the Monotone Convergence Theorem (MCT):
    \begin{align*}
    \int g \, d\nu &= \lim_{n \to \infty} \int \varphi_n \, d\nu \quad \text{(by MCT)}\\
    &= \lim_{n \to \infty} \int \varphi_n \cdot \frac{d\nu}{d\mu} \, d\mu \quad \text{(by Step 2)}\\
    &= \int \lim_{n \to \infty} \left(\varphi_n \cdot \frac{d\nu}{d\mu}\right) \, d\mu \quad \text{(by MCT)}\\
    &= \int g \cdot \frac{d\nu}{d\mu} \, d\mu
    \end{align*}
    
    \textbf{Step 4:} For a general Borel function $g$, we decompose it as $g = g^+ - g^-$, where $g^+ = \max(g, 0)$ and $g^- = \max(-g, 0)$. Applying the result from Step 3 to both $g^+$ and $g^-$, we obtain:
    \begin{align*}
    \int g \, d\nu &= \int g^+ \, d\nu - \int g^- \, d\nu\\
    &= \int g^+ \cdot \frac{d\nu}{d\mu} \, d\mu - \int g^- \cdot \frac{d\nu}{d\mu} \, d\mu\\
    &= \int (g^+ - g^-) \cdot \frac{d\nu}{d\mu} \, d\mu\\
    &= \int g \cdot \frac{d\nu}{d\mu} \, d\mu
    \end{align*}
    
    This completes the proof of the chain rule.
    \end{proof}
\begin{example}[]{}
    \begin{align*}
        &\text{eg}: (\Omega, \mathcal{F}, \mathbb{P}), \quad X \sim N(0,1). \text{ Let } \theta > 0. \quad X + \theta \sim N(\theta, 1)\\
        &\text{We can define a new prob. measure } \mathbb{Q}, \text{ s.t. under } \mathbb{Q}, \; X + \theta \sim N(0,1).\\
        &\text{Soln}: \text{ Let } \frac{d\mathbb{Q}}{d\mathbb{P}} = e^{-\theta X - \frac{1}{2}\theta^2}\\
        &\mathbb{Q} \text{ is a prob. measure: } \mathbb{Q}(\Omega) = \int_{\Omega} d\mathbb{Q} = \int_{\Omega} \frac{d\mathbb{Q}}{d\mathbb{P}}d\mathbb{P} = \int_{\Omega} e^{-\theta X - \frac{1}{2}\theta^2}d\mathbb{P} = e^{-\frac{1}{2}\theta^2}\mathbb{E}^{\mathbb{P}}[e^{-\theta X}]\\
        &\hspace{7.3cm} = e^{-\frac{1}{2}\theta^2} \cdot e^{\frac{1}{2}\theta^2} = 1 \quad \text{MGF for } N(0,1)\\
        &\text{Compute } \mathbb{E}^{\mathbb{Q}}[e^{t(X+\theta)}] = \mathbb{E}^{\mathbb{P}}\left[e^{t(X+\theta)}\frac{d\mathbb{Q}}{d\mathbb{P}}\right]\\
        &\forall t \in \mathbb{R} \hspace{1.7cm} = \mathbb{E}^{\mathbb{P}}\left[e^{t(X+\theta)}e^{-\theta X - \frac{1}{2}\theta^2}\right] = e^{t\theta - \frac{1}{2}\theta^2}\mathbb{E}^{\mathbb{P}}[e^{(t-\theta)X}]\\
        &\hspace{7.3cm} = e^{t\theta - \frac{1}{2}\theta^2}e^{\frac{1}{2}(t-\theta)^2} = e^{\frac{1}{2}t^2}\\
        &\Rightarrow X+\theta \sim^{\mathbb{Q}} N(0,1)
        \end{align*}
\end{example}
\textbf{Fact:}
\begin{align*}
    \text{ if } M(t):=\mathbb{E}[e^{tX}] \text{ converges in } t \in (-\delta,\delta) \text{ for some } \delta > 0.\\
    \text{Then } \{M(t), t \in (-\delta,\delta)\} \text{ determines the distribution of } X.
    \end{align*}
\subsection{Joint distribution}
Given probability space $ (\Omega_1, F_1, P_1), (\Omega_2, F_2, P_2)$ can define the product space $ (\Omega_1\times\Omega_2, F_1\otimes F_2, P_1\otimes P_2) $ where$$
    F_1\otimes F_2=\sigma(\{A_1\times A_2:A_1\in F_1,A_2\in F_2\})
$$  define $$
    P_1\otimes P_2(A_1\times A_2)=P_1(A_1)P_2(A_2)
$$ and extend to $ F_1\otimes F_2 $ by Caratheodory extension theorem
\begin{definition}[Joint distribution]{}
    Joint distribution of $ X,Y $ is $ F:\mathbb{R}^2\rightarrow[0,1] $ s.t.  $$
        F(x,y)=P(X\leq x,Y\leq y)
    $$
\end{definition}
\begin{definition}[]{}
If  X,Y are continuous random variables, then the joint density function $ f:\mathbb{R}^2\rightarrow[0,+\infty) $ is given by\begin{align*}{}{}
F(x,y)&=\int_{-\infty}^{x}\int_{-\infty}^{y}f(u,v)dudv\\
f(x,y)&=\frac{\partial^2 F}{\partial x\partial y}
\end{align*} 
\end{definition}
\begin{remark}[]{}
(1)
\begin{align*}{}{}
P[a< X\leq b,c<Y\leq d]&=\int_{a}^{b}\int_{c}^{d}f(u,v)dudv\\
\end{align*}
(2):by uniqueness of extension,\begin{align*}{}{}
B(R^2)=\sigma\{\text{left open right closed cubes in } R^2\}\\
P((X,Y)\in A)=\int_{A}f(u,v)dudv\text{, for every } A\in B(R^2)\\
\end{align*}
\end{remark}
\begin{align*}{}{}
    \text{May cover individual distribution from the joint distribution}\\
    F_X(x)=P(X\leq x)=P(X\leq x,Y\in R)=\int_{-\infty}^{x}\int_{-\infty}^{+\infty}f(u,v)dudv\\
    f_X(x)=\int_{-\infty}^{+\infty}f(x,v)dv,f_Y(y)=\int_{-\infty}^{+\infty}f(u,y)du
\end{align*}
\begin{remark}[]{}
if X,Y are independent$\Rightarrow F(x,y)=F_X(x)F_Y(y)\Leftrightarrow(cont.) f(x,y)=f_X(x)f_Y(y) $
\end{remark}
\begin{example}[]{}
 \begin{enumerate}[label=\circled{\arabic*}] 
\item  \begin{align*}
    &\text{If X, Y have joint density function}\\
    &f(x,y) = \frac{\alpha^x}{x!}\frac{\beta^y}{y!}e^{-\alpha-\beta} \quad \forall x,y \in \mathbb{N}.\\
    &\text{Soln: } \bullet \text{ X, Y are indep, because } f(x,y) = f_1(x)f_2(y).\\
    &\quad \bullet f_X(x) = \sum_{y \in \mathbb{N}} \frac{\alpha^x}{x!}\frac{\beta^y}{y!}e^{-\alpha-\beta} = \frac{\alpha^x}{x!}e^{-\alpha-\beta}\sum_{y \in \mathbb{N}}\frac{\beta^y}{y!} = \frac{\alpha^x}{x!}e^{-\alpha}\\
    &\quad \Rightarrow X \sim \text{Poisson}(\alpha)\\
    &\quad \text{Similarly, } f_Y(y) = \sum_{x \in \mathbb{N}} \frac{\alpha^x}{x!}\frac{\beta^y}{y!}e^{-\alpha-\beta} = \ldots \quad Y \sim \text{Poisson}(\beta).
    \end{align*}
 \item \begin{align*}
     &\text{ The joint density function of } X, Y \text{ is } f(x,y) = 
    \begin{cases}
    e^{-(x+y)} & \text{if } 0 \leq x, y < \infty\\
    0 & \text{else}
    \end{cases}\\
    &\text{Find the density funct. of } \frac{X}{Y}.\\
    \underline{\text{Soln}}: & F_{\frac{X}{Y}}(a) = P\left(\frac{X}{Y} \leq a\right) = \iint_{[x \leq ay]} f(x,y) \, dx \, dy\\
    &= \int_0^{+\infty} \int_0^{ay} e^{-(x+y)} \, dx \, dy = \int_0^{+\infty} [1-e^{-ay}]e^{-y} \, dy\\
    &= \int_0^{+\infty} (1-e^{-ay})e^{-y} \, dy = 1-\frac{1}{a+1}.\\
    &f_{\frac{X}{Y}}(a) = \frac{1}{(a+1)^2}, \quad a \geq 0
    \end{align*}
 \end{enumerate}
\end{example}
\section{Lecture 15 (03-19)}
\subsection{Joint Distribution Function and Density Function}

\begin{definition}[Joint Distribution Function]{}
For random variables $X$ and $Y$, the joint distribution function is defined as:
\begin{align*}
F(x,y) = \mathbb{P}(X \leq x, Y \leq y)
\end{align*}
\end{definition}

\begin{definition}[Joint Density Function]{}
For continuous random variables $X$ and $Y$, if the second-order partial derivative of their joint distribution function $F(x,y)$ exists, the joint density function is defined as:
\begin{align*}
f(x,y) = \frac{\partial^2 F(x,y)}{\partial x \partial y}
\end{align*}
For any measurable set $A \subset \mathbb{R}^2$, we have:
\begin{align*}
\mathbb{P}((X,Y) \in A) = \iint_A f(x,y) \,dx\,dy
\end{align*}
\end{definition}

\subsection{Distribution of Sum of Random Variables}

\begin{proposition}[Density Function of Sum]{}
Let $X$ and $Y$ be continuous random variables with joint density function $f(x,y)$. The density function of $X+Y$ is:
\begin{align*}
f_{X+Y}(z) = \int_{-\infty}^{+\infty} f(x, z-x) \,dx
\end{align*}
\end{proposition}

\begin{proof}[]{}
First, the distribution function of $X+Y$ can be represented as:
\begin{align*}
F_{X+Y}(z) &= \mathbb{P}(X+Y \leq z)\\
&= \iint_{x+y \leq z} f(x,y) \,dx\,dy\\
&= \int_{-\infty}^{+\infty}\int_{-\infty}^{z-x} f(x,y) \,dy \,dx
\end{align*}

Differentiating with respect to $z$, we obtain the density function:
\begin{align*}
f_{X+Y}(z) &= \frac{d}{dz}F_{X+Y}(z)\\
&= \frac{d}{dz}\int_{-\infty}^{+\infty}\int_{-\infty}^{z-x} f(x,y) \,dy \,dx\\
&= \int_{-\infty}^{+\infty} f(x, z-x) \,dx
\end{align*}
where we applied Leibniz's integral rule to exchange the order of integration and differentiation.
\end{proof}

\begin{example}[Sum of Independent Standard Normal Random Variables]{}
Let $X, Y \sim N(0,1)$ be independent. Prove that $X+Y \sim N(0,2)$.
\end{example}

\begin{proof}[]{}
Since $X$ and $Y$ are independent, their joint density function is the product of their individual density functions:
\begin{align*}
f(x,y) = f_X(x) \cdot f_Y(y) = \frac{1}{\sqrt{2\pi}}e^{-\frac{x^2}{2}} \cdot \frac{1}{\sqrt{2\pi}}e^{-\frac{y^2}{2}}
\end{align*}

Computing the density function of $X+Y$:
\begin{align*}
f_{X+Y}(z) &= \int_{-\infty}^{+\infty} f(x, z-x) \,dx\\
&= \int_{-\infty}^{+\infty} \frac{1}{\sqrt{2\pi}}e^{-\frac{x^2}{2}} \cdot \frac{1}{\sqrt{2\pi}}e^{-\frac{(z-x)^2}{2}} \,dx\\
&= \frac{1}{2\pi}\int_{-\infty}^{+\infty} e^{-\frac{1}{2}(x^2 + (z-x)^2)} \,dx
\end{align*}

Computing the exponent term:
\begin{align*}
x^2 + (z-x)^2 &= x^2 + z^2 - 2zx + x^2\\
&= 2x^2 - 2zx + z^2\\
&= 2(x^2 - zx + \frac{z^2}{2})\\
&= 2(x - \frac{z}{2})^2 + \frac{z^2}{2} - \frac{z^2}{2}\\
&= 2(x - \frac{z}{2})^2 + \frac{z^2}{2}
\end{align*}

Substituting back into the original integral:
\begin{align*}
f_{X+Y}(z) &= \frac{1}{2\pi}\int_{-\infty}^{+\infty} e^{-\frac{1}{2}[2(x - \frac{z}{2})^2 + \frac{z^2}{2}]} \,dx\\
&= \frac{1}{2\pi}e^{-\frac{z^2}{4}}\int_{-\infty}^{+\infty} e^{-(x - \frac{z}{2})^2} \,dx
\end{align*}

Letting $u = x - \frac{z}{2}$, we have $dx = du$, and the integration limits transform:
\begin{align*}
f_{X+Y}(z) &= \frac{1}{2\pi}e^{-\frac{z^2}{4}}\int_{-\infty}^{+\infty} e^{-u^2} \,du\\
&= \frac{1}{2\pi}e^{-\frac{z^2}{4}} \cdot \sqrt{\pi}\\
&= \frac{1}{\sqrt{2\pi \cdot 2}}e^{-\frac{z^2}{4}}
\end{align*}

This is precisely the density function of $N(0,2)$, therefore $X+Y \sim N(0,2)$.
\end{proof}

\textbf{Fact}:Sum of Independent Normal Random Variables\\
If $X \sim N(\mu_1, \sigma_1^2)$, $Y \sim N(\mu_2, \sigma_2^2)$ and $X$ and $Y$ are independent, then $X+Y \sim N(\mu_1+\mu_2, \sigma_1^2+\sigma_2^2)$.
    \begin{align*}
        \text{Soln2:} &\text{ Use that moment generating funct. } M(t) = \mathbb{E}[e^{tX}] \text{ determines the distr.}\\
        &M_{X+Y}(t) = \mathbb{E}[e^{t(X+Y)}] \stackrel{\text{ind.}}{=} \mathbb{E}[e^{tX} \cdot \mathbb{E}[e^{tY}]] = e^{\frac{1}{2}t^2} \cdot e^{\frac{1}{2}t^2} = e^{t^2}, \quad X+Y \sim N(0,2)\\
        &\text{Here we used if } X \sim N(\mu, \sigma^2), \mathbb{E}[e^{tX}] = e^{\mu t + \frac{1}{2}\sigma^2 t^2}.
        \end{align*} 
    \begin{example}[Problems with Uniform Distributions]{}
    Let $X, Y$ be independent random variables, each uniformly distributed on $[0,1]$ (i.e., $X, Y \sim \text{Uniform}[0,1]$, with $f_X(x) = \mathbf{1}_{[0,1]}(x)$).
    \end{example}
    
    \begin{problem}[]
    Compute the joint density function of $X+Y$.
    \end{problem}
    
    \begin{solution}
        For two independent random variables $X$ and $Y$, the density function of their sum $Z=X+Y$ can be calculated using the convolution formula:
\begin{align*}
f_{X+Y}(z) &= \int_{-\infty}^{+\infty} f_X(x)f_Y(z-x) \, dx\\
&= \int_{0}^{1} \mathbf{1}_{[0,1]}(x)\mathbf{1}_{[0,1]}(z-x) \, dx
\end{align*}

Since $f_X(x) = \mathbf{1}_{[0,1]}(x)$ and $f_Y(y) = \mathbf{1}_{[0,1]}(y)$, the integration region must satisfy both $0 \leq x \leq 1$ and $0 \leq z-x \leq 1$, which means $0 \leq x \leq 1$ and $z-1 \leq x \leq z$.

This can be divided into three cases:
\begin{align*}
f_{X+Y}(z) = \int_{0}^{1} \mathbf{1}_{0 \leq z-x \leq 1} \, dx = 
\begin{cases}
\int_{0}^{z} 1 \, dx = z, & \text{if } z \in [0,1]\\
\int_{z-1}^{1} 1 \, dx = 2-z, & \text{if } z \in (1,2]\\
0, & \text{otherwise}
\end{cases}
\end{align*}

Therefore, the density function of $X+Y$ is:
\begin{align*}
f_{X+Y}(z) = 
\begin{cases}
z, & \text{if } z \in [0,1]\\
2-z, & \text{if } z \in (1,2]\\
0, & \text{otherwise}
\end{cases}
\end{align*}

This is a triangular distribution, reaching its maximum value of 1 at $z=1$.

    \end{solution}
    
    \begin{problem}[]
    Let $X_1, \ldots, X_n$ be independent random variables, each uniformly distributed on $[0,1]$. Compute $F_{X_1+\ldots+X_n}(z)$ for $z \in [0,1]$.
    \end{problem}
    
    \begin{solution}
        We will use induction to prove that $F_{X_1+\ldots+X_n}(z) = \frac{z^n}{n!}$ for $z \in [0,1]$.

\textbf{Base case}: When $n=1$, $F_{X_1}(z) = z$ for $z \in [0,1]$, which is the distribution function of a uniform distribution on $[0,1]$.

\textbf{Induction hypothesis}: Assume that for $n-1$, we have $F_{X_1+\ldots+X_{n-1}}(z) = \frac{z^{n-1}}{(n-1)!}$ for $z \in [0,1]$.

\textbf{Induction step}: We need to prove that $F_{X_1+\ldots+X_n}(z) = \frac{z^n}{n!}$ for $z \in [0,1]$.

Using the convolution formula and the induction hypothesis, we can calculate:
\begin{align*}
F_n(z) &= \int_{-\infty}^{+\infty}\int_{-\infty}^{z-x} f(x)f_{n-1}(y) \, dy \, dx\\
&= \int_{-\infty}^{+\infty} f(x)F_{n-1}(z-x) \, dx
\end{align*}

For $z \in [0,1]$, substituting the known conditions:
\begin{align*}
F_n(z) &= \int_{0}^{z} \frac{(z-x)^{n-1}}{(n-1)!} \, dx\\
&= \frac{1}{(n-1)!}\int_{0}^{z} (z-x)^{n-1} \, dx
\end{align*}

Using the substitution $u = z-x$, $dx = -du$, when $x=0$ we have $u=z$, and when $x=z$ we have $u=0$:
\begin{align*}
F_n(z) &= \frac{1}{(n-1)!}\int_{z}^{0} u^{n-1} (-du)\\
&= \frac{1}{(n-1)!}\int_{0}^{z} u^{n-1} \, du\\
&= \frac{1}{(n-1)!} \cdot \frac{z^n}{n}\\
&= \frac{z^n}{n!}
\end{align*}

Therefore, we have proven that for $z \in [0,1]$, $F_{X_1+\ldots+X_n}(z) = \frac{z^n}{n!}$.

Note: This result is only valid for $z \in [0,1]$. For $z > 1$, the distribution function expression becomes more complex.

    \end{solution}
    
    \begin{problem}[]
    Let $X_1, X_2, \ldots$ be independent random variables, each uniformly distributed on $[0,1]$. Define $N = \min\{n \in \mathbb{N}: X_1 + X_2 + \ldots + X_n > 1\}$. Compute $\mathbb{E}[N]$.
    \end{problem}
    
    \begin{solution}
        We define $N = \min\{n \in \mathbb{N}: X_1 + X_2 + \ldots + X_n > 1\}$.

First, observe that the event $\{N \geq n\}$ is equivalent to the event $\{X_1 + X_2 + \ldots + X_{n-1} \leq 1\}$. This is because $N \geq n$ means that the sum of the first $n-1$ random variables is not yet sufficient to exceed 1.

Therefore:
\begin{align*}
P(N \geq n) &= P(X_1 + X_2 + \ldots + X_{n-1} \leq 1)\\
&= F_{X_1+\ldots+X_{n-1}}(1)\\
&= \frac{1^{n-1}}{(n-1)!}\\
&= \frac{1}{(n-1)!}
\end{align*}

Using the formula for the expectation of a discrete random variable $E[N] = \sum_{n=1}^{\infty} P(N \geq n)$, we have:
\begin{align*}
E[N] &= \sum_{n=1}^{\infty} P(N \geq n)\\
&= \sum_{n=1}^{\infty} \frac{1}{(n-1)!}\\
&= \sum_{n=0}^{\infty} \frac{1}{n!}\\
&= e
\end{align*}

The last step follows from the fact that $\sum_{n=0}^{\infty} \frac{1}{n!} = e$, which is the series expansion of the natural constant $e$.

Therefore, $E[N] = e \approx 2.71828...$

    \end{solution}
\section{Recitation (03-21)}
\subsection*{Problem 1}
Is is possible to have 2 biased dice, such that the sum is uniformly distributed in \{2,3,...12\}?
\subsection*{Problem 2}
Flip a fair coin, What is the expected time to see the 1-st occur of HHT?
\subsection*{Problem 3}
Let X be a non-negative r.v. show that$$
    E[X^r]=\int_{0}^{+\infty}rx^{r-1}P(X>x)dx,r>0
$$ 
\subsection*{Problem 4}
Gamma $ (n,\lambda) ,f_{n,\lambda}=\frac{\lambda^nx^{n-1}}{P(n)}e^{-\lambda x}$ ,where $ P(n)=(n-1)! $
\\Show that if $ X_1,X_2,...,X_n $ is independent $ Exp(\lambda) $ then $ X_1+X_2+\cdots+X_n\sim Gamma(n,\lambda) $ 
\section{Lecture 16 (03-24)}
\noindent\textbf{+5 Questions; Grade the best 4.}

\noindent\textbf{Bookwork Content} \quad (\textsf{$\star$ - proofs are examinable})

\begin{enumerate}
\item Def.\ of algebra: algebra generated by class of subsets. \\
Important examples from real line \& Discrete sets. (identify the algebra/$\sigma$-algebra generated by given class of sets)$^\star$

\item $\cdots$ $\sigma$-algebra, $\sigma$-algebra $- - - - -$ \\
$- - - - - - - - -$

\item Def.\ of Content. Subadditivity prop. Important examples from $\mathbb{R}$.

\item $\cdots$ Measure. Subadditivity$^\star$. Continuity from above/below.$^\star$

\item Lebesgue measure. Borel sets. examples of $\mathcal{B}(\mathbb{R})$. Lebesgue-Stieltjes measure.

\item Extension Thm.

\item Def.\ of $\pi$-system. Example of $\pi$-system that generates $\mathcal{B}(\mathbb{R})$. Uniqueness Thm. \\
Application of Uniqueness Thm.\ to show the uniqueness of Lebesgue measure.$^\star$

\item Def.\ of mble funct./r.v., Borel funct. $\sigma$-algebra generated by r.v.

\item Equivalent cond. for a funct. being Borel.$^\star$ Be able to prove certain funct. are m'ble. \\
Operation of m'ble funct.$^\star$

\item Construction of Lebesgue integral (simple $\rightarrow$ non-negative Borel $\rightarrow$ general) $\{X<a, a\in\mathbb{R}\}$

\item Monotone Conv. Thm.

\item absolute Cont. and Radon-Nikodym Derivative.
\end{enumerate}

\noindent\textbf{Common distributions:}

\noindent Bernoulli, Binomial, Poisson, Geometric, Uniform, Exponential, Gaussian 
\section{Lecture 17 (04-07)}
Random walk
\\Markov property
$$
    P[S_{n+m}|S_0,S_1,\cdots,S_m]=P[S_{n+m}=j|S_m]
$$ 
"position after m-th step does not depend on the info before m"
\\Let \begin{align*}{}{}
T_y^0&=0\\
T_y^k&=\inf\{n\geq T_y^{k-1}:S_n=y\}\\
\end{align*}
\begin{definition}[]{}
y is recurrent if $P[T_y^k<\infty]=1$ for all k
\\y is transient if $P[T_y^k<\infty]<1$ for some k
\end{definition}
\begin{remark}[]{}
If y is recurrent
$$
    P[T_y^k<\infty]=P[T_y^k<\infty|T_y^{k-1}<\infty] P[T_y^k<\infty]+P[T_y^k<\infty|T_y^{k-1}=\infty] P[T_y^k<\infty]
$$ 
\end{remark}
